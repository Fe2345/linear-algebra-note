\ifx\allfiles\undefined
\documentclass[12pt, a4paper, oneside, UTF8]{ctexbook}
\def\path{../config}
\input{../config/_config}
\begin{document}
% \input{../config/cover}
\else
\fi
%标题
\chapter{线性变换的表示与分解}
	\section{线性变换的特征值和特征向量}
		\subsection{特征值和特征向量的定义}
			\begin{defn}{线性变换的特征值、特征向量、特征子空间}{}
				设$V$是一个$F$上的线性空间,$A \in \hom(V,V)$

				那么如果$\exists \lambda \in F,\alpha \in V,\alpha \neq \mathbf{0}$,使得

				\begin{equation}
					A(\alpha )=\lambda \alpha 
				\end{equation}
				成立。

				那么我们称$\lambda $是$A$的一个特征值,$\alpha $是$A$的隶属于$\lambda $的一个特征向量

				同时我们定义:
				\begin{equation}
					V_\lambda := \{\alpha | A(\alpha )=\lambda \alpha \}
				\end{equation}
				称为$A$的属于特征值$\lambda $的特征子空间
			\end{defn}
			特征值代表着一个线性映射对一个向量的伸缩程度,这里我们要求向量不为零是因为:如果一个特征值只对于零向量成立,那么它过于平凡,而且会扰乱后续我们一些关于特征值数量的命题

			请注意:特征值必须至少拥有一个非零特征向量不等同于零向量不是它的特征向量。后面我们会看到,特征子空间是子空间,因此零向量一定是任意特征值的特征向量

			类似地,我们自然可以定义矩阵的特征值
			\begin{defn}{矩阵的特征值、特征向量}{}
				设$\symbfit{A} \in M_n(F)$

				那么如果$\exists \lambda \in F,\alpha \in F^n$,使得

				\begin{equation}
					\symbfit{A}\alpha =\lambda \alpha 
				\end{equation}
				成立。

				那么我们称$\lambda $是$\symbfit{A}$的一个特征值,$\alpha $是$\symbfit{A}$的隶属于$\lambda $的一个特征向量
			\end{defn}
		\subsection{特征值和特征向量的性质}
			我们探讨一些相关的基本性质
			\begin{para}{0}
				\point{}

					我们首先验证,特征子空间的确是子空间
					\begin{proposition}
						$\forall A \in \hom(V,V)$,$V_\lambda $是$V$的一个线性子空间
					\end{proposition}
					\begin{proof}
						取$\forall \alpha ,\beta \in V_\lambda ,k \in F$

						$A(\alpha +\beta )=\lambda \alpha +\lambda \beta =\lambda (\alpha +\beta ) \Rightarrow \alpha +\beta \in V_\lambda $

						$A(k\alpha )=k\lambda \alpha =\lambda (k\alpha )\Rightarrow k\alpha \in V_\lambda $

						于是命题得证
					\end{proof}
				\point{}
					
					从特征值的定义可以看出,特征值就像是把线性变换的一部分转换为纯量乘法。我们猜想:每一个特征值体现了映射的不同的“方向”,不同特征子空间的线性无关向量组的并也应该线性无关

					以下命题证实了这个猜想
					\begin{proposition}
						设$A \in \hom(V,V)$,$\lambda_1,\lambda_2$是$A$的两个特征值,$\lambda_1\neq \lambda_2$

						如果$\alpha_1,\cdots,\alpha_m \in V_{\lambda_1}$线性无关,$\beta_1,\cdots,\beta_n \in V_{\lambda_2}$线性无关

						那么$\alpha_1,\cdots,\alpha_m,\beta_1,\cdots,\beta_n$也线性无关
					\end{proposition}
					\begin{proof}
						取线性组合并设其为零:

						$k_1\alpha_1+\cdots+k_m\alpha_m+l_1\beta_1+\cdots+l_n\beta_n=\mathbf{0}$

						将$A$在其上进行变换得:$k_1A(\alpha_1)+\cdots+k_mA(\alpha_m)+l_1A(\beta_1)+\cdots+l_nA(\beta_n)=\mathbf{0}$

						$\Rightarrow k_1\lambda_1\alpha_1 +\cdots+k_m\lambda_1\alpha_m+l_1\lambda_2\beta_1+\cdots+l_n\lambda_2\beta_n=\mathbf{0}$

						但是,如果我们把最初的线性组合乘以$\lambda_1$,得:

						$k_1\lambda_1\alpha_1 +\cdots+k_m\lambda_1\alpha_m+l_1\lambda_1\beta_1+\cdots+l_n\lambda_1\beta_n=\mathbf{0}$

						将两式相减,得:

						$l_1(\lambda_2-\lambda_1)\beta_1+\cdots+l_n(\lambda_2-\lambda_1)\beta_n=\mathbf{0}$

						$\Rightarrow \forall i,l_i(\lambda_2-\lambda_1)=0 \Rightarrow l_i=0$

						$\Rightarrow k_1\alpha_1+\cdots+k_m\alpha_m=\mathbf{0} \Rightarrow \forall i, k_i =0$

						于是命题得证
					\end{proof}
					一个显然的推论是此结论的$n$个子空间的版本:
					\begin{corollary}{}
						设$A \in \hom(V,V)$,$\lambda_1,\cdots,\lambda_n$是$A$的$n$个互不相同的特征值

						如果$\forall i,\alpha_{ir_1},\cdots,\alpha_{ir_m} \in V_{\lambda_i}$线性无关

						那么向量组$\{\alpha_{jr_k}\}$也线性无关
					\end{corollary}
					\begin{proof}
						由数学归纳法易证
					\end{proof}
				\point{}
					
					容易注意到,矩阵的特征值和线性映射的特征值其实是一样的,正如下面的命题:
					\begin{proposition}
						设$V$是一个$F$上的线性空间,$\dim V = n < \aleph_0$,$\{\alpha_1,\cdots,\alpha_n\}$是$V$的一个基

						$A \in \hom(V,V)$在$\{\alpha_1,\cdots,\alpha_n\}$下的矩阵是$\symbfit{A}$

						那么,$\lambda $是$A$的一个特征值,$\alpha $是$A$的一个隶属于$\lambda $的一个特征向量$\Leftrightarrow$

						$\lambda $是$\symbfit{A}$的一个特征值,并且$\alpha $在$\{\alpha_1,\cdots,\alpha_n\}$下的坐标$\mathbf{x}$是$\symbfit{A}$的一个隶属于$\lambda $的特征向量
					\end{proposition}
					\begin{proof}
						$A(\alpha )=\lambda \alpha $

						$\Leftrightarrow \symbfit{A}\mathbf{x}=\lambda \mathbf{x}$(参见命题2.3.23)

						于是命题得证。
					\end{proof}
			\end{para}
		\subsection{特征矩阵与特征多项式}
			前面我们研究了特征值的性质,接下来我们想知道:是否可以直接去寻找计算特征值的直接方法?事实上,是可以的,我们指出:特征值是特征多项式的一个根,而特征向量是对应映射的核的一个元素

			先给出定义:
			\begin{defn}{线性变换的特征多项式}{}
				设$A \in \hom(V,V)$,我们称

				$f(\lambda )=\det(\lambda I-A)$

				为$A$的特征多项式
			\end{defn}
			类似地,我们可以定义矩阵的特征矩阵和特征多项式:
			\begin{defn}{矩阵的特征矩阵和特征多项式}{}
				设$\symbfit{A}\in M_n(F)$,我们称$\lambda \symbfit{I}-\symbfit{A}$是$\symbfit{A}$的特征矩阵

				并称$f(\lambda )=\det(\lambda \symbfit{I}-\symbfit{A})$是$\symbfit{A}$的特征多项式
			\end{defn}
			下面的性质指出了我们想要的结果:
			\begin{para}{0}
				\point{}
					\begin{proposition}
						设$V$是一个$F$上的线性空间,$\dim V = n < \aleph_0$,$\{\alpha_1,\cdots,\alpha_n\}$

						如果$A \in \hom(V,V)$在$\{\alpha_1,\cdots,\alpha_n\}$下的矩阵是$\symbfit{A}$

						那么我们断言:$A$的特征多项式$f(\lambda )=\det(\lambda I-A)$和$\symbfit{A}$的特征多项式$g(\lambda )=\det(\lambda \symbfit{I}-\symbfit{A})$相等,即$f(\lambda )=g(\lambda )$
					\end{proposition}
					\begin{proof}
						这是显然的,因为矩阵的行列式的定义就是其矩阵的行列式
					\end{proof}
				\point{}
					\begin{them}{线性变换的特征值即是特征多项式在域内的根}{}
						设
					\end{them}
			\end{para}
	\section{线性变换的对角标准型}
	\section{线性变换的不变子空间}
	\section{线性变换的最小多项式}
	\section{幂零变换的Jordan标准型}
	\section{线性变换的Jordan标准型}
\ifx\allfiles\undefined
\end{document}
\fi