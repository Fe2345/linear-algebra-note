\ifx\allfiles\undefined
\documentclass[12pt, a4paper, oneside, UTF8]{ctexbook}
\def\path{../config}
\usepackage{amsmath}
\usepackage{amsthm}
\usepackage{amssymb}
\usepackage{graphicx}
\usepackage{mathrsfs}
\usepackage{enumitem}
\usepackage{geometry}
\usepackage[colorlinks, linkcolor=black]{hyperref}
\usepackage{stackengine}
\usepackage{yhmath}
\usepackage{extarrows}
\usepackage{unicode-math}
\usepackage{tikz}
\usepackage{tikz-cd}
\usepackage{pifont}
\usepackage{pgfplots}
\usepackage{tikz-3dplot}

\usepackage{fancyhdr}
\usepackage[dvipsnames, svgnames]{xcolor}
\usepackage{listings}

\definecolor{mygreen}{rgb}{0,0.6,0}
\definecolor{mygray}{rgb}{0.5,0.5,0.5}
\definecolor{mymauve}{rgb}{0.58,0,0.82}

\graphicspath{ {figure/},{../figure/}, {config/}, {../config/} }

\linespread{1.6}

\geometry{
    top=25.4mm, 
    bottom=25.4mm, 
    left=20mm, 
    right=20mm, 
    headheight=2.17cm, 
    headsep=4mm, 
    footskip=12mm
}

\setenumerate[1]{itemsep=5pt,partopsep=0pt,parsep=\parskip,topsep=5pt}
\setitemize[1]{itemsep=5pt,partopsep=0pt,parsep=\parskip,topsep=5pt}
\setdescription{itemsep=5pt,partopsep=0pt,parsep=\parskip,topsep=5pt}

\lstset{
    language=Mathematica,
    basicstyle=\tt,
    breaklines=true,
    keywordstyle=\bfseries\color{NavyBlue}, 
    emphstyle=\bfseries\color{Rhodamine},
    commentstyle=\itshape\color{black!50!white}, 
    stringstyle=\bfseries\color{PineGreen!90!black},
    columns=flexible,
    numbers=left,
    numberstyle=\footnotesize,
    frame=tb,
    breakatwhitespace=false,
} 
\usepackage[strict]{changepage} 
\usepackage{framed}
\usepackage{tcolorbox}
\tcbuselibrary{most}

\definecolor{greenshade}{rgb}{0.90,1,0.92}
\definecolor{redshade}{rgb}{1.00,0.88,0.88}
\definecolor{brownshade}{rgb}{0.99,0.95,0.9}
\definecolor{lilacshade}{rgb}{0.95,0.93,0.98}
\definecolor{orangeshade}{rgb}{1.00,0.88,0.82}
\definecolor{lightblueshade}{rgb}{0.8,0.92,1}
\definecolor{purple}{rgb}{0.81,0.85,1}

% #### 将 config.tex 中的定理环境的对应部分替换为如下内容
% 定义单独编号,其他四个共用一个编号计数 这里只列举了五种,其他可类似定义(未定义的使用原来的也可)
\newtcbtheorem[number within=section]{defn}%
{定义}{colback=OliveGreen!10,colframe=Green!70,fonttitle=\bfseries}{def}

\newtcbtheorem[number within=section]{lemma}%
{引理}{colback=Salmon!20,colframe=Salmon!90!Black,fonttitle=\bfseries}{lem}

% 使用另一个计数器 use counter from=lemma
\newtcbtheorem[use counter from=lemma, number within=section]{them}%
{定理}{colback=SeaGreen!10!CornflowerBlue!10,colframe=RoyalPurple!55!Aquamarine!100!,fonttitle=\bfseries}{them}

\newtcbtheorem[use counter from=lemma, number within=section]{criterion}%
{准则}{colback=green!5,colframe=green!35!black,fonttitle=\bfseries}{cri}

\newtcbtheorem[use counter from=lemma, number within=section]{corollary}%
{推论}{colback=Emerald!10,colframe=cyan!40!black,fonttitle=\bfseries}{cor}
% colback=red!5,colframe=red!75!black

% 这个颜色我不喜欢
%\newtcbtheorem[number within=section]{proposition}%
%{命题}{colback=red!5,colframe=red!75!black,fonttitle=\bfseries}{cor}

% .... 命题 例 注 证明 解 使用之前的就可以(全文都是这种框框就很丑了),也可以按照上述定义 ...
\renewenvironment{proof}{\par\textbf{证明:}\;}{\qed\par}
\newenvironment{solution}{\par{\textbf{解:}}\;}{\qed\par}
\newtheorem{proposition}{\indent 命题}[section]
\newtheorem{example}{\indent \color{SeaGreen}{例}}[section] % 绿色文字的 例 ,不需要就去除\color{SeaGreen}{}
\newtheorem*{rmk}{\indent 注}
\usepackage{amssymb}
\setmathfont{LatinModernMath-Regular}
\setmathfont[range=\mathbb]{TeXGyrePagellaMath-Regular}
\def\d{\mathrm{d}}
\def\R{\mathbb{R}}
\def\C{\mathbb{C}}
\def\Q{\mathbb{Q}}
\def\N{\mathbb{N}}
\def\Z{\mathbb{Z}}
\newcommand{\bs}[1]{\boldsymbol{#1}}
\newcommand{\ora}[1]{\overrightarrow{#1}}
\newcommand{\myspace}[1]{\par\vspace{#1\baselineskip}}
\newcommand{\xrowht}[2][0]{\addstackgap[.5\dimexpr#2\relax]{\vphantom{#1}}}
\newenvironment{ca}[1][1]{\linespread{#1} \selectfont \begin{cases}}{\end{cases}}
\newenvironment{vx}[1][1]{\linespread{#1} \selectfont \begin{vmatrix}}{\end{vmatrix}}
\newcommand{\tabincell}[2]{\begin{tabular}{@{}#1@{}}#2\end{tabular}}
\newcommand{\pll}{\kern 0.56em/\kern -0.8em /\kern 0.56em}
\newcommand{\dive}[1][F]{\mathrm{div}\;\bs{#1}}
\newcommand{\rotn}[1][A]{\mathrm{rot}\;\bs{#1}}
\usepackage{xeCJK}
\setCJKmainfont{SimSun}[BoldFont={SimHei}, ItalicFont={KaiTi}] % 设置中文支持

\newcommand{\point}[1]{\item {#1}}
\newenvironment{para}[1]{%
\ifcase#1\relax
\begin{enumerate}[label=\arabic*.] % 1.2.3.
\or
\begin{enumerate}[label=\textcircled{\arabic*}] % ①②③
\or
\begin{enumerate}[label=(\roman*)] % (i)(ii)(iii)
\else
\begin{enumerate}[label=\arabic*.] % 默认格式
\fi
}{
\end{enumerate}
}

\def\myIndex{0}
% \input{\path/cover_package_\myIndex.tex}

\def\myTitle{高等代数笔记}
\def\myAuthor{Zhang Liang}
\def\myDateCover{\today}
\def\myDateForeword{\today}
\def\myForeword{前言标题}
\def\myForewordText{
    前言内容
}
\def\mySubheading{副标题}


\begin{document}
% \input{\path/cover_text_\myIndex.tex}

\newpage
\thispagestyle{empty}
\begin{center}
    \Huge\textbf{\myForeword}
\end{center}
\myForewordText
\begin{flushright}
    \begin{tabular}{c}
        \myDateForeword
    \end{tabular}
\end{flushright}

\newpage
\pagestyle{plain}
\setcounter{page}{1}
\pagenumbering{Roman}
\tableofcontents

\newpage
\pagenumbering{arabic}
\setcounter{chapter}{0}
\setcounter{page}{0}

\pagestyle{fancy}
\fancyfoot[C]{\thepage}
\renewcommand{\headrulewidth}{0.4pt}
\renewcommand{\footrulewidth}{0pt}








\else
\fi
%标题
\chapter{线性变换的表示与分解}
	\section{线性变换的特征值和特征向量}
		\subsection{特征值和特征向量的定义}
			\begin{defn}{线性变换的特征值、特征向量、特征子空间}{}
				设$V$是一个$F$上的线性空间,$A \in \hom(V,V)$

				那么如果$\exists \lambda \in F,\alpha \in V,\alpha \neq \mathbf{0}$使得

				\begin{equation}
					A(\alpha )=\lambda \alpha 
				\end{equation}
				成立

				那么我们称$\lambda $是$A$的一个特征值,$\alpha $是$A$的隶属于$\lambda $的一个特征向量

				同时我们定义:
				\begin{equation}
					V_\lambda := \{\alpha | A(\alpha )=\lambda \alpha \}
				\end{equation}
				称为$A$的属于特征值$\lambda $的特征子空间
			\end{defn}
			特征值代表着一个线性映射对一个向量的伸缩程度,这里我们要求至少能找到一个向量不为零是因为:如果一个特征值只对于零向量成立,那么它过于平凡,而且会扰乱后续我们一些关于特征值数量的命题

			类似地,我们自然可以定义矩阵的特征值
			\begin{defn}{矩阵的特征值、特征向量}{}
				设$\symbfit{A} \in M_n(F)$

				那么如果$\exists \lambda \in F,\alpha \in F^n$,使得

				\begin{equation}
					\symbfit{A}\alpha =\lambda \alpha 
				\end{equation}
				成立。

				那么我们称$\lambda $是$\symbfit{A}$的一个特征值,$\alpha $是$\symbfit{A}$的隶属于$\lambda $的一个特征向量
			\end{defn}
		\subsection{特征值和特征向量的性质}
			我们探讨一些相关的基本性质
			\begin{para}{0}
				\point{}

					我们首先验证,特征子空间的确是子空间
					\begin{proposition}
						$\forall A \in \hom(V,V)$,$V_\lambda $是$V$的一个线性子空间
					\end{proposition}
					\begin{proof}
						取$\forall \alpha ,\beta \in V_\lambda ,k \in F$

						$A(\alpha +\beta )=\lambda \alpha +\lambda \beta =\lambda (\alpha +\beta ) \Rightarrow \alpha +\beta \in V_\lambda $

						$A(k\alpha )=k\lambda \alpha =\lambda (k\alpha )\Rightarrow k\alpha \in V_\lambda $

						于是命题得证
					\end{proof}
					请注意:特征子空间并非全体特征向量的集合,而是全体特征向量和零向量的集合
				\point{}
					
					从特征值的定义可以看出,特征值就像是把线性变换的一部分转换为纯量乘法。我们猜想:每一个特征值体现了映射的不同的“方向”,不同特征子空间的线性无关向量组的并也应该线性无关

					以下命题证实了这个猜想
					\begin{proposition}
						设$A \in \hom(V,V)$,$\lambda_1,\lambda_2$是$A$的两个特征值,$\lambda_1\neq \lambda_2$

						如果$\alpha_1,\cdots,\alpha_m \in V_{\lambda_1}$线性无关,$\beta_1,\cdots,\beta_n \in V_{\lambda_2}$线性无关

						那么$\alpha_1,\cdots,\alpha_m,\beta_1,\cdots,\beta_n$也线性无关
					\end{proposition}
					\begin{proof}
						取线性组合并设其为零:

						$k_1\alpha_1+\cdots+k_m\alpha_m+l_1\beta_1+\cdots+l_n\beta_n=\mathbf{0}$

						将$A$在其上进行变换得:$k_1A(\alpha_1)+\cdots+k_mA(\alpha_m)+l_1A(\beta_1)+\cdots+l_nA(\beta_n)=\mathbf{0}$

						$\Rightarrow k_1\lambda_1\alpha_1 +\cdots+k_m\lambda_1\alpha_m+l_1\lambda_2\beta_1+\cdots+l_n\lambda_2\beta_n=\mathbf{0}$

						但是,如果我们把最初的线性组合乘以$\lambda_1$,得:

						$k_1\lambda_1\alpha_1 +\cdots+k_m\lambda_1\alpha_m+l_1\lambda_1\beta_1+\cdots+l_n\lambda_1\beta_n=\mathbf{0}$

						将两式相减,得:

						$l_1(\lambda_2-\lambda_1)\beta_1+\cdots+l_n(\lambda_2-\lambda_1)\beta_n=\mathbf{0}$

						$\Rightarrow \forall i,l_i(\lambda_2-\lambda_1)=0 \Rightarrow l_i=0$

						$\Rightarrow k_1\alpha_1+\cdots+k_m\alpha_m=\mathbf{0} \Rightarrow \forall i, k_i =0$

						于是命题得证
					\end{proof}
					一个显然的推论是此结论的$n$个子空间的版本:
					\begin{corollary}{}
						设$A \in \hom(V,V)$,$\lambda_1,\cdots,\lambda_n$是$A$的$n$个互不相同的特征值

						如果$\forall i,\alpha_{ir_1},\cdots,\alpha_{ir_m} \in V_{\lambda_i}$线性无关

						那么向量组$\{\alpha_{jr_k}\}$也线性无关
					\end{corollary}
					\begin{proof}
						由数学归纳法易证
					\end{proof}
				\point{}
					
					容易注意到,矩阵的特征值和线性映射的特征值其实是一样的,正如下面的命题:
					\begin{proposition}
						设$V$是一个$F$上的线性空间,$\dim V = n < \aleph_0$,$\{\alpha_1,\cdots,\alpha_n\}$是$V$的一个基

						$A \in \hom(V,V)$在$\{\alpha_1,\cdots,\alpha_n\}$下的矩阵是$\symbfit{A}$

						那么,$\lambda $是$A$的一个特征值,$\alpha $是$A$的一个隶属于$\lambda $的一个特征向量$\Leftrightarrow$

						$\lambda $是$\symbfit{A}$的一个特征值,并且$\alpha $在$\{\alpha_1,\cdots,\alpha_n\}$下的坐标$\mathbf{x}$是$\symbfit{A}$的一个隶属于$\lambda $的特征向量
					\end{proposition}
					\begin{proof}
						$A(\alpha )=\lambda \alpha $

						$\Leftrightarrow \symbfit{A}\mathbf{x}=\lambda \mathbf{x}$(参见命题2.3.23)

						于是命题得证。
					\end{proof}
			\end{para}
		\subsection{特征矩阵与特征多项式}
			前面我们研究了特征值的性质,接下来我们想知道:是否可以直接去寻找计算特征值的直接方法?事实上,是可以的,我们指出:特征值是特征多项式的一个根,而特征向量是对应映射的核的一个元素

			先给出定义:
			\begin{defn}{线性变换的特征多项式}{}
				设$A \in \hom(V,V)$,我们称

				$\chi_A(\lambda )=\det(\lambda I-A)$

				为$A$的特征多项式
			\end{defn}
			类似地,我们可以定义矩阵的特征矩阵和特征多项式:
			\begin{defn}{矩阵的特征矩阵和特征多项式}{}
				设$\symbfit{A}\in M_n(F)$,我们称$\lambda \symbfit{I}-\symbfit{A}$是$\symbfit{A}$的特征矩阵

				并称$\chi_{\symbfit{A}}(\lambda )=\det(\lambda \symbfit{I}-\symbfit{A})$是$\symbfit{A}$的特征多项式
			\end{defn}
			下面的性质指出了我们想要的结果:
			\begin{para}{0}
				\point{}
					\begin{proposition}
						设$V$是一个$F$上的线性空间,$\dim V = n < \aleph_0$,$\{\alpha_1,\cdots,\alpha_n\}$

						如果$A \in \hom(V,V)$在$\{\alpha_1,\cdots,\alpha_n\}$下的矩阵是$\symbfit{A}$

						那么我们断言:$\chi_A(\lambda )=\chi_{\symbfit{A}}(\lambda )$
					\end{proposition}
					\begin{proof}
						这是显然的,因为矩阵的行列式的定义就是其矩阵的行列式
					\end{proof}
				\point{}
					\begin{proposition}
						$\forall \symbfit{A} \in M_n(F),\chi_{\symbfit{A}}\in F[\lambda ]$
					\end{proposition}
					\begin{proof}
						设$\symbfit{A}=(a_{ij})\in M_n(F)$

						那么,$\lambda\symbfit{I}-\symbfit{A}=\begin{pmatrix}
							\lambda-a_{11} & -a_{12} & \cdots & -a_{1n} \\
							-a_{21} & \lambda-a_{22} & \cdots & -a_{2n} \\
							\vdots & \vdots & \ddots & \vdots \\
							-a_{n1} & -a_{n2} & \cdots & \lambda-a_{nn}
						\end{pmatrix}$

						由行列式的置换展开可知,$\chi_{\symbfit{A}}(\lambda )\in F[\lambda ]$
					\end{proof}
					\begin{corollary}{}{}
						$\forall A \in \hom(V,V),\dim V = n < \aleph_0,\chi_A(\lambda )\in F[\lambda ]$
					\end{corollary}
					\begin{proof}
						这是显然的,因为我们只需要任取一个基$\{\alpha_1,\cdots,\alpha_n\}$,并利用$A$在此基下的矩阵$\symbfit{A}$

						利用前面线性映射与其矩阵的特征多项式相同的命题,即可得证。
					\end{proof}
				\point{}
					\begin{them}{线性变换的特征值即是特征多项式在域内的根}{}
						设$V$是一个$F$上的$n$维线性空间,$A \in \hom(V,V)$

						那么:$\lambda $是$A$的一个特征值,$\alpha \in V_\lambda \Leftrightarrow \chi_A(\lambda )=0,\alpha \in \ker(\lambda I-A)$
					\end{them}
					\begin{proof}
						$\lambda $是$A$的一个特征值,$\alpha \in V_\lambda$

						$\Leftrightarrow A(\alpha )=\lambda \alpha,V_\lambda \neq \{\mathbf{0}\}$

						$\Leftrightarrow (\lambda I-A)(\alpha )=\mathbf{0},\ker (\lambda I-A) \neq \{\mathbf{0}\}$

						$\Leftrightarrow \rank(\lambda I-A) < n,\alpha \in \ker(\lambda I-A)$

						$\Leftrightarrow \chi_A(\lambda )=0,\alpha \in \ker (\lambda I-A)$
					\end{proof}
					其自然推论是其矩阵版本:
					\begin{corollary}{}{}
						设$\symbfit{A} \in M_n(F)$

						那么:$\lambda $是$\symbfit{A}$的一个特征值,$\mathbf{x} \in F^n$是$\symbfit{A}$从属于$\lambda $的特征向量$\Leftrightarrow \chi_{\symbfit{A}}(\lambda )=0,(\lambda\symbfit{I}-\symbfit{A})\mathbf{x}=\mathbf{0},\mathbf{x}\neq \mathbf{0}$
					\end{corollary}
					这个推论也指出了一下结果:
					\begin{corollary}{}{}
						设$\symbfit{A},\symbfit{B} \in M_n(F),\symbfit{A}\sim \symbfit{B}$

						那么$\symbfit{A}$和$\symbfit{B}$具有相同的特征值,并且$\chi_{\symbfit{A}}(\lambda )=\chi_{\symbfit{B}}(\lambda )$
					\end{corollary}
					我们之后不再完全讨论矩阵版本的特征值相关定理,因为上面的结果已经说明了:相似矩阵,以及相互对应的矩阵和映射在特征值理论中并无区别
				\point{}
					\begin{proposition}
						设$V$是一个$F$上的$n$维线性空间,$A \in \hom(V,V)$

						那么:

						\begin{equation}
							\chi_A(\lambda )=\lambda^n - \tr(A) \lambda^{n-1} + \cdots + (-1)^{n-k} \left(\sum\limits_{j_1 < \cdots < j_{n-k}} A_{j_1,\cdots,j_{n-k}}^{j_1,\cdots,j_{n-k}}\right) \lambda^k + \cdots + (-1)^n \det(A)
						\end{equation}
					\end{proposition}
					\begin{proof}
						事实上,$\lambda^k$项即是下面的行列式求和:
						$\sum\limits_{j_1' < \cdots < j_{k}'}\begin{vmatrix}
						-a_{11} & \cdots & 0 & \cdots & 0 & \cdots & -a_{1n} \\
						\vdots & \ddots & \vdots & & \vdots & & \vdots \\
						-1 & \cdots & \lambda & \cdots & 0 & \cdots & 1 \\
						\vdots & & \vdots & \ddots & \vdots &  & \vdots \\
						-1 & \cdots & 0 & \cdots & \lambda & \cdots & 1 \\
						\vdots & & \vdots &  & \vdots & \ddots & \vdots \\
						-a_{n1} & \cdots & 0 & \cdots & 0 & \cdots & -a_{nn}
						\end{vmatrix}$
						其中行列式的$j_1',\cdots,j_{k}'$列为仅在第$j_i$个元素为$\lambda $,其他位置都是$0$的列。

						这是因为,如果想要产生$\lambda^k$,那么求和中的每一项必须选取对角线上的$k$个元素;同时,如果在进一步展开中选取了$-a_{ii}$而不是$\lambda $,那么会导致次数降低

						注意到:$\begin{vmatrix}
						-a_{11} & \cdots & 0 & \cdots & 0 & \cdots & -a_{1n} \\
						\vdots & \ddots & \vdots & & \vdots & & \vdots \\
						-1 & \cdots & \lambda & \cdots & 0 & \cdots & 1 \\
						\vdots & & \vdots & \ddots & \vdots &  & \vdots \\
						-1 & \cdots & 0 & \cdots & \lambda & \cdots & 1 \\
						\vdots & & \vdots &  & \vdots & \ddots & \vdots \\
						-a_{n1} & \cdots & 0 & \cdots & 0 & \cdots & -a_{nn}
						\end{vmatrix}$

						$=(-1)^{(j_1'+\cdots+j_{k}')+(j_1'+\cdots+j_{k}')}\det(diag\{\lambda,\cdots,\lambda \})(-1)^{n-k}A_{j_1,\cdots,j_{n-k}}^{j_1,\cdots,j_{n-k}}$

						这个结果只需要对$j_1',\cdots,j_k'$列展开即可得出,其中$(-1)^{n-k}$是将$-a_{ij}$前面的负号提出去得到的,而$j_1,\cdots,j_{n-k}$是与$j_1',\cdots,j_k'$互补的有限序列

						于是命题得证。
					\end{proof}
					其推论是以下结论:
					\begin{corollary}{}{}
						设$F$是一个代数闭域,$V$是一个$F$上的$n$维线性空间,$A \in \hom(V,V)$

						设$\lambda_1,\cdots,\lambda_n$是$A$的特征值($\chi_A(\lambda )$中的重根按重数计算)

						那么有:
						\begin{equation}
							\prod\limits_{i=1}^{n} \lambda_i = \det(A)
						\end{equation}
						\begin{equation}
							\sum\limits_{i=1}^{n} \lambda_i = \tr(A)
						\end{equation}
					\end{corollary}
					\begin{proof}
						对$\chi_A(\lambda )$作唯一分解:

						$\chi_A(\lambda )=\prod\limits_{i=1}^{n}(\lambda -\lambda_i)$

						$=\lambda^n - (\lambda_1+\cdots+\lambda_n)\lambda^{n-1}+\cdots+(-1)^n \prod_{i=1}^{n}\lambda_i$

						结合前面的命题既得。
					\end{proof}
					请注意:这个命题必须要求$F$代数闭,否则可能$\chi_A(\lambda )$的根并非全体特征值,而导致论证失效。
			\end{para}
	\section{线性变换的对角标准型}
		\subsection{对角标准型的定义}
			\begin{defn}{线性映射的对角标准型}{}
				设$V$是一个$F$上的一个线性空间,$\dim V = n < \aleph_0$

				如果存在一个基$\{\alpha_1,\cdots,\alpha_n\}$,使$A$在其上的矩阵$\symbfit{A}$是一个对角矩阵

				那么我们称$\symbfit{A}$为$A$的对角标准型,此时称$A$可对角化。
			\end{defn}
			类似地,我们可以定义矩阵的对角标准型
			\begin{defn}{矩阵的对角标准型}{}
				设$\symbfit{A} \in M_n(F)$,如果存在$\symbfit{P}\in GL_n(F)$和对角矩阵$\symbfit{D}\in M_n(F)$,使得

				$\symbfit{A}=\symbfit{P}^{-1}\symbfit{D}\symbfit{P}$

				那么我们称$\symbfit{D}$为$\symbfit{A}$的对角标准型,此时称$\symbfit{A}$可对角化。
			\end{defn}
			容易看出,线性映射的对角标准型与矩阵的对角标准型是等价的。特别地,如果认为$\symbfit{A}$是$A \in \hom(F^n,F^n)$在$\{\mathbf{e}_i\}$下的矩阵,那么容易发现:其实$\symbfit{P}^{-1}$的列向量就是$A$的特征向量。
		\subsection{线性变换可对角化的条件}
			观察对角矩阵的结构,既得以下显然的条件:
			\begin{them}{线性映射可对角化的条件(1)}{}
				设$V$是一个$F$上的一个线性空间,$\dim V = n < \aleph_0$

				那么:$A \in \hom(V,V)$可对角化$\Leftrightarrow$存在一个由$A$的特征向量组成的$V$的基
			\end{them}
			\begin{proof}
				$A \in \hom(V,V)$可对角化

				$\Leftrightarrow$存在一个基$\{\alpha_1,\cdots,\alpha_n\}$,使得$A$在其上的矩阵为$diag\{\lambda_1,\cdots,\lambda_n\}$

				$\Leftrightarrow A(\alpha_1,\cdots,\alpha_n)=diag\{\lambda_1,\cdots,\lambda_n\}(\alpha_1,\cdots,\alpha_n)$

				$\Leftrightarrow A(\alpha_1,\cdots,\alpha_n)=(\lambda_1\alpha_1,\cdots,\lambda_n \alpha_n)$

				$\Leftrightarrow \lambda_i$是$A$的特征值,$\alpha_i \in V_{\lambda_i}$

				$\Leftrightarrow $存在一个由$A$的特征向量组成的$V$的基
			\end{proof}
			\begin{corollary}{线性映射可对角化的条件(2)}{}
				
			\end{corollary}
	\section{线性变换的不变子空间}
	\section{线性变换的最小多项式}
	\section{幂零变换的Jordan标准型}
	\section{线性变换的Jordan标准型}
\ifx\allfiles\undefined
\end{document}
\fi