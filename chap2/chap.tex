\ifx\allfiles\undefined
\documentclass[12pt, a4paper, oneside, UTF8]{ctexbook}
\def\path{../config}
\usepackage{amsmath}
\usepackage{amsthm}
\usepackage{amssymb}
\usepackage{graphicx}
\usepackage{mathrsfs}
\usepackage{enumitem}
\usepackage{geometry}
\usepackage[colorlinks, linkcolor=black]{hyperref}
\usepackage{stackengine}
\usepackage{yhmath}
\usepackage{extarrows}
\usepackage{unicode-math}
\usepackage{tikz}
\usepackage{tikz-cd}
\usepackage{pifont}
\usepackage{pgfplots}
\usepackage{tikz-3dplot}

\usepackage{fancyhdr}
\usepackage[dvipsnames, svgnames]{xcolor}
\usepackage{listings}

\definecolor{mygreen}{rgb}{0,0.6,0}
\definecolor{mygray}{rgb}{0.5,0.5,0.5}
\definecolor{mymauve}{rgb}{0.58,0,0.82}

\graphicspath{ {figure/},{../figure/}, {config/}, {../config/} }

\linespread{1.6}

\geometry{
    top=25.4mm, 
    bottom=25.4mm, 
    left=20mm, 
    right=20mm, 
    headheight=2.17cm, 
    headsep=4mm, 
    footskip=12mm
}

\setenumerate[1]{itemsep=5pt,partopsep=0pt,parsep=\parskip,topsep=5pt}
\setitemize[1]{itemsep=5pt,partopsep=0pt,parsep=\parskip,topsep=5pt}
\setdescription{itemsep=5pt,partopsep=0pt,parsep=\parskip,topsep=5pt}

\lstset{
    language=Mathematica,
    basicstyle=\tt,
    breaklines=true,
    keywordstyle=\bfseries\color{NavyBlue}, 
    emphstyle=\bfseries\color{Rhodamine},
    commentstyle=\itshape\color{black!50!white}, 
    stringstyle=\bfseries\color{PineGreen!90!black},
    columns=flexible,
    numbers=left,
    numberstyle=\footnotesize,
    frame=tb,
    breakatwhitespace=false,
} 
\usepackage[strict]{changepage} 
\usepackage{framed}
\usepackage{tcolorbox}
\tcbuselibrary{most}

\definecolor{greenshade}{rgb}{0.90,1,0.92}
\definecolor{redshade}{rgb}{1.00,0.88,0.88}
\definecolor{brownshade}{rgb}{0.99,0.95,0.9}
\definecolor{lilacshade}{rgb}{0.95,0.93,0.98}
\definecolor{orangeshade}{rgb}{1.00,0.88,0.82}
\definecolor{lightblueshade}{rgb}{0.8,0.92,1}
\definecolor{purple}{rgb}{0.81,0.85,1}

% #### 将 config.tex 中的定理环境的对应部分替换为如下内容
% 定义单独编号,其他四个共用一个编号计数 这里只列举了五种,其他可类似定义(未定义的使用原来的也可)
\newtcbtheorem[number within=section]{defn}%
{定义}{colback=OliveGreen!10,colframe=Green!70,fonttitle=\bfseries}{def}

\newtcbtheorem[number within=section]{lemma}%
{引理}{colback=Salmon!20,colframe=Salmon!90!Black,fonttitle=\bfseries}{lem}

% 使用另一个计数器 use counter from=lemma
\newtcbtheorem[use counter from=lemma, number within=section]{them}%
{定理}{colback=SeaGreen!10!CornflowerBlue!10,colframe=RoyalPurple!55!Aquamarine!100!,fonttitle=\bfseries}{them}

\newtcbtheorem[use counter from=lemma, number within=section]{criterion}%
{准则}{colback=green!5,colframe=green!35!black,fonttitle=\bfseries}{cri}

\newtcbtheorem[use counter from=lemma, number within=section]{corollary}%
{推论}{colback=Emerald!10,colframe=cyan!40!black,fonttitle=\bfseries}{cor}
% colback=red!5,colframe=red!75!black

% 这个颜色我不喜欢
%\newtcbtheorem[number within=section]{proposition}%
%{命题}{colback=red!5,colframe=red!75!black,fonttitle=\bfseries}{cor}

% .... 命题 例 注 证明 解 使用之前的就可以(全文都是这种框框就很丑了),也可以按照上述定义 ...
\renewenvironment{proof}{\par\textbf{证明:}\;}{\qed\par}
\newenvironment{solution}{\par{\textbf{解:}}\;}{\qed\par}
\newtheorem{proposition}{\indent 命题}[section]
\newtheorem{example}{\indent \color{SeaGreen}{例}}[section] % 绿色文字的 例 ,不需要就去除\color{SeaGreen}{}
\newtheorem*{rmk}{\indent 注}
\usepackage{amssymb}
\setmathfont{LatinModernMath-Regular}
\setmathfont[range=\mathbb]{TeXGyrePagellaMath-Regular}
\def\d{\mathrm{d}}
\def\R{\mathbb{R}}
\def\C{\mathbb{C}}
\def\Q{\mathbb{Q}}
\def\N{\mathbb{N}}
\def\Z{\mathbb{Z}}
\newcommand{\bs}[1]{\boldsymbol{#1}}
\newcommand{\ora}[1]{\overrightarrow{#1}}
\newcommand{\myspace}[1]{\par\vspace{#1\baselineskip}}
\newcommand{\xrowht}[2][0]{\addstackgap[.5\dimexpr#2\relax]{\vphantom{#1}}}
\newenvironment{ca}[1][1]{\linespread{#1} \selectfont \begin{cases}}{\end{cases}}
\newenvironment{vx}[1][1]{\linespread{#1} \selectfont \begin{vmatrix}}{\end{vmatrix}}
\newcommand{\tabincell}[2]{\begin{tabular}{@{}#1@{}}#2\end{tabular}}
\newcommand{\pll}{\kern 0.56em/\kern -0.8em /\kern 0.56em}
\newcommand{\dive}[1][F]{\mathrm{div}\;\bs{#1}}
\newcommand{\rotn}[1][A]{\mathrm{rot}\;\bs{#1}}
\usepackage{xeCJK}
\setCJKmainfont{SimSun}[BoldFont={SimHei}, ItalicFont={KaiTi}] % 设置中文支持

\newcommand{\point}[1]{\item {#1}}
\newenvironment{para}[1]{%
\ifcase#1\relax
\begin{enumerate}[label=\arabic*.] % 1.2.3.
\or
\begin{enumerate}[label=\textcircled{\arabic*}] % ①②③
\or
\begin{enumerate}[label=(\roman*)] % (i)(ii)(iii)
\else
\begin{enumerate}[label=\arabic*.] % 默认格式
\fi
}{
\end{enumerate}
}

\def\myIndex{0}
% \input{\path/cover_package_\myIndex.tex}

\def\myTitle{高等代数笔记}
\def\myAuthor{Zhang Liang}
\def\myDateCover{\today}
\def\myDateForeword{\today}
\def\myForeword{前言标题}
\def\myForewordText{
    前言内容
}
\def\mySubheading{副标题}


\begin{document}
% \input{\path/cover_text_\myIndex.tex}

\newpage
\thispagestyle{empty}
\begin{center}
    \Huge\textbf{\myForeword}
\end{center}
\myForewordText
\begin{flushright}
    \begin{tabular}{c}
        \myDateForeword
    \end{tabular}
\end{flushright}

\newpage
\pagestyle{plain}
\setcounter{page}{1}
\pagenumbering{Roman}
\tableofcontents

\newpage
\pagenumbering{arabic}
\setcounter{chapter}{0}
\setcounter{page}{0}

\pagestyle{fancy}
\fancyfoot[C]{\thepage}
\renewcommand{\headrulewidth}{0.4pt}
\renewcommand{\footrulewidth}{0pt}








\else
\fi
%标题
\chapter{线性映射}
	本章开始,我们转向线性映射的研究。

	我们将用三章完成线性映射的研究。本章我们将从映射最基本的研究方式:向量的作用开始

	线性映射的独特之处在于:一方面它能在常规的映射加法和纯量乘法下构成一个线性空间;另一方面,如果我们将复合视为乘法,它可以构成一个幺环。

	第一节中,我们将给出线性映射的定义及运算,并研究基本性质;
	
	第二节中,我们将研究两种由线性映射导出的子空间,核和像,并借此提出一个概念:秩。它和我们之前的秩也有很强的联系;

	第三节到第五节中,我们将研究矩阵,它是将线性映射在基下的作用写成的一张数表,非常便于在数值上研究矩阵;

	第六节中,我们将研究行列式,它是一个反对称多线性函数,我们以此为工具,为后续我们对线性映射分解的研究铺垫。
	\section{线性映射的定义和运算}
		\subsection{线性映射的定义}
			我们首先给出线性映射的定义
			\begin{defn}{线性映射}{}
				设$V_1,V_2$是一个$F$上的两个线性空间,,映射$A:V_1 \to V_2$如果满足:

				$\forall \alpha ,\beta \in V_1,k \in F$

				$A(\alpha+\beta )=A(\alpha )+A(\beta )$

				$A(k\alpha )=kA(\alpha )$

				那么我们称$A$是一个从$V_1$到$V_2$的线性映射

				全体$V_1$到$V_2$的线性映射的集合记作$\hom_F (V_1,V_2)$,或简记作$\hom (V_1,V_2)$

				特别地,如果$V_1=V_2$,我们称$A$是一个$V_1$上的线性变换
			\end{defn}
			有一些常用的线性映射,我们在这里列出来:
			\begin{defn}{一些常用的线性映射}{}
				\begin{enumerate}
					\item 恒等变换:$I:V \ni \alpha \mapsto \alpha \in V$
					\item 数乘变换:$k:V \ni \alpha \mapsto k\alpha \in V$
					\item 零变换:$0:V_1 \ni \alpha \mapsto \mathbf{0}_{V_2} \in V_2$
				\end{enumerate}
			\end{defn}
		\subsection{线性映射的运算}
			前面我们定义了线性映射,现在我们开始赋予$\hom (V_1,V_2)$线性空间和环的性质。

			我们会定义三种运算:加法、纯量乘法、乘法
			\begin{defn}{线性映射的运算}{}
				我们定义:
				
				映射$+:\hom(V_1,V_2) \times \hom(V_1,V_2) \rightarrow \hom(V_1,V_2)$,称为加法,如果满足:
				\begin{equation}
					\forall A,B \in \hom(V_1,V_2),\alpha \in V_1,(A+B)(\alpha )=A(\alpha )+B(\alpha )
				\end{equation}
				映射$\cdot:F \times \hom(V_1,V_2) \rightarrow \hom(V_1,V_2)$,称为纯量乘法,如果满足:
				\begin{equation}
					\forall k \in F,A \in \hom(V_1,V_2),\alpha \in F,(k\cdot A)(\alpha )=kA(\alpha )
				\end{equation}
				映射$\circ:\hom(V_1,V_2) \times \hom(V_1,V_2) \rightarrow \hom(V_1,V_2)$,称为乘法,如果满足:
				\begin{equation}
					\forall A,B \in \hom(V_1,V_2),\alpha \in V_1,(A\circ B)(\alpha )=A\left(B(\alpha )\right)
				\end{equation}
			\end{defn}
			我们也常常把$k\cdot A$简记为$kA$,将$A \circ B$简记为$AB$

			显然,$\left(\hom(V_1,V_2),F,+,\cdot\right)$是一个线性空间,$0$是它的零向量;

			$\left(\hom(V_1,V_2),+,\circ\right)$是一个幺环,$0$是它的加法单位元,$I$是它的乘法单位元

			除此之外,还有一些运算,但是它们是针对特殊的线性映射的,比如说:
			\begin{defn}{线性变换的幂}{}
				$\forall A \in \hom(V,V)$

				我们定义:$A^m:= \begin{cases}
					A\circ A^{m-1},m \geqslant 1 \\
					I,m = 0
				\end{cases},m \geqslant 0$
			\end{defn}
			如果一个映射的幂不会使其本身变化,我们称它是一个幂等变换
			\begin{defn}{幂等映射}{}
				$A \in \hom(V,V)$如果有:

				$A = A^2$

				我们称它是一个幂等变换
			\end{defn}
			我们不再讨论其他的运算,我们接下来转入线性映射一般性质的研究
		\subsection{线性映射的性质}
			\begin{para}{0}
				\point{}
					\begin{proposition}
						$\forall A \in \hom(V_1,V_2),A(\mathbf{0}_{V_1}) = \mathbf{0}_{V_2}$
					\end{proposition}
					\begin{proof}
						$A(\mathbf{0}_{V_1})=A(0\cdot \mathbf{0}_{V_1})=0\cdot A(\mathbf{0}_{V_1})=\mathbf{0}_{V_2}$
					\end{proof}
				\point{}
					\begin{proposition}
						$\forall A \in \hom(V_1,V_2),A(-\alpha )=-A(\alpha )$
					\end{proposition}
					\begin{proof}
						$A(-\alpha )=A((-1)\cdot \alpha )=(-1)\cdot A(\alpha )=-A(\alpha )$
					\end{proof}
				\point{}
					\begin{proposition}
						$\forall A \in \hom(V_1,V_2),A(\sum\limits_{i=1}^{n} k_i \alpha_i)=\sum\limits_{i=1}^{n} k_i A(\alpha_i)$
					\end{proposition}
					\begin{proof}
						对$n$使用数学归纳法易证。
					\end{proof}
					值得注意,这个定理并不能随意地推广到$\aleph_0$,因为此时依赖于度量线性空间或线性映射的进一步性质。
				\point{}
					\begin{proposition}
						$\forall A \in \hom(V,V),m \geqslant 1, A^m = A^{m-1} \circ A$
					\end{proposition}
					\begin{proof}
						对$m$作数学归纳法。

						首先,当$m=2$时,$A^2 = A \circ A$,命题成立

						现在假设$m$时成立,我们来证明$m+1$时命题也成立:

						$A^m = A \circ A^{m-1} = A \circ A^{m-2} \circ A = A^{m-1} \circ A$,于是命题得证。
					\end{proof}
					这个命题看似显然,但是是必要的,因为线性映射环不交换。这个命题指出:递归式地推导幂时,从左右方向都是等价的。进一步,在递归中不断变换方向也不会影响结果。
				\point{}
					\begin{proposition}
						$\forall A \in \hom(V_1,V_2)$

						如果$\alpha_1,\cdots,\alpha_s$线性相关,那么$A(\alpha_1),\cdots,A(\alpha_s)$也线性相关
					\end{proposition}
					\begin{proof}
						$\alpha_1,\cdots,\alpha_s$线性相关

						$\Rightarrow \exists k_1,\cdots,k_s,k_1\alpha_1+\cdots+k_n\alpha_s=\mathbf{0}$,其中$k_1,\cdots,k_s$不全为零

						$\Rightarrow A(k_1\alpha_1+\cdots+k_n\alpha_s)=k_1A(\alpha_1)+\cdots+k_nA(\alpha_s)=\mathbf{0}$

						$\Rightarrow A(\alpha_1),\cdots,A(\alpha_n)$线性无关
					\end{proof}
					值得注意的是,不同于同构映射,在这个命题中把线性相关改为线性无关会使命题变得不成立。比如说,零映射会把任何线性无关的向量组变得线性相关
				\point{}
					\begin{proposition}
						$\forall T,Y \in \hom(V_1,V_2)$,$B$是$V_1$的一个基

						如果$\forall \alpha \in B,T(\alpha )=Y(\beta )$,那么$T=Y$
					\end{proposition}
					\begin{proof}
						只需证明:$\forall \gamma \in V_1,T(\gamma )=Y(\gamma )$

						因为$B$是$V_1$的基,所以一定有$\gamma =k_1 \alpha_1+\cdots+k_s \alpha_s,\alpha_1,\cdots,\alpha_s \in B$

						此时,$T(\gamma )=k_1 T(\alpha_1)+\cdots+ k_s T(\alpha_s)=k_1 Y(\alpha_1)+\cdots+ k_s Y(\alpha_s)=Y(\gamma )$,于是命题得证。
					\end{proof}
					这个命题指出,线性映射完全由其在基上的作用决定,因为我们其实只需要指定基上的像就指定了线性映射本身。
			\end{para}
			最后,还有一个问题未被解决:$\hom(V,W)$的维数和$\dim V,\dim W$有什么关系。这个问题比较复杂,因为涉及无限维时,维数的公式会变得和有限情形完全不同。在讲解行列式后,我们将在附录一中解决这个问题。
	\section{线性映射的核和像}
		本节中,我们将借助核与像继续研究线性映射。所谓核,即是线性映射映到零的那一部分;像则是值域。

		同时,我们会引入对偶映射,借助本节中核与像的工具,我们将看到对偶和本身之间的联系。
			\subsection{核与像的定义}
				\begin{defn}{线性映射的核}{}
					设$A \in \hom(V_1,V_2)$,我们定义:
					\begin{equation}
						\ker A := \{\alpha \in V_1 | A(\alpha )=\mathbf{0}_{V_2}\}
					\end{equation}

					称为线性映射$A$的核
				\end{defn}
				\begin{defn}{线性映射的像}{}
					设$A \in \hom(V_1,V_2)$,我们定义:
					\begin{equation}
						\im A := A(V_1) := \{A(\alpha)| \alpha \in V_1\}
					\end{equation}·

					称为线性映射$A$的像
				\end{defn}
				特别地,核与像的维数我们分别称为零化度和秩:
				\begin{defn}{线性映射的零化度}{}
					设$A \in \hom(V_1,V_2)$,我们定义:
					\begin{equation}
						\operatorname{nullity} (A) := \dim(\ker A)
					\end{equation}
					称为线性映射$A$的零化度
				\end{defn}
				\begin{defn}{线性映射的秩}{}
					设$A \in \hom(V_1,V_2)$,我们定义:
					\begin{equation}
						\rank (A) := \dim(\im A)
					\end{equation}
					称为线性映射$A$的秩
				\end{defn}
				事实上,线性映射的秩和之前我们曾提及的向量组的秩有着很大的联系,我们将在后续看到这一点。

				为了方便后续性质的研究,接下来我们给出对偶映射的概念
				\begin{defn}{对偶映射}{}
					设$A \in \hom(V_1,V_2)$,我们定义对偶映射$T^* \in \hom (V_2^*,V_1^*)$

					如果满足:$\forall f \in V_2^*,\alpha \in V_1$
					\begin{equation}
							\left(T^*(f)\right)(\alpha )=(f\circ T)(\alpha )
					\end{equation}
				\end{defn}
			\subsection{核与像的性质}
				接下来研究核与像的性质
				\begin{para}{0}
					\point{}
						\begin{proposition}
							$\forall A \in \hom(V,W),\ker A,\im A$都是线性空间
						\end{proposition}
						\begin{proof}
							先证明$\ker A$是一个线性空间

							注意到,$\mathbf{0}_V \in \ker A$,因为$A(\mathbf{0}_V)=\mathbf{0}_W$,因此$\ker A$非空

							那么,只需注意到$\forall \alpha ,\beta \in \ker A,A(\alpha +\beta )=A(\alpha )+A(\beta )=\mathbf{0}_{W} \Rightarrow \alpha + \beta \in \ker A$

							$\forall \alpha \in \ker A,k \in F,A(k\alpha )=kA(\alpha )=\mathbf{0}_W \Rightarrow k\alpha \in \ker A$

							再证明$\im A$是一个线性空间

							注意到,$\mathbf{0}_W = A(\mathbf{0}_V) \in \im A \Rightarrow \im A \neq \emptyset$

							那么,只需注意到$\forall A(\alpha) ,A(\beta) \in \im A,A(\alpha )+A(\beta )=A(\alpha +\beta )\in \im A$

							$\forall A(\alpha )\in \im A,k \in F,kA(\alpha )=A(k\alpha )\in \im A$

							于是命题得证
						\end{proof}
					\point{}
						\begin{proposition}
							$A \in \hom(V,W)$,那么:

							$A$是单射$\Leftrightarrow \ker A = \{\mathbf{0}_V\}$
						\end{proposition}
						\begin{proof}
							先证明充分性。注意到,$A(\mathbf{0}_V)=\mathbf{0}_W$,而$A$为单射,于是一定有$\ker A = \{\mathbf{0}_V\}$

							再证明必要性。设$A(\alpha )=A(\beta )$,那么$A(\alpha -\beta )=\mathbf{0}_W \Rightarrow \alpha -\beta =\mathbf{0}_V \Rightarrow \alpha =\beta \Rightarrow A$是单射
						\end{proof}
						\point{}
						\begin{proposition}
							$A \in \hom(V,W)$,那么:

							$A$是单射$\Leftrightarrow \im A = W$
						\end{proposition}
						\begin{proof}
							这是显然的。
						\end{proof}
					\point{秩-零化度定理}
						在给出定理前,我们先给出一个引理。它是一个线性空间的“同态基本定理”,证明方法也很相似
						\begin{lemma}{}{}
							$\forall A \in \hom(V,W)$
							\begin{equation}
								V/\ker A \cong \im A
							\end{equation}
						\end{lemma}
						\begin{proof}
							设$\phi : V/\ker A \ni \alpha + \ker A \mapsto A(\alpha ) \in \im A$

							首先验证它的确是一个映射

							假设$\alpha +\ker A=\beta +\ker A$

							那么$\alpha -\beta \in \ker A \Rightarrow A(\alpha -\beta )=\mathbf{0}_W \Rightarrow A(\alpha )=A(\beta )$

							所以$\phi $的确是一个映射

							接下来验证它是一个单射

							设$\phi (\alpha + \ker A)=\phi (\beta +\ker A)$
							
							$\Rightarrow A(\alpha )=A(\beta ) \Rightarrow A(\alpha -\beta )=\mathbf{0}_W$

							$\Rightarrow \alpha - \beta  \in \ker A \Rightarrow \alpha +\ker A=\beta +\ker A$

							于是$\phi $是单射。$\phi $显然是满射

							接下来验证线性性:

							$\forall \alpha +\ker A,\beta +\ker A \in V/\ker A,k \in F$
							
							$\phi \left((\alpha +\ker A)+(\beta +\ker A)\right)=\phi \left((\alpha +\beta )+\ker A\right)=A(\alpha +\beta )=A(\alpha )+A(\beta )=\phi (\alpha +\ker A)+\phi (\beta +\ker A)$
						
							$\phi \left(k\cdot (\alpha +\ker A)\right)=\phi (k\alpha +\ker A)=A(k\alpha )=kA(\alpha )=k\phi (\alpha +\ker A)$
						
							于是命题得证。
						\end{proof}

						它的直接推论是被称为秩-零化度定理的结论,它揭示了秩和零化度的联系
						\begin{them}{秩-零化度定理}{}
							$\forall A \in \hom(V,W)$
							\begin{equation}
								\rank(A)+\operatorname{nullity}(A) = \dim V
							\end{equation}
						\end{them}
						\begin{proof}
							$V/\ker A \cong \im A$

							$\Rightarrow \dim(V/\ker A)=\dim \im A$

							$\Rightarrow \dim V - \dim \ker A = \dim \im A$

							$\Rightarrow \rank(A)+\operatorname{nullity}(A)= \dim V$
						\end{proof}
					\point{有限维映射和其对偶的秩相同}
						\begin{proposition}
							设$A \in \hom(V,W),\dim V,\dim W = n < \aleph_0$

							那么$\rank(A)=\rank(A^*)$
						\end{proposition}
						\begin{proof}
							
						\end{proof}
					\point{}
						\begin{proposition}
							设$A \in \hom(V,W)$,如果$\dim V=\dim W$,那么:

							$A$是满射$\Leftrightarrow A$是单射
						\end{proposition}
				\end{para}
	\section{矩阵}
		\subsection{矩阵的定义}
			\begin{defn}{矩阵}{}
				形如以下的矩形阵列称为一个域$F$上的矩阵
				$\begin{pmatrix}
					a_{11} & \cdots & a_{1n} \\
					\vdots & \ddots & \vdots \\
					a_{m1} & \cdots & a_{mn}
				\end{pmatrix},a_{ij} \in F$

				简记为$(a_{ij})_{m \times n}$或$(a_{ij})$。$m$称为矩阵的行数,$n$称为矩阵的列数。

				特别地,如果$m=n$,我们称它是一个$m$阶方阵。

				$F$上的全体$m \times n$矩阵的集合记作$M_{m \times n} (F)$,特别地如果$m=n$,记作$M_n (F)$。

				我们也将矩阵$\symbfit{A}$在$m$行$n$列处的元素记作$\symbfit{A}_{ij}$
			\end{defn}
		\subsection{矩阵的运算}
		\begin{para}{0}
			\point{相等}
			    \begin{defn}{矩阵的相等}{}
				    设$A\in M_{m\times n}(F),B\in M_{m \times n}(F)$,如果$\forall i,j,A(i;j)=B(i;j)$,则称$A=B$。
				\end{defn}
			\point{转置}
				\begin{defn}{矩阵的转置}{}
				    设$A\in M_{m \times n}(F)$,
					
					我们定义矩阵$A^T \in M_{n \times m}(F)$为满足$A^T(i;j)=A(i;j)$的矩阵,称为$A$的转置。
				\end{defn}
			\point{加法}
				\begin{defn}{矩阵的加法}{}
				    设$A\in M_{m \times n}(F),B\in M_{m \times n}(F)$
					
					我们定义:$(A+B)(i;j)=A(i;j)+B(i;j)$。
				\end{defn}
			\point{纯量乘法}
				\begin{defn}{矩阵的纯量乘法}{}
				    设$A\in M_{m \times n}(F),k \in F$,
					
					我们定义矩阵$k\cdot A \in M_{m\times n}(F)$为满足$(k\cdot  A)(i;j)=k\cdot A(i;j)$的矩阵。
				\end{defn}
			\point{乘法}
				\begin{defn}{矩阵的乘法}{}
				    设$A\in M_{m\times n}(F),B\in M_{n \times p}(F)$,
					
					我们定义矩阵$A\cdot B \in M_{m\times p}(F)$为满足$(A\cdot B)(i;j)=\sum_{k=1}^n A(i;k)B(k;j)$的矩阵
				\end{defn}
			\point{幂}
				\begin{defn}{方阵的幂}{}
					设$A \in M_n (F)$是一个方阵,
					我们定义:$A^k=A\cdot A^{k-1}$
				\end{defn}
		\end{para}
		\subsection{矩阵的性质}
	\section{特殊矩阵}
	\section{可逆矩阵}
	\section{行列式}
		\subsection{行列式的定义和性质}
			\begin{defn}{行列式}{}
				设$F$是一个域,$V$是$F$上的一个线性空间,并且$dim_F V = n$

				映射$\det: V^n \rightarrow F$如果满足:

				\ding{172} $\det(\alpha_1,\cdots,\alpha_i + \beta_i,\cdots,\alpha_n)=\det(\alpha_1,\cdots,\alpha_i,\cdots,\alpha_n)+\det(\alpha_1,\cdots,\beta_i,\cdots,\alpha_n)$

				\ding{173} $\forall k\in F,\det(\alpha_1,\cdots,k\cdot \alpha_i,\cdots,\alpha_n)=k \cdot \det(\alpha_1,\cdots,\alpha_i,\cdots,\alpha_n)$

				\ding{173} $\det(\alpha_1,\cdots,\alpha_i,\cdots,\alpha_j,\cdots,\alpha_n)=-\det(\alpha_1,\cdots,\alpha_j,\cdots,\alpha_i,\cdots,\alpha_n)$

				\ding{174} 存在$V$的一组基$\gamma_i,\cdots,\gamma_n,\det(\gamma_1,\cdots,\gamma_n)=1$

				那么我们称$\det$是一个$V$上的$n$阶行列式
			\end{defn}
			由行列式的定义,我们可以推导出行列式的基本性质
			\begin{proposition}
				向量组$\alpha_1,\cdots,\alpha_i,\cdots,\alpha_j,\cdots,\alpha_n$如果有$\alpha_i=\alpha_j$

				那么$\det (\alpha_1,\cdots,\alpha_i,\cdots,\alpha_j,\cdots,\alpha_n)=0$
			\end{proposition}
			\begin{proof}
				$\det (\alpha_1,\cdots,\alpha_i,\cdots,\alpha_j,\cdots,\alpha_n)=-\det (\alpha_1,\cdots,\alpha_j,\cdots,\alpha_i,\cdots,\alpha_n)$

				但因为$\alpha_i=\alpha_j$,所以必有$\det (\alpha_1,\cdots,\alpha_i,\cdots,\alpha_j,\cdots,\alpha_n)=0$
			\end{proof}
			进一步我们可以推出,如果两个变量成系数关系,那么行列式也为零
			\begin{corollary}{存在成比例变量的行列式为零}{}
				向量组$\alpha_1,\cdots,\alpha_i,\cdots,\alpha_j,\cdots,\alpha_n$如果有$\alpha_i=k\alpha_j,k\in F$

				那么$\det (\alpha_1,\cdots,\alpha_i,\cdots,\alpha_j,\cdots,\alpha_n)=0$
			\end{corollary}
			\begin{proof}
				$\det (\alpha_1,\cdots,\alpha_i,\cdots,\alpha_j,\cdots,\alpha_n)=k\cdot \det (\alpha_1,\cdots,\alpha_j,\cdots,\alpha_j,\cdots,\alpha_n)=0$
			\end{proof}
		\subsection{行列式在基上的展开}
			\begin{them}{行列式的展开}{}
				设$F$是一个域,$V$是$F$上的一个线性空间,并且$dim_F V = n$,
				
				$V$上的$n$阶行列式$\det$满足$\det(\gamma_1,\cdots,\gamma_n)=1$,其中$\{\gamma_1,\cdots,\gamma_n\}$是$V$的一组基

				那么,有:
				\begin{equation}
					\det(\alpha_1,\cdots,\alpha_n)=\sum_{\sigma \in S_n} \text{sgn}(\sigma) \prod_{i=1}^n a_{i,\sigma(i)}
				\end{equation}
				其中$\alpha_i = \sum\limits_{j=1}^{n} a_{i,j} \gamma_j$
			\end{them}
			\begin{proof}
				$\det (\alpha_1,\cdots,\alpha_n)=\det\left(\sum\limits_{i_1=1}^{n} a_{1,i_1} \gamma_{i_1},\cdots,\sum\limits_{i_n=1}^{n} a_{n,i_n} \gamma_{i_n}\right)$

				$=\sum\limits_{i_1=1}^{n}\cdots \sum\limits_{i_n=1}^{n} \left(\prod\limits_{k=1}^{n}a_{k,i_k} \det(\alpha_{i_1},\cdots,\alpha_{i_n})\right)$

				$=\sum\limits_{\sigma \in S_n} \left(\prod\limits_{k=1}^{n}a_{k,\sigma(k)} \text{sgn}(\sigma )\right)$
			\end{proof}
			事实上,我们也可以改变第一个求和指标,使之称为一个固定但是可以随意选取的置换
			\begin{corollary}{}{}
				设$F$是一个域,$V$是$F$上的一个线性空间,并且$dim_F V = n$,
				
				$V$上的$n$阶行列式$\det$满足$\det(\gamma_1,\cdots,\gamma_n)=1$,其中$\{\gamma_1,\cdots,\gamma_n\}$是$V$的一组基

				那么,有:
				\begin{equation}
					\det(\alpha_1,\cdots,\alpha_n)=\text{sgn}(\rho )\sum_{\sigma \in S_n} \text{sgn}(\sigma) \prod_{i=1}^n a_{\rho(i),\sigma(i)}
				\end{equation}
				其中$\alpha_i = \sum\limits_{j=1}^{n} a_{i,j} \gamma_j$,$\rho$是一个置换
			\end{corollary}
			\begin{proof}
				$\det(\alpha_1,\cdots,\alpha_n)=\sum_{\tau  \in S_n} \text{sgn}(\tau ) \prod_{i=1}^n a_{i,\tau (i)}$

				对指标作置换$\rho $,累乘的结果不会变化,所以有:

				$\det(\alpha_1,\cdots,\alpha_n)=\sum_{\tau \in S_n} \text{sgn}(\tau ) \prod_{i=1}^n a_{\rho(i),(\rho \circ \tau )(i)}$

				记$\sigma =\rho \circ \tau $,那么$\det(\alpha_1,\cdots,\alpha_n)=\sum_{\rho^{-1} \circ \sigma  \in S_n} \text{sgn}(\rho^{-1} \circ \sigma) \prod_{i=1}^n a_{\rho(i),\sigma (i)}$

				但是,$\rho^{-1} \circ \sigma \in S_n$其实就是$\sigma \in S_n$,并且我们知道$\text{sgn}(\rho^{-1} \circ \sigma)=\text{sgn}(\rho )\text{sgn}(\sigma )$

				所以$\det(\alpha_1,\cdots,\alpha_n)=\text{sgn}(\rho )\sum_{\sigma \in S_n} \text{sgn}(\sigma) \prod_{i=1}^n a_{\rho(i),\sigma(i)}$
			\end{proof}
		\subsection{矩阵的行列式}
			我们之前已经指出,$M_n (F) \cong F^{n^2} \cong (F^n)^n$,因此,我们可以对矩阵定义行列式:
			\begin{defn}{矩阵的行列式}{}
				设矩阵$A = (\alpha_1,\cdots,\alpha_n)\in M_n (F)$,我们定义:

				$|A|=\det (A) := \det(\alpha_1,\cdots,\alpha_n)$

				并且有$\det(e_1,\cdots,e_n)=1$,其中$e_i$是标准基向量$(0,\cdots,1,\cdots,0)$,$1$在第$i$个位置上。
			\end{defn}
			矩阵的行列式也可以类似地在标准基上展开
			\begin{them}{矩阵的行列式的展开}{}
				设$F$是一个域,矩阵$A=(a_{ij}) \in M_{n\times n}(F)$
				那么,有:
				\begin{equation}
					|A|=\sum_{\sigma \in S_n} \text{sgn}(\sigma) \prod_{i=1}^n a_{i,\sigma(i)}
				\end{equation}
			\end{them}
		\subsection{矩阵的行列式的余子式展开}
		\subsection{矩阵乘积的行列式}
\ifx\allfiles\undefined
\end{document}
\fi