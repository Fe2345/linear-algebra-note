\ifx\allfiles\undefined
\documentclass[12pt, a4paper, oneside, UTF8]{ctexbook}
\def\path{../config}
\usepackage{amsmath}
\usepackage{amsthm}
\usepackage{amssymb}
\usepackage{graphicx}
\usepackage{mathrsfs}
\usepackage{enumitem}
\usepackage{geometry}
\usepackage[colorlinks, linkcolor=black]{hyperref}
\usepackage{stackengine}
\usepackage{yhmath}
\usepackage{extarrows}
\usepackage{unicode-math}
\usepackage{tikz}
\usepackage{tikz-cd}
\usepackage{pifont}
\usepackage{pgfplots}
\usepackage{tikz-3dplot}

\usepackage{fancyhdr}
\usepackage[dvipsnames, svgnames]{xcolor}
\usepackage{listings}

\definecolor{mygreen}{rgb}{0,0.6,0}
\definecolor{mygray}{rgb}{0.5,0.5,0.5}
\definecolor{mymauve}{rgb}{0.58,0,0.82}

\graphicspath{ {figure/},{../figure/}, {config/}, {../config/} }

\linespread{1.6}

\geometry{
    top=25.4mm, 
    bottom=25.4mm, 
    left=20mm, 
    right=20mm, 
    headheight=2.17cm, 
    headsep=4mm, 
    footskip=12mm
}

\setenumerate[1]{itemsep=5pt,partopsep=0pt,parsep=\parskip,topsep=5pt}
\setitemize[1]{itemsep=5pt,partopsep=0pt,parsep=\parskip,topsep=5pt}
\setdescription{itemsep=5pt,partopsep=0pt,parsep=\parskip,topsep=5pt}

\lstset{
    language=Mathematica,
    basicstyle=\tt,
    breaklines=true,
    keywordstyle=\bfseries\color{NavyBlue}, 
    emphstyle=\bfseries\color{Rhodamine},
    commentstyle=\itshape\color{black!50!white}, 
    stringstyle=\bfseries\color{PineGreen!90!black},
    columns=flexible,
    numbers=left,
    numberstyle=\footnotesize,
    frame=tb,
    breakatwhitespace=false,
} 
\usepackage[strict]{changepage} 
\usepackage{framed}
\usepackage{tcolorbox}
\tcbuselibrary{most}

\definecolor{greenshade}{rgb}{0.90,1,0.92}
\definecolor{redshade}{rgb}{1.00,0.88,0.88}
\definecolor{brownshade}{rgb}{0.99,0.95,0.9}
\definecolor{lilacshade}{rgb}{0.95,0.93,0.98}
\definecolor{orangeshade}{rgb}{1.00,0.88,0.82}
\definecolor{lightblueshade}{rgb}{0.8,0.92,1}
\definecolor{purple}{rgb}{0.81,0.85,1}

% #### 将 config.tex 中的定理环境的对应部分替换为如下内容
% 定义单独编号,其他四个共用一个编号计数 这里只列举了五种,其他可类似定义(未定义的使用原来的也可)
\newtcbtheorem[number within=section]{defn}%
{定义}{colback=OliveGreen!10,colframe=Green!70,fonttitle=\bfseries}{def}

\newtcbtheorem[number within=section]{lemma}%
{引理}{colback=Salmon!20,colframe=Salmon!90!Black,fonttitle=\bfseries}{lem}

% 使用另一个计数器 use counter from=lemma
\newtcbtheorem[use counter from=lemma, number within=section]{them}%
{定理}{colback=SeaGreen!10!CornflowerBlue!10,colframe=RoyalPurple!55!Aquamarine!100!,fonttitle=\bfseries}{them}

\newtcbtheorem[use counter from=lemma, number within=section]{criterion}%
{准则}{colback=green!5,colframe=green!35!black,fonttitle=\bfseries}{cri}

\newtcbtheorem[use counter from=lemma, number within=section]{corollary}%
{推论}{colback=Emerald!10,colframe=cyan!40!black,fonttitle=\bfseries}{cor}
% colback=red!5,colframe=red!75!black

% 这个颜色我不喜欢
%\newtcbtheorem[number within=section]{proposition}%
%{命题}{colback=red!5,colframe=red!75!black,fonttitle=\bfseries}{cor}

% .... 命题 例 注 证明 解 使用之前的就可以(全文都是这种框框就很丑了),也可以按照上述定义 ...
\renewenvironment{proof}{\par\textbf{证明:}\;}{\qed\par}
\newenvironment{solution}{\par{\textbf{解:}}\;}{\qed\par}
\newtheorem{proposition}{\indent 命题}[section]
\newtheorem{example}{\indent \color{SeaGreen}{例}}[section] % 绿色文字的 例 ,不需要就去除\color{SeaGreen}{}
\newtheorem*{rmk}{\indent 注}
\usepackage{amssymb}
\setmathfont{LatinModernMath-Regular}
\setmathfont[range=\mathbb]{TeXGyrePagellaMath-Regular}
\def\d{\mathrm{d}}
\def\R{\mathbb{R}}
\def\C{\mathbb{C}}
\def\Q{\mathbb{Q}}
\def\N{\mathbb{N}}
\def\Z{\mathbb{Z}}
\newcommand{\bs}[1]{\boldsymbol{#1}}
\newcommand{\ora}[1]{\overrightarrow{#1}}
\newcommand{\myspace}[1]{\par\vspace{#1\baselineskip}}
\newcommand{\xrowht}[2][0]{\addstackgap[.5\dimexpr#2\relax]{\vphantom{#1}}}
\newenvironment{ca}[1][1]{\linespread{#1} \selectfont \begin{cases}}{\end{cases}}
\newenvironment{vx}[1][1]{\linespread{#1} \selectfont \begin{vmatrix}}{\end{vmatrix}}
\newcommand{\tabincell}[2]{\begin{tabular}{@{}#1@{}}#2\end{tabular}}
\newcommand{\pll}{\kern 0.56em/\kern -0.8em /\kern 0.56em}
\newcommand{\dive}[1][F]{\mathrm{div}\;\bs{#1}}
\newcommand{\rotn}[1][A]{\mathrm{rot}\;\bs{#1}}
\usepackage{xeCJK}
\setCJKmainfont{SimSun}[BoldFont={SimHei}, ItalicFont={KaiTi}] % 设置中文支持

\newcommand{\point}[1]{\item {#1}}
\newenvironment{para}[1]{%
\ifcase#1\relax
\begin{enumerate}[label=\arabic*.] % 1.2.3.
\or
\begin{enumerate}[label=\textcircled{\arabic*}] % ①②③
\or
\begin{enumerate}[label=(\roman*)] % (i)(ii)(iii)
\else
\begin{enumerate}[label=\arabic*.] % 默认格式
\fi
}{
\end{enumerate}
}

\def\myIndex{0}
% \input{\path/cover_package_\myIndex.tex}

\def\myTitle{高等代数笔记}
\def\myAuthor{Zhang Liang}
\def\myDateCover{\today}
\def\myDateForeword{\today}
\def\myForeword{前言标题}
\def\myForewordText{
    前言内容
}
\def\mySubheading{副标题}


\begin{document}
% \input{\path/cover_text_\myIndex.tex}

\newpage
\thispagestyle{empty}
\begin{center}
    \Huge\textbf{\myForeword}
\end{center}
\myForewordText
\begin{flushright}
    \begin{tabular}{c}
        \myDateForeword
    \end{tabular}
\end{flushright}

\newpage
\pagestyle{plain}
\setcounter{page}{1}
\pagenumbering{Roman}
\tableofcontents

\newpage
\pagenumbering{arabic}
\setcounter{chapter}{0}
\setcounter{page}{0}

\pagestyle{fancy}
\fancyfoot[C]{\thepage}
\renewcommand{\headrulewidth}{0.4pt}
\renewcommand{\footrulewidth}{0pt}








\else
\fi
%标题
\chapter{线性映射}
	本章开始,我们转向线性映射的研究。

	我们将用三章完成线性映射的研究。本章我们将从映射最基本的研究方式:向量的作用开始

	线性映射的独特之处在于:一方面它能在常规的映射加法和纯量乘法下构成一个线性空间;另一方面,如果我们将复合视为乘法,它可以构成一个幺环。

	第一节中,我们将给出线性映射的定义及运算,并研究基本性质;
	
	第二节中,我们将研究两种由线性映射导出的子空间,核和像,并借此提出一个概念:秩。它和我们之前的秩也有很强的联系;

	第三节到第五节中,我们将研究矩阵,它是将线性映射在基下的作用写成的一张数表,非常便于在数值上研究矩阵;

	第六节中,我们将研究行列式,它是一个反对称多线性函数,我们以此为工具,为后续我们对线性映射分解的研究铺垫。
	\section{线性映射的定义和运算}
		\subsection{线性映射的定义}
			我们首先给出线性映射的定义
			\begin{defn}{线性映射}{}
				设$V_1,V_2$是一个$F$上的两个线性空间,,映射$A:V_1 \to V_2$如果满足:

				$\forall \alpha ,\beta \in V_1,k \in F$

				$A(\alpha+\beta )=A(\alpha )+A(\beta )$

				$A(k\alpha )=kA(\alpha )$

				那么我们称$A$是一个从$V_1$到$V_2$的线性映射

				全体$V_1$到$V_2$的线性映射的集合记作$\hom_F (V_1,V_2)$,或简记作$\hom (V_1,V_2)$

				特别地,如果$V_1=V_2$,我们称$A$是一个$V_1$上的线性变换
			\end{defn}
			有一些常用的线性映射,我们在这里列出来:
			\begin{defn}{一些常用的线性映射}{}
				\begin{enumerate}
					\item 恒等变换:$I:V \ni \alpha \mapsto \alpha \in V$
					\item 数乘变换:$k:V \ni \alpha \mapsto k\alpha \in V$
					\item 零变换:$0:V_1 \ni \alpha \mapsto \mathbf{0}_{V_2} \in V_2$
				\end{enumerate}
			\end{defn}
		\subsection{线性映射的运算}
			前面我们定义了线性映射,现在我们开始赋予$\hom (V_1,V_2)$线性空间和环的性质。

			我们会定义三种运算:加法、纯量乘法、乘法
			\begin{defn}{线性映射的运算}{}
				我们定义:
				
				映射$+:\hom(V_1,V_2) \times \hom(V_1,V_2) \rightarrow \hom(V_1,V_2)$,称为加法,如果满足:
				\begin{equation}
					\forall A,B \in \hom(V_1,V_2),\alpha \in V_1,(A+B)(\alpha )=A(\alpha )+B(\alpha )
				\end{equation}
				映射$\cdot:F \times \hom(V_1,V_2) \rightarrow \hom(V_1,V_2)$,称为纯量乘法,如果满足:
				\begin{equation}
					\forall k \in F,A \in \hom(V_1,V_2),\alpha \in F,(k\cdot A)(\alpha )=kA(\alpha )
				\end{equation}
				映射$\circ:\hom(V_1,V_2) \times \hom(V_1,V_2) \rightarrow \hom(V_1,V_2)$,称为乘法,如果满足:
				\begin{equation}
					\forall A,B \in \hom(V_1,V_2),\alpha \in V_1,(A\circ B)(\alpha )=A\left(B(\alpha )\right)
				\end{equation}
			\end{defn}
			我们也常常把$k\cdot A$简记为$kA$,将$A \circ B$简记为$AB$

			显然,$\left(\hom(V_1,V_2),F,+,\cdot\right)$是一个线性空间,$0$是它的零向量;

			$\left(\hom(V_1,V_2),+,\circ\right)$是一个幺环,$0$是它的加法单位元,$I$是它的乘法单位元

			除此之外,还有一些运算,但是它们是针对特殊的线性映射的,比如说:
			\begin{defn}{线性变换的幂}{}
				$\forall A \in \hom(V,V)$

				我们定义:$A^m:= \begin{cases}
					A\circ A^{m-1},m \geqslant 1 \\
					I,m = 0
				\end{cases},m \geqslant 0$
			\end{defn}
			如果一个映射的幂不会使其本身变化,我们称它是一个幂等变换
			\begin{defn}{幂等映射}{}
				$A \in \hom(V,V)$如果有:

				$A = A^2$

				我们称它是一个幂等变换
			\end{defn}
			我们不再讨论其他的运算,我们接下来转入线性映射一般性质的研究
		\subsection{线性映射的性质}
			\begin{para}{0}
				\point{}
					\begin{proposition}
						$\forall A \in \hom(V_1,V_2),A(\mathbf{0}_{V_1}) = \mathbf{0}_{V_2}$
					\end{proposition}
					\begin{proof}
						$A(\mathbf{0}_{V_1})=A(0\cdot \mathbf{0}_{V_1})=0\cdot A(\mathbf{0}_{V_1})=\mathbf{0}_{V_2}$
					\end{proof}
				\point{}
					\begin{proposition}
						$\forall A \in \hom(V_1,V_2),A(-\alpha )=-A(\alpha )$
					\end{proposition}
					\begin{proof}
						$A(-\alpha )=A((-1)\cdot \alpha )=(-1)\cdot A(\alpha )=-A(\alpha )$
					\end{proof}
				\point{}
					\begin{proposition}
						$\forall A \in \hom(V_1,V_2),A(\sum\limits_{i=1}^{n} k_i \alpha_i)=\sum\limits_{i=1}^{n} k_i A(\alpha_i)$
					\end{proposition}
					\begin{proof}
						对$n$使用数学归纳法易证。
					\end{proof}
					值得注意,这个定理并不能随意地推广到$\aleph_0$,因为此时依赖于度量线性空间或线性映射的进一步性质。
				\point{}
					\begin{proposition}
						$\forall A \in \hom(V,V),m \geqslant 1, A^m = A^{m-1} \circ A$
					\end{proposition}
					\begin{proof}
						对$m$作数学归纳法。

						首先,当$m=2$时,$A^2 = A \circ A$,命题成立

						现在假设$m$时成立,我们来证明$m+1$时命题也成立:

						$A^m = A \circ A^{m-1} = A \circ A^{m-2} \circ A = A^{m-1} \circ A$,于是命题得证。
					\end{proof}
					这个命题看似显然,但是是必要的,因为线性映射环不交换。这个命题指出:递归式地推导幂时,从左右方向都是等价的。进一步,在递归中不断变换方向也不会影响结果。
				\point{}
					\begin{proposition}
						$\forall A \in \hom(V_1,V_2)$

						如果$\alpha_1,\cdots,\alpha_s$线性相关,那么$A(\alpha_1),\cdots,A(\alpha_s)$也线性相关
					\end{proposition}
					\begin{proof}
						$\alpha_1,\cdots,\alpha_s$线性相关

						$\Rightarrow \exists k_1,\cdots,k_s,k_1\alpha_1+\cdots+k_n\alpha_s=\mathbf{0}$,其中$k_1,\cdots,k_s$不全为零

						$\Rightarrow A(k_1\alpha_1+\cdots+k_n\alpha_s)=k_1A(\alpha_1)+\cdots+k_nA(\alpha_s)=\mathbf{0}$

						$\Rightarrow A(\alpha_1),\cdots,A(\alpha_n)$线性无关
					\end{proof}
					值得注意的是,不同于同构映射,在这个命题中把线性相关改为线性无关会使命题变得不成立。比如说,零映射会把任何线性无关的向量组变得线性相关
				\point{}
					\begin{proposition}
						$\forall T,Y \in \hom(V_1,V_2)$,$B$是$V_1$的一个基

						如果$\forall \alpha \in B,T(\alpha )=Y(\beta )$,那么$T=Y$
					\end{proposition}
					\begin{proof}
						只需证明:$\forall \gamma \in V_1,T(\gamma )=Y(\gamma )$

						因为$B$是$V_1$的基,所以一定有$\gamma =k_1 \alpha_1+\cdots+k_s \alpha_s,\alpha_1,\cdots,\alpha_s \in B$

						此时,$T(\gamma )=k_1 T(\alpha_1)+\cdots+ k_s T(\alpha_s)=k_1 Y(\alpha_1)+\cdots+ k_s Y(\alpha_s)=Y(\gamma )$,于是命题得证。
					\end{proof}
					这个命题指出,线性映射完全由其在基上的作用决定,因为我们其实只需要指定基上的像就指定了线性映射本身。
			\end{para}
			最后,还有一个问题未被解决:$\hom(V,W)$的维数和$\dim V,\dim W$有什么关系。这个问题比较复杂,因为涉及无限维时,维数的公式会变得和有限情形完全不同。在讲解行列式后,我们将在附录一中解决这个问题。
	\section{线性映射的核和像}
		本节中,我们将借助核与像继续研究线性映射。所谓核,即是线性映射映到零的那一部分;像则是值域。

		同时,我们会引入对偶映射,借助本节中核与像的工具,我们将看到对偶和本身之间的联系。
			\subsection{核与像的定义}
				\begin{defn}{线性映射的核}{}
					设$A \in \hom(V_1,V_2)$,我们定义:
					\begin{equation}
						\ker A := \{\alpha \in V_1 | A(\alpha )=\mathbf{0}_{V_2}\}
					\end{equation}

					称为线性映射$A$的核
				\end{defn}
				\begin{defn}{线性映射的像}{}
					设$A \in \hom(V_1,V_2)$,我们定义:
					\begin{equation}
						\im A := A(V_1) := \{A(\alpha)| \alpha \in V_1\}
					\end{equation}·

					称为线性映射$A$的像
				\end{defn}
				特别地,核与像的维数我们分别称为零化度和秩:
				\begin{defn}{线性映射的零化度}{}
					设$A \in \hom(V_1,V_2)$,我们定义:
					\begin{equation}
						\operatorname{nullity} (A) := \dim(\ker A)
					\end{equation}
					称为线性映射$A$的零化度
				\end{defn}
				\begin{defn}{线性映射的秩}{}
					设$A \in \hom(V_1,V_2)$,我们定义:
					\begin{equation}
						\rank (A) := \dim(\im A)
					\end{equation}
					称为线性映射$A$的秩
				\end{defn}
				事实上,线性映射的秩和之前我们曾提及的向量组的秩有着很大的联系,我们将在后续看到这一点。

				为了方便后续性质的研究,接下来我们给出对偶映射的概念
				\begin{defn}{对偶映射}{}
					设$T \in \hom(V_1,V_2)$,我们定义对偶映射$T^* \in \hom (V_2^*,V_1^*)$

					如果满足:$\forall f \in V_2^*,\alpha \in V_1$
					\begin{equation}
							\left(T^*(f)\right)(\alpha )=(f\circ T)(\alpha )
					\end{equation}
				\end{defn}
			\subsection{核与像的性质}
				接下来研究核与像的性质
				\begin{para}{0}
					\point{}
						\begin{proposition}
							$\forall A \in \hom(V,W),\ker A,\im A$都是线性空间
						\end{proposition}
						\begin{proof}
							先证明$\ker A$是一个线性空间

							注意到,$\mathbf{0}_V \in \ker A$,因为$A(\mathbf{0}_V)=\mathbf{0}_W$,因此$\ker A$非空

							那么,只需注意到$\forall \alpha ,\beta \in \ker A,A(\alpha +\beta )=A(\alpha )+A(\beta )=\mathbf{0}_{W} \Rightarrow \alpha + \beta \in \ker A$

							$\forall \alpha \in \ker A,k \in F,A(k\alpha )=kA(\alpha )=\mathbf{0}_W \Rightarrow k\alpha \in \ker A$

							再证明$\im A$是一个线性空间

							注意到,$\mathbf{0}_W = A(\mathbf{0}_V) \in \im A \Rightarrow \im A \neq \emptyset$

							那么,只需注意到$\forall A(\alpha) ,A(\beta) \in \im A,A(\alpha )+A(\beta )=A(\alpha +\beta )\in \im A$

							$\forall A(\alpha )\in \im A,k \in F,kA(\alpha )=A(k\alpha )\in \im A$

							于是命题得证
						\end{proof}
					\point{}
						\begin{proposition}
							$A \in \hom(V,W)$,那么:

							$A$是单射$\Leftrightarrow \ker A = \{\mathbf{0}_V\}$
						\end{proposition}
						\begin{proof}
							先证明充分性。注意到,$A(\mathbf{0}_V)=\mathbf{0}_W$,而$A$为单射,于是一定有$\ker A = \{\mathbf{0}_V\}$

							再证明必要性。设$A(\alpha )=A(\beta )$,那么$A(\alpha -\beta )=\mathbf{0}_W \Rightarrow \alpha -\beta =\mathbf{0}_V \Rightarrow \alpha =\beta \Rightarrow A$是单射
						\end{proof}
						\point{}
						\begin{proposition}
							$A \in \hom(V,W)$,那么:

							$A$是单射$\Leftrightarrow \im A = W$
						\end{proposition}
						\begin{proof}
							这是显然的。
						\end{proof}
					\point{秩-零化度定理}
						在给出定理前,我们先给出一个引理。它是一个线性空间的“同态基本定理”,证明方法也很相似
						\begin{lemma}{}{}
							$\forall A \in \hom(V,W)$
							\begin{equation}
								V/\ker A \cong \im A
							\end{equation}
						\end{lemma}
						\begin{proof}
							设$\phi : V/\ker A \ni \alpha + \ker A \mapsto A(\alpha ) \in \im A$

							首先验证它的确是一个映射

							假设$\alpha +\ker A=\beta +\ker A$

							那么$\alpha -\beta \in \ker A \Rightarrow A(\alpha -\beta )=\mathbf{0}_W \Rightarrow A(\alpha )=A(\beta )$

							所以$\phi $的确是一个映射

							接下来验证它是一个单射

							设$\phi (\alpha + \ker A)=\phi (\beta +\ker A)$
							
							$\Rightarrow A(\alpha )=A(\beta ) \Rightarrow A(\alpha -\beta )=\mathbf{0}_W$

							$\Rightarrow \alpha - \beta  \in \ker A \Rightarrow \alpha +\ker A=\beta +\ker A$

							于是$\phi $是单射。$\phi $显然是满射

							接下来验证线性性:

							$\forall \alpha +\ker A,\beta +\ker A \in V/\ker A,k \in F$
							
							$\phi \left((\alpha +\ker A)+(\beta +\ker A)\right)=\phi \left((\alpha +\beta )+\ker A\right)=A(\alpha +\beta )=A(\alpha )+A(\beta )=\phi (\alpha +\ker A)+\phi (\beta +\ker A)$
						
							$\phi \left(k\cdot (\alpha +\ker A)\right)=\phi (k\alpha +\ker A)=A(k\alpha )=kA(\alpha )=k\phi (\alpha +\ker A)$
						
							于是命题得证。
						\end{proof}

						它的直接推论是被称为秩-零化度定理的结论,它揭示了秩和零化度的联系
						\begin{them}{秩-零化度定理}{}
							$\forall A \in \hom(V,W)$
							\begin{equation}
								\rank(A)+\operatorname{nullity}(A) = \dim V
							\end{equation}
						\end{them}
						\begin{proof}
							$V/\ker A \cong \im A$

							$\Rightarrow \dim(V/\ker A)=\dim \im A$

							$\Rightarrow \dim V - \dim \ker A = \dim \im A$

							$\Rightarrow \rank(A)+\operatorname{nullity}(A)= \dim V$
						\end{proof}
					\point{有限维映射和其对偶的秩相同}
						我们先证明一个引理:
						\begin{lemma}{子空间及其零化子维数和为原空间维数}{}
							设$V$是一个$F$上的线性空间,$\dim V = n < \aleph_0$,$U$是$V$的一个线性子空间

							我们定义$U'=\{A \in V^*| \forall \alpha \in U,A(\alpha )=0\}$

							那么有:$\dim U+\dim U'=\dim V$
						\end{lemma}
						\begin{proof}
							设$\dim U = m,0 \leqslant m\leqslant n$

							取$U$的一个基$\{\alpha_1,\cdots,\alpha_m\}$,并补全为$V$的一个基$\{\alpha_1,\cdots,\alpha_n\}$

							设$\{\alpha_1,\cdots,\alpha_n\}$的对偶基为$\{f_1,\cdots,f_n\}$

							我们来证明:$\{f_{m+1},\cdots,f_n\}$是$U'$的一个基

							取$\forall A \in U'$。因为$U' \subseteq V^*$,那么$A$一定可以被对偶基线性表出

							不妨设$A = \sum\limits_{i=1}^{n} k_i f_i$

							注意到,当$1 \leqslant j \leqslant m,A(\alpha_j)=\sum\limits_{i=1}^{n} k_i f_i(\alpha_j)=k_j$

							但因为$\alpha_j \in U$,因此$k_j = A(\alpha_j)=0$

							因此有$A = \sum\limits_{i=m+1}^{n} k_i f_i$

							这说明$U'$可由$\{f_{m+1},\cdot,f_n\}$线性表出。又因为$\{f_1,\cdots,f_n\}$线性无关,因此$\{f_{m+1},\cdot,f_n\}$也线性无关

							因此$\{f_{m+1},\cdot,f_n\}$是$U'$的一个基,那么$\dim U'=n-m$,命题得证。
						\end{proof}
						接下来给出命题
						\begin{proposition}
							设$T \in \hom(V,W),\dim V=m < \aleph_0,\dim W = n< \aleph_0$

							那么$\rank(T)=\rank(T^*)$
						\end{proposition}
						\begin{proof}
							只需注意到,$\ker T^* = \{f \in W^*| T^*(f) = \mathbf{0}_{V^*}\}$

							$=\{f \in W^*| f \circ T = \mathbf{0}_{V^*}\}$

							$=\{f \in W^*| \forall \alpha \in V,f(T(\alpha )) = 0\}$

							$=\{f \in W^*| \forall \beta  \in \im T,f(\beta ) = 0\}$

							而按照引理,有:$\dim \im T + \dim \{f \in W^*| \forall \beta  \in \im T,f(\beta ) = 0\} = \dim W$

							$\Rightarrow \dim \im T +\dim \ker T^* = \dim W$

							再运用秩-零化度定理,有:$\dim \im T^* + \dim \ker T^* = \dim W^* =\dim W$

							因此有$\dim \im T = \dim \im T^* \Rightarrow \rank(T)=\rank(T^*)$,命题得证
						\end{proof}
					\point{}
						\begin{proposition}
							设$A \in \hom(V,W)$,如果$\dim V=\dim W$,那么:

							$A$是满射$\Leftrightarrow A$是单射
						\end{proposition}
						\begin{proof}
							因为$V/\ker A \cong \im A$

							$A$是满射$\Leftrightarrow \im A = W \Leftrightarrow \im A \cong W \Leftrightarrow \im A \cong W \cong V$

							$\Leftrightarrow V \cong V/\ker A \Leftrightarrow \dim \ker A = 0 \Leftrightarrow \ker A = \{\mathbf{0}_V\} \Leftrightarrow A$是单射
						\end{proof}
				\end{para}
	\section{矩阵}
		为了方便后续的研究,我们引入一种“数表”,即矩阵。它和有限维空间上的线性映射完全同构
		\subsection{矩阵的定义}
			\begin{defn}{矩阵}{}
				形如以下的矩形阵列称为一个域$F$上的矩阵
				$\begin{pmatrix}
					a_{11} & \cdots & a_{1n} \\
					\vdots & \ddots & \vdots \\
					a_{m1} & \cdots & a_{mn}
				\end{pmatrix},a_{ij} \in F$

				简记为$(a_{ij})_{m \times n}$或$(a_{ij})$。$m$称为矩阵的行数,$n$称为矩阵的列数。

				特别地,如果$m=n$,我们称它是一个$m$阶方阵。

				$F$上的全体$m \times n$矩阵的集合记作$M_{m \times n} (F)$,特别地如果$m=n$,记作$M_n (F)$。

				我们也将矩阵$\symbfit{A}$在$m$行$n$列处的元素记作$\symbfit{A}_{ij}$
			\end{defn}
			我们也常常将$\symbfit{A}=(a_{ij})_{m \times n}$记作$(\alpha_1,\cdots,\alpha_n)$或$\begin{pmatrix}
				\beta_1 \\
				\vdots \\
				\beta_m
			\end{pmatrix}$,其中$\alpha_i = \begin{pmatrix}
				a_{1i} \\
				\vdots \\
				a_{mi}
			\end{pmatrix},\beta_j = (a_{j1},\cdots,a_{jn})$
			可以看出来,矩阵也是可以视为一个向量组,因此,我们也可以对其定义秩:
			\begin{defn}{矩阵的秩}{}
				设$\symbfit{A}=(\alpha_1,\cdots,\alpha_n) \in M_{m \times n}(F)$

				我们定义:
				\begin{equation}
					\rank(\symbfit{A})=\rank(\alpha_1,\cdots,\alpha_n)
				\end{equation}
				称为矩阵的秩
			\end{defn}
			后续我们会看到,矩阵秩、线性映射秩、向量组秩其实是统一的
		\subsection{矩阵的运算}
		\begin{para}{0}
			\point{相等}
			    \begin{defn}{矩阵的相等}{}
				    设$\symbfit{A}\in M_{m\times n}(F),\symbfit{B}\in M_{m \times n}(F)$,如果$\forall i,j,\symbfit{A}(i;j)=\symbfit{B}(i;j)$,则称$\symbfit{A}=\symbfit{B}$。
				\end{defn}
			\point{转置}
				\begin{defn}{矩阵的转置}{}
				    设$\symbfit{A}\in M_{m \times n}(F)$,

					我们定义矩阵$\symbfit{A}^T \in M_{n \times m}(F)$为满足$\symbfit{A}^T(i;j)=\symbfit{A}(j;i)$的矩阵,称为$\symbfit{A}$的转置。
				\end{defn}
			\point{加法}
				\begin{defn}{矩阵的加法}{}
				    设$\symbfit{A}\in M_{m \times n}(F),\symbfit{B}\in M_{m \times n}(F)$
					
					我们定义:$(\symbfit{A}+\symbfit{B})(i;j)=\symbfit{A}(i;j)+\symbfit{B}(i;j)$。
				\end{defn}
			\point{纯量乘法}
				\begin{defn}{矩阵的纯量乘法}{}
				    设$\symbfit{A}\in M_{m \times n}(F),k \in F$,

					我们定义矩阵$k\cdot \symbfit{A} \in M_{m\times n}(F)$为满足$(k\cdot  \symbfit{A})(i;j)=k\cdot \symbfit{A}(i;j)$的矩阵。
				\end{defn}
			\point{乘法}
				\begin{defn}{矩阵的乘法}{}
				    设$\symbfit{A}\in M_{m\times n}(F),\symbfit{B}\in M_{n \times p}(F)$,

					我们定义矩阵$\symbfit{A}\cdot \symbfit{B} \in M_{m\times p}(F)$为满足$(\symbfit{A}\cdot \symbfit{B})(i;j)=\sum_{k=1}^n \symbfit{A}(i;k)\symbfit{B}(k;j)$的矩阵
				\end{defn}
			\point{幂}
				\begin{defn}{方阵的幂}{}
					$\forall \symbfit{A} \in M_{n}(F)$

					我们定义:$\symbfit{A}^m:= \begin{cases}
						\symbfit{A} \cdot \symbfit{A}^{m-1},m \geqslant 1 \\
						\symbfit{I},m = 0
					\end{cases},m \geqslant 0$
				\end{defn}
			\point{和行向量的乘法}
				
			最后,为了后续研究线性映射,我们再定义一种新运算
				\begin{defn}{$V^n$中的行向量和矩阵的乘积}{}
					设$(\alpha_1,\cdots,\alpha_m)\in V^m,\symbfit{A}=\begin{pmatrix}
					a_{11} & \cdots & a_{1n} \\
					\vdots & \ddots & \vdots \\
					a_{m1} & \cdots & a_{mn}
					\end{pmatrix}\in M_{m \times n}(F)$,
					
					我们定义:

					$(\alpha_1,\cdots,\alpha_m) \cdot \symbfit{A} = (\sum_{i=1}^m a_{i1}\alpha_i  , \cdots , \sum_{i=1}^m a_{im}\alpha_i)$
				\end{defn}
				值得注意的是,这个运算并不是矩阵乘法的退化,因为此处乘法的左元中的向量是一个一般的向量,而不一定是列向量
		\end{para}
		\subsection{线性映射的矩阵}
			给出矩阵及其运算的基本定义后,我们可以开始讨论如何将线性映射写成矩阵了。
			\begin{defn}{线性映射的矩阵}{}
				设$A \in \hom(V,W),\dim V = m < \aleph_0,\dim W = n < \aleph_0$

				取$V$的一个基$\{\alpha_1,\cdots,\alpha_m\}$,$W$的一个基$\{\beta_1,\cdots,\beta_n\}$

				如果矩阵$\symbfit{A}=(a_{ij}) \in M_{m \times n}(F)$满足:

				\begin{equation}
					\left(A(\alpha_1),\cdots,A(\alpha_m)\right) = (\beta_1,\cdots,\beta_n)\begin{pmatrix}
						a_{11} & \cdots & a_{1m} \\
						\vdots & \ddots & \vdots \\
						a_{n1} & \cdots & a_{nm}
					\end{pmatrix}
				\end{equation}
				那么我们称$\symbfit{A}$是线性映射$A$在$\{\alpha_1,\cdots,\alpha_m\}$和$\{\beta_1,\cdots,\beta_n\}$下的矩阵

				特别地,如果$V=W$且$\alpha_i = \beta_i$,我们称$\symbfit{A}$是线性变换$A$在$\{\alpha_1,\cdots,\alpha_m\}$下的矩阵
			\end{defn}
			我们也常常简记$\left(A(\alpha_1),\cdots,A(\alpha_m)\right)$为$A(\alpha_1,\cdots,\alpha_m)$

			我们习惯上用大写字母表示线性映射,而用加粗的大写字母表示它在某个基下的矩阵
			
			我们接下来会看到,矩阵的运算其实就是线性映射的运算
		\subsection{矩阵的性质}
			\begin{para}{0}
				\point{}
					\begin{proposition}
						$(\alpha_1,\cdots,\alpha_m)(\symbfit{A}+\symbfit{B})=(\alpha_1,\cdots,\alpha_m)\symbfit{A}+(\alpha_1,\cdots,\alpha_m)\symbfit{B}$
						
						$(\alpha_1,\cdots,\alpha_m)(k\symbfit{A})=k\left((\alpha_1,\cdots,\alpha_m)\symbfit{A}\right)$

						$(\alpha_1,\cdots,\alpha_m)(\symbfit{A}\symbfit{B})=\left((\alpha_1,\cdots,\alpha_m)\symbfit{A}\right)\symbfit{B}$
					\end{proposition}
					\begin{proof}
						$(\alpha_1,\cdots,\alpha_m)(\symbfit{A}+\symbfit{B})=(\sum_{i=1}^m (\symbfit{A}+\symbfit{B})(i;1)\alpha_i  , \cdots , \sum_{i=1}^m (\symbfit{A}+\symbfit{B})(i;m)\alpha_i)$

						$=(\sum_{i=1}^m \symbfit{A}(i;1)\alpha_i  , \cdots , \sum_{i=1}^m \symbfit{A}(i;m)\alpha_i)+(\sum_{i=1}^m \symbfit{B}(i;1)\alpha_i  , \cdots , \sum_{i=1}^m \symbfit{B}(i;m)\alpha_i)$

						$=(\alpha_1,\cdots,\alpha_m)\symbfit{A}+(\alpha_1,\cdots,\alpha_m)\symbfit{B}$

						$(\alpha_1,\cdots,\alpha_m)(k\symbfit{A})=(\sum_{i=1}^m (k\symbfit{A})(i;1)\alpha_i  , \cdots , \sum_{i=1}^m (k\symbfit{A})(i;m)\alpha_i)$

						$=k(\sum_{i=1}^m \symbfit{A}(i;1)\alpha_i  , \cdots , \sum_{i=1}^m \symbfit{A}(i;m)\alpha_i)$

						$=k\left((\alpha_1,\cdots,\alpha_m)\symbfit{A}\right)$

						设$\symbfit{A} \in M_{m \times n}(F),\symbfit{B} \in M_{n \times s}(F)$
						
						$(\alpha_1,\cdots,\alpha_m)(\symbfit{A}\symbfit{B})=(\sum_{i=1}^m (\symbfit{A}\symbfit{B})(i;1)\alpha_i  , \cdots , \sum_{i=1}^m (\symbfit{A}\symbfit{B})(i;m)\alpha_i)$

						$=(\sum_{i=1}^m (\sum\limits_{j=1}^{n}\symbfit{A}(i;j)\symbfit{B}(j;1))\alpha_i  , \cdots , \sum_{i=1}^m (\sum\limits_{j=1}^{n}\symbfit{A}(i;j)\symbfit{B}(j;m))\alpha_i)$

						$=(\sum\limits_{j=1}^{n}\symbfit{B}(j;1) (\sum_{i=1}^m\symbfit{A}(i;j)\alpha_i)  , \cdots , \sum\limits_{j=1}^{n} \symbfit{B}(j;m)(\sum_{i=1}^m\symbfit{A}(i;j)\alpha_i))$
					
						$=(\sum_{i=1}^m\symbfit{A}(i;1)\alpha_i  , \cdots ,\sum_{i=1}^m\symbfit{A}(i;m)\alpha_i)\symbfit{B}$

						$=\left((\alpha_1,\cdots,\alpha_m)\symbfit{A}\right)\symbfit{B}$
					\end{proof}
				\point{}
					\begin{proposition}
						设$A,B \in \hom(V,W)$,其中$V,W$是$F$上的有限维线性空间

						分别取$V,W$的一个基$\{\alpha_1,\cdots,\alpha_m\},\{\beta_1,\cdots,\beta_n\}$

						设$A,B$在$\{\alpha_1,\cdots,\alpha_m\},\{\beta_1,\cdots,\beta_n\}$下的矩阵分别为$\symbfit{A},\symbfit{B}$

						那么:$A+B$在$\{\alpha_1,\cdots,\alpha_m\},\{\beta_1,\cdots,\beta_n\}$下的矩阵为$\symbfit{A}+\symbfit{B}$
					\end{proposition}
					\begin{proof}
						只需证明:$(A+B)(\alpha_1,\cdots,\alpha_m)=(\beta_1,\cdots,\beta_n)(\symbfit{A}+\symbfit{B})$
					
						注意到,$(A+B)(\alpha_1,\cdots,\alpha_m)=A(\alpha_1,\cdots,\alpha_m)+B(\alpha_1,\cdots,\alpha_m)$

						$=(\beta_1,\cdots,\beta_n)\symbfit{A}+(\beta_1,\cdots,\beta_n)\symbfit{B}=(\beta_1,\cdots,\beta_n)(\symbfit{A}+\symbfit{B})$

						命题得证。
					\end{proof}
				\point{}
					\begin{proposition}
						设$A \in \hom(V,W),k \in F$,其中$V,W$是$F$上的有限维线性空间

						分别取$V,W$的一个基$\{\alpha_1,\cdots,\alpha_m\},\{\beta_1,\cdots,\beta_n\}$

						设$A$在$\{\alpha_1,\cdots,\alpha_m\},\{\beta_1,\cdots,\beta_n\}$下的矩阵为$\symbfit{A}$

						那么:$kA$在$\{\alpha_1,\cdots,\alpha_m\},\{\beta_1,\cdots,\beta_n\}$下的矩阵为$k\symbfit{A}$
					\end{proposition}
					\begin{proof}
						只需证明:$(kA)(\alpha_1,\cdots,\alpha_m)=(\beta_1,\cdots,\beta_n)(k\symbfit{A})$
					
						注意到,$(kA)(\alpha_1,\cdots,\alpha_m)=k\left(A(\alpha_1,\cdots,\alpha_m)\right)$

						$=k\left((\beta_1,\cdots,\beta_n)\symbfit{A}\right)=(\beta_1,\cdots,\beta_n)(k\symbfit{A})$

						命题得证。
					\end{proof}
				\point{}
					\begin{proposition}
						设$A\in \hom(W,U),B \in \hom(V,W)$,其中$V,W$是$F$上的有限维线性空间

						分别取$V,W,U$的一个基$\{\alpha_1,\cdots,\alpha_m\},\{\beta_1,\cdots,\beta_n\},\{\gamma_1,\cdots,\gamma_k\}$

						设$A$在$\{\beta_1,\cdots,\beta_n\},\{\gamma_1,\cdots,\gamma_k\}$下的矩阵为$\symbfit{A}$
						
						$B$在$\{\alpha_1,\cdots,\alpha_m\},\{\beta_1,\cdots,\beta_n\}$下的矩阵为$\symbfit{B}$

						那么:$AB$在$\{\alpha_1,\cdots,\alpha_m\},\{\gamma_1,\cdots,\gamma_k\}$下的矩阵为$\symbfit{A}\symbfit{B}$
					\end{proposition}
					\begin{proof}
						只需证明:$(AB)(\alpha_1,\cdots,\alpha_m)=(\gamma_1,\cdots,\gamma_k)(\symbfit{A}\symbfit{B})$
					
						注意到,$(AB)(\alpha_1,\cdots,\alpha_m)=A\left(B(\alpha_1,\cdots,\alpha_m)\right)$

						$=A((\beta_1,\cdots,\beta_n)\symbfit{B})=A(\sum\limits_{i=1}^{n}\symbfit{B}(i,1)\beta_i,\cdots,\sum\limits_{i=1}^{n}\symbfit{B}(i,m)\beta_i)$

						$=(\sum\limits_{i=1}^{n}\symbfit{B}(i,1)A(\beta_i),\cdots,\sum\limits_{i=1}^{n}\symbfit{B}(i,m)A(\beta_i))=\left(A(\beta_1),\cdots,A(\beta_n)\right)\symbfit{B}$

						$\left((\gamma_1,\cdots,\gamma_k)\symbfit{A}\right)\symbfit{B}=\left((\gamma_1,\cdots,\gamma_k)\right)(\symbfit{A}\symbfit{B})$
						
						命题得证。
					\end{proof}
					至此,我们证明了有限维到有限维的线性映射的运算和矩阵的运算别无二致
				\point{}
					\begin{proposition}
						$\dim_F M_{m \times n}(F) = mn$
					\end{proposition}
					\begin{proof}
						设$\symbfit{J_{ij}} \in M_{m \times n}(F)$满足:

						$\symbfit{J_{ij}}(k;l)=\begin{cases}
							1_F,k=i \wedge l=j \\
							0_F,k \neq i \vee l \neq j
						\end{cases}$

						显然$\{\symbfit{J}_{ij}\}$线性无关

						那么,只需注意到:

						$\forall \symbfit{A} \in M_{m \times n}(F),\symbfit{A} = \sum\limits_{i=1}^{m}\left(\sum\limits_{j=1}^{n}A(i;j)\symbfit{J_{ij}}\right)$
					
						于是$\dim_F M_{m \times n}(F)=\card \{\symbfit{J}_{ij}\}=mn$,命题得证。
					\end{proof}
				\point{}
					\begin{proposition}
						设$V,W$是$F$上的线性空间,$\dim V = n < \aleph_0,\dim W = m < \aleph_0$

						那么:$\hom(V,W) \cong M_{m \times n}(F)$
					\end{proposition}
					\begin{proof}
						注意到:$\dim hom(V,W)=mn$

						依照前面的命题和同构的性质,命题得证。
					\end{proof}
					此命题中我们使用了有关线性映射空间的维数的结论,证明参见附录。
				\point{矩阵乘法的交换律}
					\begin{proposition}
						设$\symbfit{A} \in M_{m \times n}(F),\symbfit{B} \in M_{n \times k}(F),\symbfit{C} \in M_{k \times s}(F)$

						那么有:$\symbfit{A}(\symbfit{B}\symbfit{C})=(\symbfit{A}\symbfit{B})\symbfit{C}$
					\end{proposition}
					\begin{proposition}
						分别任取$F$上的线性空间$V,W,U,H$,使得$\dim V = n,\dim W=m,\dim U = k ,\dim H = s$

						依照前面的命题,一定能找到$A \in \hom(V,W),B \in \hom(U,V),C \in \hom(H,U)$,使得$\symbfit{A},\symbfit{B},\symbfit{C}$分别是它们的矩阵

						注意到,$\symbfit{A}(\symbfit{B}\symbfit{C}),(\symbfit{A}\symbfit{B})\symbfit{C}$分别是$A(BC),(AB)C$的矩阵。
						
						而因为$A(BC)=(AB)C$,所以一定有$\symbfit{A}(\symbfit{B}\symbfit{C})=(\symbfit{A}\symbfit{B})\symbfit{C}$,命题得证。
					\end{proposition}
				\point{矩阵乘法的分配律}
					\begin{proposition}
						如果$\symbfit{A} \in M_{m \times n}(F),\symbfit{B} \in M_{n \times k}(F),\symbfit{C} \in M_{n \times k}(F)$

						那么有:$\symbfit{A}(\symbfit{B}+\symbfit{C})=\symbfit{A}\symbfit{B}+\symbfit{A}\symbfit{C}$;

						如果$\symbfit{A} \in M_{n \times k}(F),\symbfit{B} \in M_{n \times k}(F),\symbfit{C} \in M_{k \times s}(F)$

						那么有:$(\symbfit{A}+\symbfit{B})\symbfit{C}=\symbfit{A}\symbfit{C}+\symbfit{B}\symbfit{C}$
					\end{proposition}
					\begin{proposition}
						分别任取$F$上的线性空间$V,W,U,H$,使得$\dim V = n,\dim W=m,\dim U = k ,\dim H = s$

						先证左分配律:
						
						依照前面的命题,一定能找到$A \in \hom(V,W),B \in \hom(U,V),C \in \hom(U,V)$,使得$\symbfit{A},\symbfit{B},\symbfit{C}$分别是它们的矩阵

						注意到,$\symbfit{A}(\symbfit{B}+\symbfit{C}),\symbfit{A}\symbfit{B}+\symbfit{A}\symbfit{C}$分别是$A(B+C),AB+AC$的矩阵。
						
						而因为$A(B+C)=AB+AC$,所以一定有$\symbfit{A}(\symbfit{B}+\symbfit{C})=\symbfit{A}\symbfit{B}+\symbfit{A}\symbfit{C}$

						再证右分配律:

						依照前面的命题,一定能找到$A \in \hom(U,V),B \in \hom(U,V),C \in \hom(H,U)$,使得$\symbfit{A},\symbfit{B},\symbfit{C}$分别是它们的矩阵

						注意到,$(\symbfit{A}+\symbfit{B})\symbfit{C},\symbfit{A}\symbfit{C}+\symbfit{B}\symbfit{C}$分别是$(A+B)C,AC+BC$的矩阵。
						
						而因为$(A+B)C=AC+BC$,所以一定有$(\symbfit{A}+\symbfit{B})\symbfit{C}=\symbfit{A}\symbfit{C}+\symbfit{B}\symbfit{C}$
						
						命题得证。
					\end{proposition}
					于是,我们证明了$(M_n(F),+,\cdot)$是一个环。事实上,它是一个非交换幺环。

					非交换是显然的,我们接下来讨论它的幺元。
				\point{单位矩阵的性质}
					\begin{proposition}
						设$\symbfit{I_n} \in M_n(F)$满足:$\symbfit{I_n}(i;j)=\delta_{ij}$

						那么有:$\forall \symbfit{A} \in M_{n \times m}(F),\symbfit{I_n}\symbfit{A}=\symbfit{A}$

						$\forall \symbfit{A} \in M_{m \times n}(F),\symbfit{A}\symbfit{I_n}=\symbfit{A}$

						$\forall \symbfit{A} \in M_{n}(F),\symbfit{A}\symbfit{I_n}=\symbfit{I_n}\symbfit{A}=\symbfit{A}$
					\end{proposition}
					\begin{proof}
						事实上,以上都是显然的,因为只需注意到:

						任取一个$n$维线性空间$V$和它的一个基$\{\alpha_1,\cdots,\alpha_n\}$

						$I(\alpha_1,\cdots,\alpha_n)=(\alpha_1,\cdots,\alpha_n)=(\alpha_1,\cdots,\alpha_n)\symbfit{I_n}$

						因此,$\symbfit{A}$是$I$在任意基上的矩阵,因此上述性质是显然的。
					\end{proof}
					上述结论说明:$\symbfit{I_n}$是$M_n(F)$的乘法单位元,它一般也被称为单位矩阵

					在证明过程中,我们还发现:不同于绝大多数线性映射的矩阵和其基的选择有关,恒等变换在任意基下的矩阵都是单位矩阵

					我们将在后续用一节专门研究可逆矩阵
				\point{}
					\begin{proposition}
						$\forall k \in F,\symbfit{A} \in M_{n \times m}(F),\symbfit{B} \in M_{m \times k}(F)$

						$k(\symbfit{A}\symbfit{B})=(k\symbfit{A})\symbfit{B}=\symbfit{A}(k\symbfit{B})$
					\end{proposition}
					\begin{proposition}
						随意取三个$F$上的线性空间$V,W,U$,使得$\dim V = m,\dim W = n,\dim U = k$

						依照前面的命题,一定能找到$A \in \hom(V,W),B \in \hom(U,V)$,使得$\symbfit{A},\symbfit{B}$分别是它们的矩阵

						注意到,$k(\symbfit{A}\symbfit{B}),(k\symbfit{A})\symbfit{B},\symbfit{A}(k\symbfit{B})$分别是$k(AB),(kA)B,A(kB)$的矩阵。
						
						而因为$k(AB)=(kA)B=A(kB)$,所以一定有$k(\symbfit{A}\symbfit{B})=(k\symbfit{A})\symbfit{B}=\symbfit{A}(k\symbfit{B})$
						
						命题得证。
					\end{proposition}
				\point{}
					\begin{proposition}
						$\symbfit{A}^{m} = \symbfit{A} \cdot \symbfit{A}^{m-1} = \symbfit{A}^{m-1}\cdot \symbfit{A},m \geqslant 1$
					\end{proposition}
					\begin{proof}
						只需证明:$\forall A \in \hom(V,V)$,任取$V$的一个基$\{\alpha_1,\cdots,\alpha_n\}$,如果$\symbfit{A}$是$A$在$\{\alpha_1,\cdots,\alpha_n\}$下的矩阵,那么$\symbfit{A}^m$是$A^m$在$\{\alpha_1,\cdots,\alpha_n\}$下的矩阵

						使用数学归纳法。首先,当$m=1$时,结论就假设,显然成立

						现在假设$m$时命题成立,于是$\symbfit{A}^{m+1}=A \circ A^m$是在$\{\alpha_1,\cdots,\alpha_n\}$下的矩阵为$\symbfit{A}\cdot\symbfit{A}^m=\symbfit{A}^{m+1}$

						因此结论成立,依照之前我们证明的线性映射性质,命题显然成立。
					\end{proof}
				\point{}
					\begin{proposition}
						$(\symbfit{A}+\symbfit{B})^T = \symbfit{A}^T+\symbfit{B}^T$
					\end{proposition}
					\begin{proof}
						$(\symbfit{A}+\symbfit{B})^T (j;i) = (\symbfit{A}+\symbfit{B})(i;j) = \symbfit{A}(i;j)+\symbfit{B}(i;j)=\symbfit{A}^T(j;i)+\symbfit{B}^T(j;i)$

						命题得证
					\end{proof}
				\point{}
					\begin{proposition}
						$(k\symbfit{A})^T = k\symbfit{A}^T$
					\end{proposition}
					\begin{proof}
						$(k\symbfit{A})^T(j;i)=(k\symbfit{A})(i;j)=k\symbfit{A}(i;j) = k\symbfit{A}^T(j;i)$

						命题得证
					\end{proof}
				\point{}
					\begin{proposition}
						设$V,W$是$F$上的两个线性空间,$\dim V = n < \aleph_0,\dim W = m < \aleph_0$

						分别取$V,W$的一个基$\{\alpha_1,\cdots,\alpha_n\},\{\beta_1,\cdots,\beta_m\}$,设其对偶基是$\{f_1,\cdots,f_n\},\{g_1,\cdots,g_m\}$

						$\forall A \in \hom(V,W)$,如果它在$\{\alpha_1,\cdots,\alpha_n\},\{\beta_1,\cdots,\beta_m\}$下的矩阵是$\symbfit{A}$

						那么$A^* \in \hom(W^*,V^*)$在$\{g_1,\cdots,g_m\},\{f_1,\cdots,f_n\}$下的矩阵是$\symbfit{A}^T$
					\end{proposition}
					\begin{proof}
						设$A^* \in \hom(W^*,V^*)$在$\{g_1,\cdots,g_m\},\{f_1,\cdots,f_n\}$下的矩阵是$\symbfit{A}'$

						$A^*(g_j)=\sum\limits_{i=1}^{n} \symbfit{A}'(i;j) f_i$

						$\Rightarrow A^*(g_j)(\alpha_k)=\sum\limits_{i=1}^{n} \symbfit{A}'(i;j) f_i(\alpha_k)$

						$\Rightarrow (g_jA)(\alpha_k)=\sum\limits_{i=1}^{n} \symbfit{A}'(i;j) \delta_{ik}$

						$\Rightarrow g_j(\sum\limits_{l=1}^{m} \symbfit{A}(l;k) \beta_l)=\symbfit{A}'(k;j)$

						$\Rightarrow \symbfit{A}(j;k)=\symbfit{A}'(k;j)$

						$\Rightarrow \symbfit{A}'=\symbfit{A}^T$

						命题得证
					\end{proof}
				\point{}
					我们先证明一个引理
					\begin{lemma}{}{}
						设$V,W,U$是$F$上的两个线性空间,$\dim V = n < \aleph_0,\dim W = m < \aleph_0,\dim U = k < \aleph_0$

						$\forall A \in \hom(V,W),B \in \hom(U,V)$

						那么$(AB)^*=B^*A^*$
					\end{lemma}
					\begin{proof}
						$\forall g \in W^*,\gamma \in U$

						$(AB)^*(g)(\gamma )=(gAB)(\gamma )=B^*(A^*(g))(\gamma )=(B^*A^*)(g)(\gamma )$

						$\Rightarrow (AB)^*=B^*A^*$
					\end{proof}
					\begin{proposition}
						$(\symbfit{A}\symbfit{B})^T = \symbfit{B}^T\symbfit{A}^T$
					\end{proposition}
					\begin{proof}
						任取$F$上的三个线性空间$V,W,U$,使得$\dim V = m < \aleph_0,\dim W = n < \aleph_0,\dim U = k < \aleph_0$

						分别取$V,W,U$的一个基$\{\alpha_1,\cdots,\alpha_m\},\{\beta_1,\cdots,\beta_n\},\{\gamma_1,\cdots,\gamma_k\}$

						并记其对偶基为$\{f_1,\cdots,f_m\},\{g_1,\cdots,g_n\},\{u_1,\cdots,u_k\}$

						显然存在$A \in \hom(V,W)$,使得它在$\{\alpha_1,\cdots,\alpha_m\},\{\beta_1,\cdots,\beta_n\}$下的矩阵为$\symbfit{A}$
						
						存在$B \in \hom(U,V)$,使得它在$\{\gamma_1,\cdots,\gamma_k\},\{\alpha_1,\cdots,\alpha_m\}$下的矩阵为$\symbfit{B}$

						注意到:$(\symbfit{A}\symbfit{B})^T$是$(AB)^*$在$\{g_1,\cdots,g_n\},\{u_1,\cdots,u_k\}$下的矩阵

						$\symbfit{B}^T\symbfit{A}^T$是$B^*A^*$在$\{g_1,\cdots,g_n\},\{u_1,\cdots,u_k\}$下的矩阵

						那么,因为$(AB)^*=B^*A^*$,所以一定有$(\symbfit{A}\symbfit{B})^T = \symbfit{B}^T\symbfit{A}^T$
					\end{proof}
				\point{}
					\begin{proposition}
						设$V,W$是$F$上的两个线性空间,$\dim V = m < \aleph_0,\dim W = n < \aleph_0$

						任取$V,W$的一个基$\{\alpha_1,\cdots,\alpha_m\},\{\beta_1,\cdots,\beta_n\}$
						
						$\forall A \in \hom(V,W)$,设$\symbfit{A}$是$A$在$\{\alpha_1,\cdots,\alpha_m\},\{\beta_1,\cdots,\beta_n\}$下的矩阵

						那么有:$\rank(A)=\rank(\symbfit{A})$
					\end{proposition}
					\begin{proof}
						设$\symbfit{A}=(\gamma_1,\cdots,\gamma _m)$

						我们只需证明:$\dim \im(A)=\dim \span(\gamma_1,\cdots,\gamma _m)$,因为$\rank(\gamma_1,\cdots,\gamma _m)=\dim \span(\gamma_1,\cdots,\gamma _m)$(参阅命题1.4.7)

						显然,$\im(A)=\span(A(\alpha_1),\cdots,A(\alpha_m))$

						考虑自然同构:$\phi : W \ni \sum\limits_{i=1}^{n} a_i \beta_i \mapsto \begin{pmatrix}
							a_1 \\
							\vdots \\
							a_n
						\end{pmatrix} \in F^n$

						显然$\phi $是一个同构映射。

						那么,只需注意到:$\forall k_1A(\alpha_1)+\cdots,k_mA(\alpha_m)\in \im(A)$
						
						$\phi (k_1A(\alpha_1)+\cdots,k_mA(\alpha_m))=k_1\gamma_1+\cdots,k_m\gamma_m \in \span(\alpha_1,\cdots,\alpha_m)$

						$\forall k_1\gamma_1+\cdots,k_m\gamma_m \in \span(\alpha_1,\cdots,\alpha_m)$

						$\phi^{-1}(k_1\gamma_1+\cdots,k_m\gamma_m)=k_1A(\alpha_1)+\cdots,k_mA(\alpha_m)\in \im(A)$

						因此,$\im(A) \cong \span(\gamma_1,\cdots,\gamma _m)$,于是命题得证
					\end{proof}
					至此,我们证明了,矩阵、向量组、线性映射的秩是统一的。
				\point{矩阵行秩等于列秩}
					\begin{proposition}
						设$\symbfit{A}=(\alpha_1,\cdots,\alpha_n)=\begin{pmatrix}
							\beta_1 \\
							\vdots \\
							\beta_m
						\end{pmatrix}$
						那么,$\rank(\alpha_1,\cdots,\alpha_n)=\rank(\beta_1,\cdots,\beta_m)$
					\end{proposition}
					\begin{proof}
						设$V,W$是域$F$上的两个线性空间,$\dim V = n < \aleph_0,\dim W = m < \aleph_0$

						分别任取$V,W$的一个基$\{\gamma_1,\cdots,\gamma_n\},\{\delta_1,\cdots,\delta_m\}$

						显然存在$A \in \hom(V,W)$,使得它在$\{\gamma_1,\cdots,\gamma_n\},\{\delta_1,\cdots,\delta_m\}$下的矩阵是$\symbfit{A}$

						于是有:$\rank(\alpha_1,\cdots,\alpha_n)=\rank(\symbfit{A})=\rank(A)=\rank(A^*)=\rank(\symbfit{A}^T)=\rank(\beta_1,\cdots,\beta_m)$

						于是命题得证。
					\end{proof}
				\point{}
					\begin{proposition}
						设$\symbfit{A} \in M_{m \times n}(F),\symbfit{B} \in M_{n \times k}(F)$

						$\rank(\symbfit{A}\symbfit{B})\leqslant \min\{\rank(\symbfit{A}),\rank(\symbfit{B})\}$
					\end{proposition}
					\begin{proof}
						先证明$\rank(\symbfit{A}\symbfit{B})\leqslant \rank(\symbfit{A})$

						设$V,W,U$是$F$上的三个线性空间,$\dim V = m < \aleph_0,\dim W = n < \aleph_0,\dim U = k < \aleph_0$

						任取$V,W,U$的一个基$\{\alpha_1,\cdots,\alpha_m\},\{\beta_1,\cdots,\beta_n\},\{\gamma_1,\cdots,\gamma_k\}$

						显然存在$A \in \hom(V,W),B \in \hom(U,V)$,使得$A$在$\{\alpha_1,\cdots,\alpha_m\},\{\beta_1,\cdots,\beta_n\}$下的矩阵是$\symbfit{A}$,$B$在$\{\gamma_1,\cdots,\gamma_k\},\{\alpha_1,\cdots,\alpha_m\}$下的矩阵为$\symbfit{B}$

						于是,$\rank(\symbfit{A}\symbfit{B})=\rank(AB)=\dim \im(AB)$

						注意到:$\im(AB)=\{A(B(\alpha ))|\alpha \in U\} \subseteq \{A(\beta )|\beta \in V\} = \im(A)$

						$\Rightarrow \dim \im (AB) \leqslant \dim \im (A) \Rightarrow \rank(AB) \leqslant \rank(A) \Rightarrow \rank(\symbfit{A}\symbfit{B})\leqslant \rank(\symbfit{A})$

						又注意到:$\rank(\symbfit{A}\symbfit{B})=\rank\left((\symbfit{A}\symbfit{B})^T\right)=\rank(\symbfit{B}^T\symbfit{A}^T)\leqslant \rank(\symbfit{B}^T)=\rank(\symbfit{B})$

						于是命题得证。
					\end{proof}
				\point{}
					\begin{proposition}
						$A$是幂等变换$\Leftrightarrow \symbfit{A}$是幂等矩阵
					\end{proposition}
					\begin{proof}
						
					\end{proof}
				\point{}
					\begin{proposition}
						设$V,W$是域$F$上的两个线性空间,$\dim V = n < \aleph_0,\dim W = m < \aleph_0$

						设$V,W$的一个基分别是$\{\alpha _1,\cdots,\alpha _n\},\{\beta _1,\cdots,\beta _m\}$

						$\forall A \in \hom(V,W)$,如果它在$\{\alpha _1,\cdots,\alpha _n\},\{\beta _1,\cdots,\beta _m\}$下的矩阵是$\symbfit{A}$

						$\gamma \in V$满足$\gamma =(\alpha_1,\cdots,\alpha_n)\mathbf{x}$,其中$\mathbf{x} \in M_{n\times 1}(F)$

						那么,$A(\gamma )$在$\{\beta_1,\cdots,\beta_m\}$下的矩阵是$\symbfit{A}\mathbf{x}$。
					\end{proposition}
					\begin{proof}
						
					\end{proof}
			\end{para}
	\section{特殊矩阵}
		本节中我们介绍一些后续需要用到的特殊矩阵
		\begin{para}{0}
			\point{}
				\begin{defn}{单位矩阵}{}
					$n$阶矩阵环$M_n(F)$的乘法单位元称为单位矩阵,记作$\symbfit{I_n}$,或简记为$\symbfit{I}$
				\end{defn}
				在前面一节,我们已经知道,$\symbfit{I}$的具体形式,其实就是满足$\symbfit{I}(i;j)=\delta_{ij}$的矩阵

				也就是$\symbfit{I_n}=\begin{pmatrix}
					1 & \cdots & 0 \\
					\vdots & \ddots & \vdots \\
					0 & \cdots & 1
				\end{pmatrix}$
		\end{para}
	\section{可逆矩阵}
		可逆矩阵其实就是矩阵环$M_n(F)$中的可逆元。本节中我们将研究可逆矩阵的一些基本性质,以及矩阵环中的特殊性质。最后,我们会看到它的一个简单应用:过渡矩阵和相似
		\subsection{可逆矩阵的定义}
			\begin{defn}{可逆矩阵}{}
				设$\symbfit{A} \in M_n(F)$,如果$\exists \symbfit{A}^{-1} \in M_n(F)$,使得:

				\begin{equation}
					\symbfit{A}\symbfit{A}^{-1}=\symbfit{A}^{-1}\symbfit{A}=\symbfit{I}
				\end{equation}
				那么我们称$\symbfit{A}$是一个可逆矩阵,并称$\symbfit{A}^{-1}$是$\symbfit{A}$的逆矩阵

				我们记全体可逆矩阵的集合为$GL_n(F)$,称为$F$上的$n$阶可逆矩阵群
			\end{defn}
		\subsection{可逆矩阵的性质}
			\begin{para}{0}
				\point{}
					\begin{proposition}
						$\symbfit{A} \in GL_n(F) \Rightarrow \left(\symbfit{A}^{-1}\right)^{-1}=\symbfit{A}$
					\end{proposition}
					\begin{proof}
						依照定义显然。
					\end{proof}
				\point{}
					\begin{proposition}
						设$V$是一个$F$上的线性空间,且它们一个基是$\{\alpha _1,\cdots,\alpha _n\}$

						则$\forall A \in \hom(V,V)$,若$A$在$\{\alpha_1,\cdots,\alpha_n\}$下的矩阵是$\symbfit{A}$,那么:
						
						$A$可逆$\Leftrightarrow \symbfit{A} \in GL_n(F)$

						且此时$A^{-1}$在$\{\alpha_1,\cdots,\alpha_n\}$下的矩阵是$\symbfit{A}^{-1}$
					\end{proposition}
			\end{para}
		\subsection{过渡矩阵}
		\subsection{矩阵的相似}
	\section{迹和行列式}
		\subsection{张量积}
			在研究迹和行列式前,我们先研究张量积。张量积一般应用于多重线性映射的研究中,后面我们会看到:行列式是一种特殊的多重线性映射。

			张量积的本意是将两个线性空间组合在一起形成新的空间,并且希望可以将原来的两个空间和新的空间对应。

			在双线性映射,即对每一分量都线性的映射的研究中,我们将两个空间组合在一起,视为同一空间研究

			我们希望:新的空间和原来的两个空间应该别无二致。因此,如果确立了两个空间$V,W$到张量积空间$U$的映射,我们希望每一个$V,W$到$Z$的双线性映射,都只能确定唯一的$U$到$Z$的映射。

			以上想法给出了以下定义:
			\begin{defn}{双线性映射}{}
				设$V,W,Z$是三个线性空间,

				映射$f:V \times W \to Z$如果满足:

				\begin{enumerate}
					\item $\forall \alpha ,\beta \in V,\gamma \in W,f(\alpha +\beta ,\gamma )=f(\alpha ,\gamma )+f(\beta +\gamma )$
					\item $\forall \alpha \in V,\beta ,\gamma \in W,f(\alpha ,\beta +\gamma )=f(\alpha ,\beta )+f(\alpha ,\gamma )$
					\item $\forall k \in F,\alpha \in V,\beta \in W,f(k\cdot\alpha ,\beta )=k\cdot f(\alpha ,\beta ),f(\alpha ,k\cdot \beta )=k\cdot f(\alpha ,\beta )$ 
				\end{enumerate}
				那么我们称$f$是一个双线性映射
			\end{defn}
			\begin{defn}{线性空间的张量积}{}
				设$V,W,Z$是三个$F$上的线性空间,如果线性空间$V \otimes W$满足:

				存在映射$f:V \times W \to V \otimes W$,使得:

				任意的$f:V \times W \to Z$的双线性映射$g$,存在唯一的线性映射$h:V \otimes W \to Z$,使得$g = h \circ f$

				那么,我们称$V \otimes W$是$V$和$W$的张量积

				此时,我们也记$\alpha \otimes \beta := f(\alpha ,\beta )$
			\end{defn}
			在继续研究之前,我们需要先证明张量积的确是存在的:
			\begin{proposition}
				任意两个$F$上的线性空间$V,W$的张量积$V \otimes W$是存在的,并且在同构意义下唯一
			\end{proposition}
			我们也容易证明,张量积在同构意义下满足结合律
			\begin{proposition}
				$(V \otimes W) \otimes Z \cong V \otimes (W \otimes Z)$
			\end{proposition}
		\subsection{张量和外积}
			我们首先介绍张量的概念
			\begin{defn}{张量}{}
				设$V$是$F$上的一个线性空间,我们定义:
				\begin{equation}
					T_{s}^{r}:= T^{r} V \otimes T^{s} V^*
				\end{equation}
				其中
				\begin{equation}
					T^{p} W:= \begin{cases}
						F,p=0 \\
						W \otimes T^{p-1} W,p \geqslant 1
					\end{cases}
				\end{equation}
				我们称$T_{s}^{r}$中的元素为一个$(r,s)$型张量
			\end{defn}
			\begin{defn}{张量代数}{}
				我们定义:
				\begin{equation}
					TV = \bigoplus_{r=0}^{\infty} T^{r} V
				\end{equation}
				称为$V$上的张量代数

				其中,$TV$的加法和纯量乘法为映射的加法和纯量乘法

				同时,我们定义$TV$中的乘法为:$\forall f,g \in TV$

				$(f\cdot g)(k)=\sum\limits_{i+j=k} f(i) \otimes g(j)$
			\end{defn}
			\begin{defn}{外代数}{}
				记$S=\{\alpha \otimes \alpha |\alpha \in V\},I=(S)$

				我们定义:
				\begin{equation}
					\bigwedge  V = TV / I
				\end{equation}
				称为$V$上的外代数

				特别地,我们记:$I_p = I \cap T^{p} V,\bigwedge\limits^{p} V = T^{p}V/I_p$
			\end{defn}

		\subsection{行列式的定义和性质}
			\begin{defn}{行列式}{}
				设$F$是一个域,$V$是$F$上的一个线性空间,并且$dim_F V = n$

				映射$\det: V^n \rightarrow F$如果满足:

				\ding{172} $\det(\alpha_1,\cdots,\alpha_i + \beta_i,\cdots,\alpha_n)=\det(\alpha_1,\cdots,\alpha_i,\cdots,\alpha_n)+\det(\alpha_1,\cdots,\beta_i,\cdots,\alpha_n)$

				\ding{173} $\forall k\in F,\det(\alpha_1,\cdots,k\cdot \alpha_i,\cdots,\alpha_n)=k \cdot \det(\alpha_1,\cdots,\alpha_i,\cdots,\alpha_n)$

				\ding{173} $\det(\alpha_1,\cdots,\alpha_i,\cdots,\alpha_j,\cdots,\alpha_n)=-\det(\alpha_1,\cdots,\alpha_j,\cdots,\alpha_i,\cdots,\alpha_n)$

				\ding{174} 存在$V$的一组基$\gamma_i,\cdots,\gamma_n,\det(\gamma_1,\cdots,\gamma_n)=1$

				那么我们称$\det$是一个$V$上的$n$阶行列式
			\end{defn}
			由行列式的定义,我们可以推导出行列式的基本性质
			\begin{proposition}
				向量组$\alpha_1,\cdots,\alpha_i,\cdots,\alpha_j,\cdots,\alpha_n$如果有$\alpha_i=\alpha_j$

				那么$\det (\alpha_1,\cdots,\alpha_i,\cdots,\alpha_j,\cdots,\alpha_n)=0$
			\end{proposition}
			\begin{proof}
				$\det (\alpha_1,\cdots,\alpha_i,\cdots,\alpha_j,\cdots,\alpha_n)=-\det (\alpha_1,\cdots,\alpha_j,\cdots,\alpha_i,\cdots,\alpha_n)$

				但因为$\alpha_i=\alpha_j$,所以必有$\det (\alpha_1,\cdots,\alpha_i,\cdots,\alpha_j,\cdots,\alpha_n)=0$
			\end{proof}
			进一步我们可以推出,如果两个变量成系数关系,那么行列式也为零
			\begin{corollary}{存在成比例变量的行列式为零}{}
				向量组$\alpha_1,\cdots,\alpha_i,\cdots,\alpha_j,\cdots,\alpha_n$如果有$\alpha_i=k\alpha_j,k\in F$

				那么$\det (\alpha_1,\cdots,\alpha_i,\cdots,\alpha_j,\cdots,\alpha_n)=0$
			\end{corollary}
			\begin{proof}
				$\det (\alpha_1,\cdots,\alpha_i,\cdots,\alpha_j,\cdots,\alpha_n)=k\cdot \det (\alpha_1,\cdots,\alpha_j,\cdots,\alpha_j,\cdots,\alpha_n)=0$
			\end{proof}
		\subsection{行列式在基上的展开}
			\begin{them}{行列式的展开}{}
				设$F$是一个域,$V$是$F$上的一个线性空间,并且$dim_F V = n$,
				
				$V$上的$n$阶行列式$\det$满足$\det(\gamma_1,\cdots,\gamma_n)=1$,其中$\{\gamma_1,\cdots,\gamma_n\}$是$V$的一组基

				那么,有:
				\begin{equation}
					\det(\alpha_1,\cdots,\alpha_n)=\sum_{\sigma \in S_n} \text{sgn}(\sigma) \prod_{i=1}^n a_{i,\sigma(i)}
				\end{equation}
				其中$\alpha_i = \sum\limits_{j=1}^{n} a_{i,j} \gamma_j$
			\end{them}
			\begin{proof}
				$\det (\alpha_1,\cdots,\alpha_n)=\det\left(\sum\limits_{i_1=1}^{n} a_{1,i_1} \gamma_{i_1},\cdots,\sum\limits_{i_n=1}^{n} a_{n,i_n} \gamma_{i_n}\right)$

				$=\sum\limits_{i_1=1}^{n}\cdots \sum\limits_{i_n=1}^{n} \left(\prod\limits_{k=1}^{n}a_{k,i_k} \det(\alpha_{i_1},\cdots,\alpha_{i_n})\right)$

				$=\sum\limits_{\sigma \in S_n} \left(\prod\limits_{k=1}^{n}a_{k,\sigma(k)} \text{sgn}(\sigma )\right)$
			\end{proof}
			事实上,我们也可以改变第一个求和指标,使之称为一个固定但是可以随意选取的置换
			\begin{corollary}{}{}
				设$F$是一个域,$V$是$F$上的一个线性空间,并且$dim_F V = n$,
				
				$V$上的$n$阶行列式$\det$满足$\det(\gamma_1,\cdots,\gamma_n)=1$,其中$\{\gamma_1,\cdots,\gamma_n\}$是$V$的一组基

				那么,有:
				\begin{equation}
					\det(\alpha_1,\cdots,\alpha_n)=\text{sgn}(\rho )\sum_{\sigma \in S_n} \text{sgn}(\sigma) \prod_{i=1}^n a_{\rho(i),\sigma(i)}
				\end{equation}
				其中$\alpha_i = \sum\limits_{j=1}^{n} a_{i,j} \gamma_j$,$\rho$是一个置换
			\end{corollary}
			\begin{proof}
				$\det(\alpha_1,\cdots,\alpha_n)=\sum_{\tau  \in S_n} \text{sgn}(\tau ) \prod_{i=1}^n a_{i,\tau (i)}$

				对指标作置换$\rho $,累乘的结果不会变化,所以有:

				$\det(\alpha_1,\cdots,\alpha_n)=\sum_{\tau \in S_n} \text{sgn}(\tau ) \prod_{i=1}^n a_{\rho(i),(\rho \circ \tau )(i)}$

				记$\sigma =\rho \circ \tau $,那么$\det(\alpha_1,\cdots,\alpha_n)=\sum_{\rho^{-1} \circ \sigma  \in S_n} \text{sgn}(\rho^{-1} \circ \sigma) \prod_{i=1}^n a_{\rho(i),\sigma (i)}$

				但是,$\rho^{-1} \circ \sigma \in S_n$其实就是$\sigma \in S_n$,并且我们知道$\text{sgn}(\rho^{-1} \circ \sigma)=\text{sgn}(\rho )\text{sgn}(\sigma )$

				所以$\det(\alpha_1,\cdots,\alpha_n)=\text{sgn}(\rho )\sum_{\sigma \in S_n} \text{sgn}(\sigma) \prod_{i=1}^n a_{\rho(i),\sigma(i)}$
			\end{proof}
		\subsection{矩阵的行列式}
			我们之前已经指出,$M_n (F) \cong F^{n^2} \cong (F^n)^n$,因此,我们可以对矩阵定义行列式:
			\begin{defn}{矩阵的行列式}{}
				设矩阵$A = (\alpha_1,\cdots,\alpha_n)\in M_n (F)$,我们定义:

				$|A|=\det (A) := \det(\alpha_1,\cdots,\alpha_n)$

				并且有$\det(e_1,\cdots,e_n)=1$,其中$e_i$是标准基向量$(0,\cdots,1,\cdots,0)$,$1$在第$i$个位置上。
			\end{defn}
			矩阵的行列式也可以类似地在标准基上展开
			\begin{them}{矩阵的行列式的展开}{}
				设$F$是一个域,矩阵$A=(a_{ij}) \in M_{n\times n}(F)$
				那么,有:
				\begin{equation}
					|A|=\sum_{\sigma \in S_n} \text{sgn}(\sigma) \prod_{i=1}^n a_{i,\sigma(i)}
				\end{equation}
			\end{them}
		\subsection{矩阵的行列式的余子式展开}
		\subsection{矩阵乘积的行列式}
\ifx\allfiles\undefined
\end{document}
\fi