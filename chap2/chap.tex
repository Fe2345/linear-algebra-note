\ifx\allfiles\undefined
\documentclass[12pt, a4paper, oneside, UTF8]{ctexbook}
\def\path{../config}
\input{../config/_config}
\begin{document}
% \input{../config/cover}
\else
\fi
%标题
\chapter{线性映射}
	\section{线性映射的定义和性质}
	\section{矩阵}
		\subsection{矩阵的定义}
			\begin{defn}{矩阵}{}
				形如以下的矩形阵列称为一个域$F$上的矩阵
				$\begin{pmatrix}
					a_{11} & \cdots & a_{1n} \\
					\vdots & \ddots & \vdots \\
					a_{m1} & \cdots & a_{mn}
				\end{pmatrix},a_{ij} \in F$

				简记为$(a_{ij})_{m \times n}$或$(a_{ij})$。$m$称为矩阵的行数,$n$称为矩阵的列数。

				特别地,如果$m=n$,我们称它是一个$m$阶方阵。

				$F$上的全体$m \times n$矩阵的集合记作$M_{m \times n} (F)$,特别地如果$m=n$,记作$M_n (F)$。

				我们也将矩阵$A$在$m$行$n$列处的元素记作$A(i;j)$
			\end{defn}
		\subsection{矩阵的运算}
		\begin{para}{0}
			\point{相等}
			    \begin{defn}{矩阵的相等}{}
				    设$A\in M_{m\times n}(F),B\in M_{m \times n}(F)$,如果$\forall i,j,A(i;j)=B(i;j)$,则称$A=B$。
				\end{defn}
			\point{转置}
				\begin{defn}{矩阵的转置}{}
				    设$A\in M_{m \times n}(F)$,
					
					我们定义矩阵$A^T \in M_{n \times m}(F)$为满足$A^T(i;j)=A(i;j)$的矩阵,称为$A$的转置。
				\end{defn}
			\point{加法}
				\begin{defn}{矩阵的加法}{}
				    设$A\in M_{m \times n}(F),B\in M_{m \times n}(F)$
					
					我们定义:$(A+B)(i;j)=A(i;j)+B(i;j)$。
				\end{defn}
			\point{纯量乘法}
				\begin{defn}{矩阵的纯量乘法}{}
				    设$A\in M_{m \times n}(F),k \in F$,
					
					我们定义矩阵$k\cdot A \in M_{m\times n}(F)$为满足$(k\cdot  A)(i;j)=k\cdot A(i;j)$的矩阵。
				\end{defn}
			\point{乘法}
				\begin{defn}{矩阵的乘法}{}
				    设$A\in M_{m\times n}(F),B\in M_{n \times p}(F)$,
					
					我们定义矩阵$A\cdot B \in M_{m\times p}(F)$为满足$(A\cdot B)(i;j)=\sum_{k=1}^n A(i;k)B(k;j)$的矩阵
				\end{defn}
			\point{幂}
				\begin{defn}{方阵的幂}{}
					设$A \in M_n (F)$是一个方阵,
					我们定义:$$A^k=A\cdot A^{k-1}$$
				\end{defn}
		\end{para}
		\subsection{矩阵的性质}

	\section{行列式}
		\subsection{行列式的定义和性质}
			\begin{defn}{行列式}{}
				设$F$是一个域,$V$是$F$上的一个线性空间,并且$dim_F V = n$

				映射$\det: V^n \rightarrow F$如果满足:

				\ding{172} $\det(\alpha_1,\cdots,\alpha_i + \beta_i,\cdots,\alpha_n)=\det(\alpha_1,\cdots,\alpha_i,\cdots,\alpha_n)+\det(\alpha_1,\cdots,\beta_i,\cdots,\alpha_n)$

				\ding{173} $\forall k\in F,\det(\alpha_1,\cdots,k\cdot \alpha_i,\cdots,\alpha_n)=k \cdot \det(\alpha_1,\cdots,\alpha_i,\cdots,\alpha_n)$

				\ding{173} $\det(\alpha_1,\cdots,\alpha_i,\cdots,\alpha_j,\cdots,\alpha_n)=-\det(\alpha_1,\cdots,\alpha_j,\cdots,\alpha_i,\cdots,\alpha_n)$

				\ding{174} 存在$V$的一组基$\gamma_i,\cdots,\gamma_n,\det(\gamma_1,\cdots,\gamma_n)=1$

				那么我们称$\det$是一个$V$上的$n$阶行列式
			\end{defn}
			由行列式的定义,我们可以推导出行列式的基本性质
			\begin{proposition}
				向量组$\alpha_1,\cdots,\alpha_i,\cdots,\alpha_j,\cdots,\alpha_n$如果有$\alpha_i=\alpha_j$

				那么$\det (\alpha_1,\cdots,\alpha_i,\cdots,\alpha_j,\cdots,\alpha_n)=0$
			\end{proposition}
			\begin{proof}
				$\det (\alpha_1,\cdots,\alpha_i,\cdots,\alpha_j,\cdots,\alpha_n)=-\det (\alpha_1,\cdots,\alpha_j,\cdots,\alpha_i,\cdots,\alpha_n)$

				但因为$\alpha_i=\alpha_j$,所以必有$\det (\alpha_1,\cdots,\alpha_i,\cdots,\alpha_j,\cdots,\alpha_n)=0$
			\end{proof}
			进一步我们可以推出,如果两个变量成系数关系,那么行列式也为零
			\begin{corollary}{存在成比例变量的行列式为零}{}
				向量组$\alpha_1,\cdots,\alpha_i,\cdots,\alpha_j,\cdots,\alpha_n$如果有$\alpha_i=k\alpha_j,k\in F$

				那么$\det (\alpha_1,\cdots,\alpha_i,\cdots,\alpha_j,\cdots,\alpha_n)=0$
			\end{corollary}
			\begin{proof}
				$\det (\alpha_1,\cdots,\alpha_i,\cdots,\alpha_j,\cdots,\alpha_n)=k\cdot \det (\alpha_1,\cdots,\alpha_j,\cdots,\alpha_j,\cdots,\alpha_n)=0$
			\end{proof}
		\subsection{行列式在基上的展开}
			\begin{them}{行列式的展开}{}
				设$F$是一个域,$V$是$F$上的一个线性空间,并且$dim_F V = n$,
				
				$V$上的$n$阶行列式$\det$满足$\det(\gamma_1,\cdots,\gamma_n)=1$,其中$\{\gamma_1,\cdots,\gamma_n\}$是$V$的一组基

				那么,有:
				\begin{equation}
					\det(\alpha_1,\cdots,\alpha_n)=\sum_{\sigma \in S_n} \text{sgn}(\sigma) \prod_{i=1}^n a_{i,\sigma(i)}
				\end{equation}
				其中$\alpha_i = \sum\limits_{j=1}^{n} a_{i,j} \gamma_j$
			\end{them}
			\begin{proof}
				$\det (\alpha_1,\cdots,\alpha_n)=\det\left(\sum\limits_{i_1=1}^{n} a_{1,i_1} \gamma_{i_1},\cdots,\sum\limits_{i_n=1}^{n} a_{n,i_n} \gamma_{i_n}\right)$

				$=\sum\limits_{i_1=1}^{n}\cdots \sum\limits_{i_n=1}^{n} \left(\prod\limits_{k=1}^{n}a_{k,i_k} \det(\alpha_{i_1},\cdots,\alpha_{i_n})\right)$

				$=\sum\limits_{\sigma \in S_n} \left(\prod\limits_{k=1}^{n}a_{k,\sigma(k)} \text{sgn}(\sigma )\right)$
			\end{proof}
			事实上,我们也可以改变第一个求和指标,使之称为一个固定但是可以随意选取的置换
			\begin{corollary}{}{}
				设$F$是一个域,$V$是$F$上的一个线性空间,并且$dim_F V = n$,
				
				$V$上的$n$阶行列式$\det$满足$\det(\gamma_1,\cdots,\gamma_n)=1$,其中$\{\gamma_1,\cdots,\gamma_n\}$是$V$的一组基

				那么,有:
				\begin{equation}
					\det(\alpha_1,\cdots,\alpha_n)=\text{sgn}(\rho )\sum_{\sigma \in S_n} \text{sgn}(\sigma) \prod_{i=1}^n a_{\rho(i),\sigma(i)}
				\end{equation}
				其中$\alpha_i = \sum\limits_{j=1}^{n} a_{i,j} \gamma_j$,$\rho$是一个置换
			\end{corollary}
			\begin{proof}
				$\det(\alpha_1,\cdots,\alpha_n)=\sum_{\tau  \in S_n} \text{sgn}(\tau ) \prod_{i=1}^n a_{i,\tau (i)}$

				对指标作置换$\rho $,累乘的结果不会变化,所以有:

				$\det(\alpha_1,\cdots,\alpha_n)=\sum_{\tau \in S_n} \text{sgn}(\tau ) \prod_{i=1}^n a_{\rho(i),(\rho \circ \tau )(i)}$

				记$\sigma =\rho \circ \tau $,那么$\det(\alpha_1,\cdots,\alpha_n)=\sum_{\rho^{-1} \circ \sigma  \in S_n} \text{sgn}(\rho^{-1} \circ \sigma) \prod_{i=1}^n a_{\rho(i),\sigma (i)}$

				但是,$\rho^{-1} \circ \sigma \in S_n$其实就是$\sigma \in S_n$,并且我们知道$\text{sgn}(\rho^{-1} \circ \sigma)=\text{sgn}(\rho )\text{sgn}(\sigma )$

				所以$\det(\alpha_1,\cdots,\alpha_n)=\text{sgn}(\rho )\sum_{\sigma \in S_n} \text{sgn}(\sigma) \prod_{i=1}^n a_{\rho(i),\sigma(i)}$
			\end{proof}
		\subsection{矩阵的行列式}
			我们之前已经指出,$M_n (F) \cong F^{n^2} \cong (F^n)^n$,因此,我们可以对矩阵定义行列式:
			\begin{defn}{矩阵的行列式}{}
				设矩阵$A = (\alpha_1,\cdots,\alpha_n)\in M_n (F)$,我们定义:

				$|A|=\det (A) := \det(\alpha_1,\cdots,\alpha_n)$

				并且有$\det(e_1,\cdots,e_n)=1$,其中$e_i$是标准基向量$(0,\cdots,1,\cdots,0)$,$1$在第$i$个位置上。
			\end{defn}
			矩阵的行列式也可以类似地在标准基上展开
			\begin{them}{矩阵的行列式的展开}{}
				设$F$是一个域,矩阵$A=(a_{ij}) \in M_{n\times n}(F)$
				那么,有:
				\begin{equation}
					|A|=\sum_{\sigma \in S_n} \text{sgn}(\sigma) \prod_{i=1}^n a_{i,\sigma(i)}
				\end{equation}
			\end{them}
		\subsection{矩阵的行列式的余子式展开}
		\subsection{矩阵乘积的行列式}
	\section{矩阵的初等变换、线性方程组的解}
	\section{可逆矩阵}
	\section{矩阵的分块}
\ifx\allfiles\undefined
\end{document}
\fi