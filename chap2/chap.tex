\ifx\allfiles\undefined
\documentclass[12pt, a4paper, oneside, UTF8]{ctexbook}
\def\path{../config}
\usepackage{amsmath}
\usepackage{amsthm}
\usepackage{amssymb}
\usepackage{graphicx}
\usepackage{mathrsfs}
\usepackage{enumitem}
\usepackage{geometry}
\usepackage[colorlinks, linkcolor=black]{hyperref}
\usepackage{stackengine}
\usepackage{yhmath}
\usepackage{extarrows}
\usepackage{unicode-math}
\usepackage{tikz}
\usepackage{tikz-cd}
\usepackage{pifont}
\usepackage{pgfplots}
\usepackage{tikz-3dplot}

\usepackage{fancyhdr}
\usepackage[dvipsnames, svgnames]{xcolor}
\usepackage{listings}

\definecolor{mygreen}{rgb}{0,0.6,0}
\definecolor{mygray}{rgb}{0.5,0.5,0.5}
\definecolor{mymauve}{rgb}{0.58,0,0.82}

\graphicspath{ {figure/},{../figure/}, {config/}, {../config/} }

\linespread{1.6}

\geometry{
    top=25.4mm, 
    bottom=25.4mm, 
    left=20mm, 
    right=20mm, 
    headheight=2.17cm, 
    headsep=4mm, 
    footskip=12mm
}

\setenumerate[1]{itemsep=5pt,partopsep=0pt,parsep=\parskip,topsep=5pt}
\setitemize[1]{itemsep=5pt,partopsep=0pt,parsep=\parskip,topsep=5pt}
\setdescription{itemsep=5pt,partopsep=0pt,parsep=\parskip,topsep=5pt}

\lstset{
    language=Mathematica,
    basicstyle=\tt,
    breaklines=true,
    keywordstyle=\bfseries\color{NavyBlue}, 
    emphstyle=\bfseries\color{Rhodamine},
    commentstyle=\itshape\color{black!50!white}, 
    stringstyle=\bfseries\color{PineGreen!90!black},
    columns=flexible,
    numbers=left,
    numberstyle=\footnotesize,
    frame=tb,
    breakatwhitespace=false,
} 
\usepackage[strict]{changepage} 
\usepackage{framed}
\usepackage{tcolorbox}
\tcbuselibrary{most}

\definecolor{greenshade}{rgb}{0.90,1,0.92}
\definecolor{redshade}{rgb}{1.00,0.88,0.88}
\definecolor{brownshade}{rgb}{0.99,0.95,0.9}
\definecolor{lilacshade}{rgb}{0.95,0.93,0.98}
\definecolor{orangeshade}{rgb}{1.00,0.88,0.82}
\definecolor{lightblueshade}{rgb}{0.8,0.92,1}
\definecolor{purple}{rgb}{0.81,0.85,1}

% #### 将 config.tex 中的定理环境的对应部分替换为如下内容
% 定义单独编号,其他四个共用一个编号计数 这里只列举了五种,其他可类似定义(未定义的使用原来的也可)
\newtcbtheorem[number within=section]{defn}%
{定义}{colback=OliveGreen!10,colframe=Green!70,fonttitle=\bfseries}{def}

\newtcbtheorem[number within=section]{lemma}%
{引理}{colback=Salmon!20,colframe=Salmon!90!Black,fonttitle=\bfseries}{lem}

% 使用另一个计数器 use counter from=lemma
\newtcbtheorem[use counter from=lemma, number within=section]{them}%
{定理}{colback=SeaGreen!10!CornflowerBlue!10,colframe=RoyalPurple!55!Aquamarine!100!,fonttitle=\bfseries}{them}

\newtcbtheorem[use counter from=lemma, number within=section]{criterion}%
{准则}{colback=green!5,colframe=green!35!black,fonttitle=\bfseries}{cri}

\newtcbtheorem[use counter from=lemma, number within=section]{corollary}%
{推论}{colback=Emerald!10,colframe=cyan!40!black,fonttitle=\bfseries}{cor}
% colback=red!5,colframe=red!75!black

% 这个颜色我不喜欢
%\newtcbtheorem[number within=section]{proposition}%
%{命题}{colback=red!5,colframe=red!75!black,fonttitle=\bfseries}{cor}

% .... 命题 例 注 证明 解 使用之前的就可以(全文都是这种框框就很丑了),也可以按照上述定义 ...
\renewenvironment{proof}{\par\textbf{证明:}\;}{\qed\par}
\newenvironment{solution}{\par{\textbf{解:}}\;}{\qed\par}
\newtheorem{proposition}{\indent 命题}[section]
\newtheorem{example}{\indent \color{SeaGreen}{例}}[section] % 绿色文字的 例 ,不需要就去除\color{SeaGreen}{}
\newtheorem*{rmk}{\indent 注}
\usepackage{amssymb}
\setmathfont{LatinModernMath-Regular}
\setmathfont[range=\mathbb]{TeXGyrePagellaMath-Regular}
\def\d{\mathrm{d}}
\def\R{\mathbb{R}}
\def\C{\mathbb{C}}
\def\Q{\mathbb{Q}}
\def\N{\mathbb{N}}
\def\Z{\mathbb{Z}}
\newcommand{\bs}[1]{\boldsymbol{#1}}
\newcommand{\ora}[1]{\overrightarrow{#1}}
\newcommand{\myspace}[1]{\par\vspace{#1\baselineskip}}
\newcommand{\xrowht}[2][0]{\addstackgap[.5\dimexpr#2\relax]{\vphantom{#1}}}
\newenvironment{ca}[1][1]{\linespread{#1} \selectfont \begin{cases}}{\end{cases}}
\newenvironment{vx}[1][1]{\linespread{#1} \selectfont \begin{vmatrix}}{\end{vmatrix}}
\newcommand{\tabincell}[2]{\begin{tabular}{@{}#1@{}}#2\end{tabular}}
\newcommand{\pll}{\kern 0.56em/\kern -0.8em /\kern 0.56em}
\newcommand{\dive}[1][F]{\mathrm{div}\;\bs{#1}}
\newcommand{\rotn}[1][A]{\mathrm{rot}\;\bs{#1}}
\usepackage{xeCJK}
\setCJKmainfont{SimSun}[BoldFont={SimHei}, ItalicFont={KaiTi}] % 设置中文支持

\newcommand{\point}[1]{\item {#1}}
\newenvironment{para}[1]{%
\ifcase#1\relax
\begin{enumerate}[label=\arabic*.] % 1.2.3.
\or
\begin{enumerate}[label=\textcircled{\arabic*}] % ①②③
\or
\begin{enumerate}[label=(\roman*)] % (i)(ii)(iii)
\else
\begin{enumerate}[label=\arabic*.] % 默认格式
\fi
}{
\end{enumerate}
}

\def\myIndex{0}
% \input{\path/cover_package_\myIndex.tex}

\def\myTitle{高等代数笔记}
\def\myAuthor{Zhang Liang}
\def\myDateCover{\today}
\def\myDateForeword{\today}
\def\myForeword{前言标题}
\def\myForewordText{
    前言内容
}
\def\mySubheading{副标题}


\begin{document}
% \input{\path/cover_text_\myIndex.tex}

\newpage
\thispagestyle{empty}
\begin{center}
    \Huge\textbf{\myForeword}
\end{center}
\myForewordText
\begin{flushright}
    \begin{tabular}{c}
        \myDateForeword
    \end{tabular}
\end{flushright}

\newpage
\pagestyle{plain}
\setcounter{page}{1}
\pagenumbering{Roman}
\tableofcontents

\newpage
\pagenumbering{arabic}
\setcounter{chapter}{0}
\setcounter{page}{0}

\pagestyle{fancy}
\fancyfoot[C]{\thepage}
\renewcommand{\headrulewidth}{0.4pt}
\renewcommand{\footrulewidth}{0pt}








\else
\fi
%标题
\chapter{线性映射}
	\section{线性映射的定义和性质}
	\section{矩阵}
		\subsection{矩阵的定义}
			\begin{defn}{矩阵}{}
				形如以下的矩形阵列称为一个域$F$上的矩阵
				$\begin{pmatrix}
					a_{11} & \cdots & a_{1n} \\
					\vdots & \ddots & \vdots \\
					a_{m1} & \cdots & a_{mn}
				\end{pmatrix},a_{ij} \in F$

				简记为$(a_{ij})_{m \times n}$或$(a_{ij})$。$m$称为矩阵的行数,$n$称为矩阵的列数。

				特别地,如果$m=n$,我们称它是一个$m$阶方阵。

				$F$上的全体$m \times n$矩阵的集合记作$M_{m \times n} (F)$,特别地如果$m=n$,记作$M_n (F)$。

				我们也将矩阵$A$在$m$行$n$列处的元素记作$A(i;j)$
			\end{defn}
		\subsection{矩阵的运算}
		\begin{para}{0}
			\point{相等}
			    \begin{defn}{矩阵的相等}{}
				    设$A\in M_{m\times n}(F),B\in M_{m \times n}(F)$,如果$\forall i,j,A(i;j)=B(i;j)$,则称$A=B$。
				\end{defn}
			\point{转置}
				\begin{defn}{矩阵的转置}{}
				    设$A\in M_{m \times n}(F)$,
					
					我们定义矩阵$A^T \in M_{n \times m}(F)$为满足$A^T(i;j)=A(i;j)$的矩阵,称为$A$的转置。
				\end{defn}
			\point{加法}
				\begin{defn}{矩阵的加法}{}
				    设$A\in M_{m \times n}(F),B\in M_{m \times n}(F)$
					
					我们定义:$(A+B)(i;j)=A(i;j)+B(i;j)$。
				\end{defn}
			\point{纯量乘法}
				\begin{defn}{矩阵的纯量乘法}{}
				    设$A\in M_{m \times n}(F),k \in F$,
					
					我们定义矩阵$k\cdot A \in M_{m\times n}(F)$为满足$(k\cdot  A)(i;j)=k\cdot A(i;j)$的矩阵。
				\end{defn}
			\point{乘法}
				\begin{defn}{矩阵的乘法}{}
				    设$A\in M_{m\times n}(F),B\in M_{n \times p}(F)$,
					
					我们定义矩阵$A\cdot B \in M_{m\times p}(F)$为满足$(A\cdot B)(i;j)=\sum_{k=1}^n A(i;k)B(k;j)$的矩阵
				\end{defn}
			\point{幂}
				\begin{defn}{方阵的幂}{}
					设$A \in M_n (F)$是一个方阵,
					我们定义:$$A^k=A\cdot A^{k-1}$$
				\end{defn}
		\end{para}
		\subsection{矩阵的性质}

	\section{行列式}
		\subsection{行列式的定义和性质}
			\begin{defn}{行列式}{}
				设$F$是一个域,$V$是$F$上的一个线性空间,并且$dim_F V = n$

				映射$\det: V^n \rightarrow F$如果满足:

				\ding{172} $\det(\alpha_1,\cdots,\alpha_i + \beta_i,\cdots,\alpha_n)=\det(\alpha_1,\cdots,\alpha_i,\cdots,\alpha_n)+\det(\alpha_1,\cdots,\beta_i,\cdots,\alpha_n)$

				\ding{173} $\forall k\in F,\det(\alpha_1,\cdots,k\cdot \alpha_i,\cdots,\alpha_n)=k \cdot \det(\alpha_1,\cdots,\alpha_i,\cdots,\alpha_n)$

				\ding{173} $\det(\alpha_1,\cdots,\alpha_i,\cdots,\alpha_j,\cdots,\alpha_n)=-\det(\alpha_1,\cdots,\alpha_j,\cdots,\alpha_i,\cdots,\alpha_n)$

				\ding{174} 存在$V$的一组基$\gamma_i,\cdots,\gamma_n,\det(\gamma_1,\cdots,\gamma_n)=1$

				那么我们称$\det$是一个$V$上的$n$阶行列式
			\end{defn}
			由行列式的定义,我们可以推导出行列式的基本性质
			\begin{proposition}
				向量组$\alpha_1,\cdots,\alpha_i,\cdots,\alpha_j,\cdots,\alpha_n$如果有$\alpha_i=\alpha_j$

				那么$\det (\alpha_1,\cdots,\alpha_i,\cdots,\alpha_j,\cdots,\alpha_n)=0$
			\end{proposition}
			\begin{proof}
				$\det (\alpha_1,\cdots,\alpha_i,\cdots,\alpha_j,\cdots,\alpha_n)=-\det (\alpha_1,\cdots,\alpha_j,\cdots,\alpha_i,\cdots,\alpha_n)$

				但因为$\alpha_i=\alpha_j$,所以必有$\det (\alpha_1,\cdots,\alpha_i,\cdots,\alpha_j,\cdots,\alpha_n)=0$
			\end{proof}
			进一步我们可以推出,如果两个变量成系数关系,那么行列式也为零
			\begin{corollary}{存在成比例变量的行列式为零}{}
				向量组$\alpha_1,\cdots,\alpha_i,\cdots,\alpha_j,\cdots,\alpha_n$如果有$\alpha_i=k\alpha_j,k\in F$

				那么$\det (\alpha_1,\cdots,\alpha_i,\cdots,\alpha_j,\cdots,\alpha_n)=0$
			\end{corollary}
			\begin{proof}
				$\det (\alpha_1,\cdots,\alpha_i,\cdots,\alpha_j,\cdots,\alpha_n)=k\cdot \det (\alpha_1,\cdots,\alpha_j,\cdots,\alpha_j,\cdots,\alpha_n)=0$
			\end{proof}
		\subsection{行列式在基上的展开}
			\begin{them}{行列式的展开}{}
				设$F$是一个域,$V$是$F$上的一个线性空间,并且$dim_F V = n$,
				
				$V$上的$n$阶行列式$\det$满足$\det(\gamma_1,\cdots,\gamma_n)=1$,其中$\{\gamma_1,\cdots,\gamma_n\}$是$V$的一组基

				那么,有:
				\begin{equation}
					\det(\alpha_1,\cdots,\alpha_n)=\sum_{\sigma \in S_n} \text{sgn}(\sigma) \prod_{i=1}^n a_{i,\sigma(i)}
				\end{equation}
				其中$\alpha_i = \sum\limits_{j=1}^{n} a_{i,j} \gamma_j$
			\end{them}
			\begin{proof}
				$\det (\alpha_1,\cdots,\alpha_n)=\det\left(\sum\limits_{i_1=1}^{n} a_{1,i_1} \gamma_{i_1},\cdots,\sum\limits_{i_n=1}^{n} a_{n,i_n} \gamma_{i_n}\right)$

				$=\sum\limits_{i_1=1}^{n}\cdots \sum\limits_{i_n=1}^{n} \left(\prod\limits_{k=1}^{n}a_{k,i_k} \det(\alpha_{i_1},\cdots,\alpha_{i_n})\right)$

				$=\sum\limits_{\sigma \in S_n} \left(\prod\limits_{k=1}^{n}a_{k,\sigma(k)} \text{sgn}(\sigma )\right)$
			\end{proof}
			事实上,我们也可以改变第一个求和指标,使之称为一个固定但是可以随意选取的置换
			\begin{corollary}{}{}
				设$F$是一个域,$V$是$F$上的一个线性空间,并且$dim_F V = n$,
				
				$V$上的$n$阶行列式$\det$满足$\det(\gamma_1,\cdots,\gamma_n)=1$,其中$\{\gamma_1,\cdots,\gamma_n\}$是$V$的一组基

				那么,有:
				\begin{equation}
					\det(\alpha_1,\cdots,\alpha_n)=\text{sgn}(\rho )\sum_{\sigma \in S_n} \text{sgn}(\sigma) \prod_{i=1}^n a_{\rho(i),\sigma(i)}
				\end{equation}
				其中$\alpha_i = \sum\limits_{j=1}^{n} a_{i,j} \gamma_j$,$\rho$是一个置换
			\end{corollary}
			\begin{proof}
				$\det(\alpha_1,\cdots,\alpha_n)=\sum_{\tau  \in S_n} \text{sgn}(\tau ) \prod_{i=1}^n a_{i,\tau (i)}$

				对指标作置换$\rho $,累乘的结果不会变化,所以有:

				$\det(\alpha_1,\cdots,\alpha_n)=\sum_{\tau \in S_n} \text{sgn}(\tau ) \prod_{i=1}^n a_{\rho(i),(\rho \circ \tau )(i)}$

				记$\sigma =\rho \circ \tau $,那么$\det(\alpha_1,\cdots,\alpha_n)=\sum_{\rho^{-1} \circ \sigma  \in S_n} \text{sgn}(\rho^{-1} \circ \sigma) \prod_{i=1}^n a_{\rho(i),\sigma (i)}$

				但是,$\rho^{-1} \circ \sigma \in S_n$其实就是$\sigma \in S_n$,并且我们知道$\text{sgn}(\rho^{-1} \circ \sigma)=\text{sgn}(\rho )\text{sgn}(\sigma )$

				所以$\det(\alpha_1,\cdots,\alpha_n)=\text{sgn}(\rho )\sum_{\sigma \in S_n} \text{sgn}(\sigma) \prod_{i=1}^n a_{\rho(i),\sigma(i)}$
			\end{proof}
		\subsection{矩阵的行列式}
			我们之前已经指出,$M_n (F) \cong F^{n^2} \cong (F^n)^n$,因此,我们可以对矩阵定义行列式:
			\begin{defn}{矩阵的行列式}{}
				设矩阵$A = (\alpha_1,\cdots,\alpha_n)\in M_n (F)$,我们定义:

				$|A|=\det (A) := \det(\alpha_1,\cdots,\alpha_n)$

				并且有$\det(e_1,\cdots,e_n)=1$,其中$e_i$是标准基向量$(0,\cdots,1,\cdots,0)$,$1$在第$i$个位置上。
			\end{defn}
			矩阵的行列式也可以类似地在标准基上展开
			\begin{them}{矩阵的行列式的展开}{}
				设$F$是一个域,矩阵$A=(a_{ij}) \in M_{n\times n}(F)$
				那么,有:
				\begin{equation}
					|A|=\sum_{\sigma \in S_n} \text{sgn}(\sigma) \prod_{i=1}^n a_{i,\sigma(i)}
				\end{equation}
			\end{them}
		\subsection{矩阵的行列式的余子式展开}
		\subsection{矩阵乘积的行列式}
	\section{矩阵的初等变换、线性方程组的解}
	\section{可逆矩阵}
	\section{矩阵的分块}
\ifx\allfiles\undefined
\end{document}
\fi