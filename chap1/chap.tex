\ifx\allfiles\undefined
\documentclass[12pt, a4paper, oneside, UTF8]{ctexbook}
\def\path{../config}
\usepackage{amsmath}
\usepackage{amsthm}
\usepackage{amssymb}
\usepackage{graphicx}
\usepackage{mathrsfs}
\usepackage{enumitem}
\usepackage{geometry}
\usepackage[colorlinks, linkcolor=black]{hyperref}
\usepackage{stackengine}
\usepackage{yhmath}
\usepackage{extarrows}
\usepackage{unicode-math}
\usepackage{tikz}
\usepackage{tikz-cd}
\usepackage{pifont}
\usepackage{pgfplots}
\usepackage{tikz-3dplot}

\usepackage{fancyhdr}
\usepackage[dvipsnames, svgnames]{xcolor}
\usepackage{listings}

\definecolor{mygreen}{rgb}{0,0.6,0}
\definecolor{mygray}{rgb}{0.5,0.5,0.5}
\definecolor{mymauve}{rgb}{0.58,0,0.82}

\graphicspath{ {figure/},{../figure/}, {config/}, {../config/} }

\linespread{1.6}

\geometry{
    top=25.4mm, 
    bottom=25.4mm, 
    left=20mm, 
    right=20mm, 
    headheight=2.17cm, 
    headsep=4mm, 
    footskip=12mm
}

\setenumerate[1]{itemsep=5pt,partopsep=0pt,parsep=\parskip,topsep=5pt}
\setitemize[1]{itemsep=5pt,partopsep=0pt,parsep=\parskip,topsep=5pt}
\setdescription{itemsep=5pt,partopsep=0pt,parsep=\parskip,topsep=5pt}

\lstset{
    language=Mathematica,
    basicstyle=\tt,
    breaklines=true,
    keywordstyle=\bfseries\color{NavyBlue}, 
    emphstyle=\bfseries\color{Rhodamine},
    commentstyle=\itshape\color{black!50!white}, 
    stringstyle=\bfseries\color{PineGreen!90!black},
    columns=flexible,
    numbers=left,
    numberstyle=\footnotesize,
    frame=tb,
    breakatwhitespace=false,
} 
\usepackage[strict]{changepage} 
\usepackage{framed}
\usepackage{tcolorbox}
\tcbuselibrary{most}

\definecolor{greenshade}{rgb}{0.90,1,0.92}
\definecolor{redshade}{rgb}{1.00,0.88,0.88}
\definecolor{brownshade}{rgb}{0.99,0.95,0.9}
\definecolor{lilacshade}{rgb}{0.95,0.93,0.98}
\definecolor{orangeshade}{rgb}{1.00,0.88,0.82}
\definecolor{lightblueshade}{rgb}{0.8,0.92,1}
\definecolor{purple}{rgb}{0.81,0.85,1}

% #### 将 config.tex 中的定理环境的对应部分替换为如下内容
% 定义单独编号,其他四个共用一个编号计数 这里只列举了五种,其他可类似定义(未定义的使用原来的也可)
\newtcbtheorem[number within=section]{defn}%
{定义}{colback=OliveGreen!10,colframe=Green!70,fonttitle=\bfseries}{def}

\newtcbtheorem[number within=section]{lemma}%
{引理}{colback=Salmon!20,colframe=Salmon!90!Black,fonttitle=\bfseries}{lem}

% 使用另一个计数器 use counter from=lemma
\newtcbtheorem[use counter from=lemma, number within=section]{them}%
{定理}{colback=SeaGreen!10!CornflowerBlue!10,colframe=RoyalPurple!55!Aquamarine!100!,fonttitle=\bfseries}{them}

\newtcbtheorem[use counter from=lemma, number within=section]{criterion}%
{准则}{colback=green!5,colframe=green!35!black,fonttitle=\bfseries}{cri}

\newtcbtheorem[use counter from=lemma, number within=section]{corollary}%
{推论}{colback=Emerald!10,colframe=cyan!40!black,fonttitle=\bfseries}{cor}
% colback=red!5,colframe=red!75!black

% 这个颜色我不喜欢
%\newtcbtheorem[number within=section]{proposition}%
%{命题}{colback=red!5,colframe=red!75!black,fonttitle=\bfseries}{cor}

% .... 命题 例 注 证明 解 使用之前的就可以(全文都是这种框框就很丑了),也可以按照上述定义 ...
\renewenvironment{proof}{\par\textbf{证明:}\;}{\qed\par}
\newenvironment{solution}{\par{\textbf{解:}}\;}{\qed\par}
\newtheorem{proposition}{\indent 命题}[section]
\newtheorem{example}{\indent \color{SeaGreen}{例}}[section] % 绿色文字的 例 ,不需要就去除\color{SeaGreen}{}
\newtheorem*{rmk}{\indent 注}
\usepackage{amssymb}
\setmathfont{LatinModernMath-Regular}
\setmathfont[range=\mathbb]{TeXGyrePagellaMath-Regular}
\def\d{\mathrm{d}}
\def\R{\mathbb{R}}
\def\C{\mathbb{C}}
\def\Q{\mathbb{Q}}
\def\N{\mathbb{N}}
\def\Z{\mathbb{Z}}
\newcommand{\bs}[1]{\boldsymbol{#1}}
\newcommand{\ora}[1]{\overrightarrow{#1}}
\newcommand{\myspace}[1]{\par\vspace{#1\baselineskip}}
\newcommand{\xrowht}[2][0]{\addstackgap[.5\dimexpr#2\relax]{\vphantom{#1}}}
\newenvironment{ca}[1][1]{\linespread{#1} \selectfont \begin{cases}}{\end{cases}}
\newenvironment{vx}[1][1]{\linespread{#1} \selectfont \begin{vmatrix}}{\end{vmatrix}}
\newcommand{\tabincell}[2]{\begin{tabular}{@{}#1@{}}#2\end{tabular}}
\newcommand{\pll}{\kern 0.56em/\kern -0.8em /\kern 0.56em}
\newcommand{\dive}[1][F]{\mathrm{div}\;\bs{#1}}
\newcommand{\rotn}[1][A]{\mathrm{rot}\;\bs{#1}}
\usepackage{xeCJK}
\setCJKmainfont{SimSun}[BoldFont={SimHei}, ItalicFont={KaiTi}] % 设置中文支持

\newcommand{\point}[1]{\item {#1}}
\newenvironment{para}[1]{%
\ifcase#1\relax
\begin{enumerate}[label=\arabic*.] % 1.2.3.
\or
\begin{enumerate}[label=\textcircled{\arabic*}] % ①②③
\or
\begin{enumerate}[label=(\roman*)] % (i)(ii)(iii)
\else
\begin{enumerate}[label=\arabic*.] % 默认格式
\fi
}{
\end{enumerate}
}

\def\myIndex{0}
% \input{\path/cover_package_\myIndex.tex}

\def\myTitle{高等代数笔记}
\def\myAuthor{Zhang Liang}
\def\myDateCover{\today}
\def\myDateForeword{\today}
\def\myForeword{前言标题}
\def\myForewordText{
    前言内容
}
\def\mySubheading{副标题}


\begin{document}
% \input{\path/cover_text_\myIndex.tex}

\newpage
\thispagestyle{empty}
\begin{center}
    \Huge\textbf{\myForeword}
\end{center}
\myForewordText
\begin{flushright}
    \begin{tabular}{c}
        \myDateForeword
    \end{tabular}
\end{flushright}

\newpage
\pagestyle{plain}
\setcounter{page}{1}
\pagenumbering{Roman}
\tableofcontents

\newpage
\pagenumbering{arabic}
\setcounter{chapter}{0}
\setcounter{page}{0}

\pagestyle{fancy}
\fancyfoot[C]{\thepage}
\renewcommand{\headrulewidth}{0.4pt}
\renewcommand{\footrulewidth}{0pt}








\else
\fi
%标题
\chapter{线性空间}
	\section{线性空间的定义}
		\subsection{线性空间的定义}
			\begin{defn}{线性空间}{}
				设$F$是一个域,$V$是一个集合,存在两个运算$+:V\times V \rightarrow V$和$\cdot : F \times V \rightarrow V$,分别称为加法和乘法,使得:

				\ding{172} $\exists \mathbf{0} \in V,\forall \alpha \in V,\mathbf{0}+\alpha=\alpha $

				\ding{173} $\forall \alpha \in V,\exists -\alpha \in V,\text{s.t. } \alpha + (-\alpha )=\mathbf{0}$

				\ding{174} $\forall \alpha, \beta \in V,\alpha + \beta = \beta + \alpha$

				\ding{175} $\forall \alpha, \beta, \gamma \in V,(\alpha+\beta)+\gamma=\alpha+(\beta+\gamma)$

				\ding{176} $\forall \alpha \in V,1 \cdot \alpha =\alpha $

				\ding{177} $\forall k,l \in F,\alpha \in V,(k\cdot l)\cdot \alpha =k\cdot (l\cdot \alpha )$

				\ding{178} $\forall k,l \in F,\alpha \in V,(k+l)\cdot \alpha=k\cdot \alpha + l\cdot \alpha  $

				\ding{179} $\forall k \in F,\alpha \in V,k\cdot (\beta + \gamma )=k\cdot \beta + k\cdot \gamma $

				那么我们称$V$是一个$F$上的线性空间(或向量空间)
			\end{defn}
		\subsection{线性子空间}
			\begin{defn}{线性子空间}{}
				设$V$是一个线性空间,集合$W \subseteq V$,如果$W$在$V$的运算构成一个线性空间,那么我们称$W$是$V$的一个线性子空间
			\end{defn}
		\subsection{线性空间的性质}
		\begin{para}{0}
			\point{}
				\begin{proposition}
					$\mathbf{0}$是唯一的
				\end{proposition}
				\begin{proof}
					不妨假设命题不成立,$\mathbf{0}_1,\mathbf{0}_2$均是零元,并且$\mathbf{0}_1 \neq \mathbf{0}_2$

					我们注意到:$\mathbf{0}_1 = \mathbf{0}_1 + \mathbf{0}_2 = \mathbf{0}_2$,与假设矛盾,于是命题得证
				\end{proof}
			\point{}
				\begin{proposition}
					$\forall \alpha \in V$,$-\alpha $是唯一的
				\end{proposition}
				\begin{proof}
					不妨假设命题不成立,$\alpha $有两个逆元$\beta_1,\beta_2$,并且$\beta_1 \neq \beta_2$

					我们注意到:$\beta_1 = \mathbf{0} + \beta_1 = (\beta_2 + \alpha )+\beta_1 = \beta_2 + (\alpha +\beta_1) = \beta_2 + \mathbf{0}= \beta_2$,与假设矛盾,于是命题得证
				\end{proof}
			\point{}
				\begin{proposition}
					$\forall \alpha \in V,0\cdot \alpha = \mathbf{0}$
				\end{proposition}
				\begin{proof}
					$0\cdot \alpha =(0+0)\cdot \alpha =0\cdot \alpha +0\cdot \alpha $

					$\Rightarrow 0\cdot \alpha + (-0\cdot \alpha )=0\cdot \alpha + 0\cdot \alpha +(-0\cdot \alpha )$

					$\Rightarrow \mathbf{0} = 0\cdot \alpha $
				\end{proof}
			\point{}
				\begin{proposition}
					$\forall k \in F,k \cdot \mathbf{0} = \mathbf{0}$
				\end{proposition}
				\begin{proof}
					$k \cdot \mathbf{0} = k\cdot (0\cdot \alpha ) = (k\cdot 0)\cdot \alpha = 0\cdot \alpha = \mathbf{0}$
				\end{proof}
			\point{}
				\begin{proposition}
					$\forall \alpha \in V,(-1)\cdot \alpha = -\alpha $
				\end{proposition}
				\begin{proof}
					$\mathbf{0} = 0\cdot \alpha =\left(1+(-1)\right)\cdot \alpha =1\cdot \alpha + (-1)\cdot \alpha $

					$\Rightarrow \mathbf{0}+(-\alpha )=(-\alpha )+\alpha + (-1)\cdot \alpha \Rightarrow (-1)\cdot \alpha = -\alpha $
				\end{proof}
			\point{}
			事实上,验证一个子集是否是线性子空间,只需要验证封闭性即可,其他的条件都是不必要的。
				\begin{proposition}
					$W \subseteq V$是$V$的线性子空间,当且仅当:
	
					$\forall k \in F,\alpha ,\beta \in V,\alpha +\beta \in V,k\cdot \alpha \in V$
				\end{proposition}
				\begin{proof}
					充分性是显然的,对于必要性,我们依次验证:
	
					首先,$0 \in F$,因此$0\cdot \alpha =\mathbf{0}\in W$

					其次,因为$-1 \in F$,所以$(-1)\cdot \alpha = -\alpha \in W$

					剩余的六条运算律,因为$W$上的运算即是$V$上的运算在$W$上的限制,所以显然成立。那么命题得证。
				\end{proof}
		\end{para}
	\section{线性组合}
	\section{极大线性无关向量组、向量组的秩}
	\section{基、线性空间的维数}
	\section{线性子空间的直和}
	\section{线性空间的同构}
	\section{商空间}
\ifx\allfiles\undefined
\end{document}
\fi