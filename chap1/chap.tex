\ifx\allfiles\undefined
\documentclass[12pt, a4paper, oneside, UTF8]{ctexbook}
\def\path{../config}
\input{../config/_config}
\pgfplotsset{compat=1.18}
\begin{document}
% \input{../config/cover}
\else
\fi
%标题
\chapter{线性空间}
	\section{线性空间的定义}
		\subsection{线性空间的定义}
			\begin{defn}{线性空间}{}
				设$F$是一个域,$V$是一个集合,存在两个运算$+:V\times V \rightarrow V$和$\cdot : F \times V \rightarrow V$,分别称为加法和乘法,使得:

				\ding{172} $\exists \mathbf{0} \in V,\forall \alpha \in V,\mathbf{0}+\alpha=\alpha $

				\ding{173} $\forall \alpha \in V,\exists -\alpha \in V,\text{s.t. } \alpha + (-\alpha )=\mathbf{0}$

				\ding{174} $\forall \alpha, \beta \in V,\alpha + \beta = \beta + \alpha$

				\ding{175} $\forall \alpha, \beta, \gamma \in V,(\alpha+\beta)+\gamma=\alpha+(\beta+\gamma)$

				\ding{176} $\forall \alpha \in V,1 \cdot \alpha =\alpha $

				\ding{177} $\forall k,l \in F,\alpha \in V,(k\cdot l)\cdot \alpha =k\cdot (l\cdot \alpha )$

				\ding{178} $\forall k,l \in F,\alpha \in V,(k+l)\cdot \alpha=k\cdot \alpha + l\cdot \alpha  $

				\ding{179} $\forall k \in F,\alpha \in V,k\cdot (\beta + \gamma )=k\cdot \beta + k\cdot \gamma $

				那么我们称$V$是一个$F$上的线性空间(或向量空间)
			\end{defn}
		\subsection{线性子空间}
			\begin{defn}{线性子空间}{}
				设$V$是一个线性空间,集合$W \subseteq V$,如果$W$在$V$的运算构成一个线性空间,那么我们称$W$是$V$的一个线性子空间
			\end{defn}
		\subsection{线性空间的性质}
		\begin{para}{0}
			\point{}
				\begin{proposition}
					$\mathbf{0}$是唯一的
				\end{proposition}
				\begin{proof}
					不妨假设命题不成立,$\mathbf{0}_1,\mathbf{0}_2$均是零元,并且$\mathbf{0}_1 \neq \mathbf{0}_2$

					我们注意到:$\mathbf{0}_1 = \mathbf{0}_1 + \mathbf{0}_2 = \mathbf{0}_2$,与假设矛盾,于是命题得证
				\end{proof}
			\point{}
				\begin{proposition}
					$\forall \alpha \in V$,$-\alpha $是唯一的
				\end{proposition}
				\begin{proof}
					不妨假设命题不成立,$\alpha $有两个逆元$\beta_1,\beta_2$,并且$\beta_1 \neq \beta_2$

					我们注意到:$\beta_1 = \mathbf{0} + \beta_1 = (\beta_2 + \alpha )+\beta_1 = \beta_2 + (\alpha +\beta_1) = \beta_2 + \mathbf{0}= \beta_2$,与假设矛盾,于是命题得证
				\end{proof}
			\point{}
				\begin{proposition}
					$\forall \alpha \in V,0\cdot \alpha = \mathbf{0}$
				\end{proposition}
				\begin{proof}
					$0\cdot \alpha =(0+0)\cdot \alpha =0\cdot \alpha +0\cdot \alpha $

					$\Rightarrow 0\cdot \alpha + (-0\cdot \alpha )=0\cdot \alpha + 0\cdot \alpha +(-0\cdot \alpha )$

					$\Rightarrow \mathbf{0} = 0\cdot \alpha $
				\end{proof}
			\point{}
				\begin{proposition}
					$\forall k \in F,k \cdot \mathbf{0} = \mathbf{0}$
				\end{proposition}
				\begin{proof}
					$k \cdot \mathbf{0} = k\cdot (0\cdot \alpha ) = (k\cdot 0)\cdot \alpha = 0\cdot \alpha = \mathbf{0}$
				\end{proof}
			\point{}
				\begin{proposition}
					$\forall \alpha \in V,(-1)\cdot \alpha = -\alpha $
				\end{proposition}
				\begin{proof}
					$\mathbf{0} = 0\cdot \alpha =\left(1+(-1)\right)\cdot \alpha =1\cdot \alpha + (-1)\cdot \alpha $

					$\Rightarrow \mathbf{0}+(-\alpha )=(-\alpha )+\alpha + (-1)\cdot \alpha \Rightarrow (-1)\cdot \alpha = -\alpha $
				\end{proof}
			\point{}
			事实上,验证一个子集是否是线性子空间,只需要验证封闭性即可,其他的条件都是不必要的。
				\begin{proposition}
					$W \subseteq V$是$V$的线性子空间,当且仅当:
	
					$\forall k \in F,\alpha ,\beta \in V,\alpha +\beta \in V,k\cdot \alpha \in V$
				\end{proposition}
				\begin{proof}
					充分性是显然的,对于必要性,我们依次验证:
	
					首先,$0 \in F$,因此$0\cdot \alpha =\mathbf{0}\in W$

					其次,因为$-1 \in F$,所以$(-1)\cdot \alpha = -\alpha \in W$

					剩余的六条运算律,因为$W$上的运算即是$V$上的运算在$W$上的限制,所以显然成立。那么命题得证。
				\end{proof}
		\end{para}
	\section{线性组合}
		本节中,我们着手研究一系列向量以相加,相乘的形式组合而成的新向量,称之为线性组合。

		其最一般的形式是$\sum\limits_{i=1}^{n} \alpha_i$,但是我们希望能够进一步拓展,考虑无穷多个向量的线性组合。

		因此,我们需要先定义向量列的极限,其最好方式就是先定义出线性空间中的度量。
		\subsection{度量线性空间}
			\begin{defn}{度量线性空间、线性组合的定义}{}
				设$V$是一个域$F$上的线性空间,映射$d:V\times V \rightarrow F$如果满足:

				\ding{172} $\forall \alpha ,\beta \in V,d(\alpha ,\beta )=0 \Leftrightarrow \alpha =\beta $
			
				\ding{173} $\forall \alpha ,\beta \in V,d(\alpha ,\beta )=d(\beta ,\alpha )$

				\ding{174} $\forall \alpha,\beta,\gamma \in V,d(\alpha,\beta )+d(\beta ,\gamma ) \geqslant d(\alpha ,\gamma )$

				那么我们称$(V,d)$是一个度量线性空间(或Banach空间),$d$称为$V$中的度量
			\end{defn}
			这样,我们就可以定义出线性组合
			\begin{defn}{线性组合}{}
				设$V$是一个域$F$上的线性空间,$\{\alpha_i \in V\}$是一个至多可数的向量组

				那么如果向量$\beta \in V$满足:$\forall \varepsilon > 0,\exists,n \in \N_+,\text{s.t. }d(\sum\limits_{i=1}^{n} k_i\alpha_i,\beta ) < \varepsilon ,k_i \in F$

				那么我们称$\beta$是一个由向量组$\{\alpha_i \in V\}$张成的的线性组合,并称$\beta $可以由$\{\alpha_i \in V\}$线性表出
				
				如果$\{\alpha_i \in V\}$可数,记作:$\beta =\sum\limits_{i=1}^{\infty} \alpha_i$,否则记作$\beta =\sum\limits_{i=1}^{n} \alpha_i$,其中$n$是向量组的元素数

				由向量组$S$张成的全体线性组合形成的子空间记作:$span_F (S)$(或$L(S)$,$\langle S \rangle $)
			\end{defn}
			我们不考虑不可数的向量组张成的线性组合,因为不可数的向量的求和的定义通常是不良好的。

			此后除非特别说明,我们默认我们所说的向量组指的是至多可数的向量组。
		\subsection{向量组的线性相关性和线性无关性}
			我们观察到一个现象:向量组张成线性子空间时,并非每一个向量都有贡献,有些向量是不必要的。就像下图中,尽管三个向量张成了整个平面,但是其实任取其中之二,就足以张成整个平面。

			% 设置3D视图角度
			\tdplotsetmaincoords{75}{125}
			\begin{tikzpicture}[tdplot_main_coords, scale=1.5]

			% 坐标原点
			\coordinate (A) at (0,0,0);

			% 向量终点坐标
			\coordinate (B) at (3.66,4.16,0);
			\coordinate (C) at (3,-2,0);
			\coordinate (D) at (4.5,0.7,0);

			% 背景平面(淡蓝色)
			\fill[cyan!15, opacity=0.7] (-5,-5,0) -- (5,-5,0) -- (5,5,0) -- (-5,5,0) -- cycle;

			% 向量
			\draw[->, thick, black] (A) -- (B);
			\draw[->, thick, black] (A) -- (C);
			\draw[->, thick, black] (A) -- (D);

			% 标签(无圆点)
			\draw (B) node[anchor=east, blue] {B};
			\draw (C) node[anchor=west, blue] {C};
			\draw (D) node[anchor=south, blue] {D};

			% 点
			\filldraw[gray] (A) circle (2pt) node[anchor=north east] {A};

			\end{tikzpicture}
			我们会自然地认为:如果每一个向量都和其他向量的“方向”不一样,就不会有不必要的向量,或者说,不应该可以在赋予非零系数时,张成零向量

			此时,向量组被称为是“线性无关的”。
			\begin{defn}{线性无关的向量组}{}
				向量组$\{\alpha_x \in V|x \in A\}$如果满足:

				$\sum\limits_{x \in A} k_x \alpha_x = \mathbf{0} \Rightarrow \forall x \in A,k_x = 0$

				那么我们称$\{\alpha_x \in V|x \in A\}$是一个线性无关的向量组
			\end{defn}
			如果一个向量组不是线性无关的,那么我们称它是线性相关的,我们可以写出以下定义
			\begin{defn}{线性相关的向量组}{}
				向量组$\{\alpha_x \in V|x \in A\}$如果满足:

				存在不全为零的一组系数$k_x,x \in A$,使得$\sum\limits_{x \in A} k_x \alpha_x = \mathbf{0}$

				那么我们称$\{\alpha_x \in V|x \in A\}$是一个线性相关关的向量组
			\end{defn}
		\subsection{线性相关和线性无关的性质}
		\begin{para}{0}
			\point{单个向量当且仅当它是零向量是线性相关}
				\begin{proposition}
					$\alpha $线性相关 $\Leftrightarrow \alpha =\mathbf{0}$
				\end{proposition}
				\begin{proof}
					先证充分性,如果$\alpha $线性相关,那么$\exists k \in F,k \neq 0,k\alpha =\mathbf{0} \Rightarrow \alpha =\mathbf{0}$

					必要性是显然的
				\end{proof}
			\point{如果一个部分组线性相关,整个向量组也是线性相关的}
				\begin{proposition}
					如果$\{\alpha_x \in V|x \in A\}$是一个向量组,它的一个部分组$\{\alpha_x \in V| x \in B\},B \subseteq A$线性相关,那么$\{\alpha_x \in V|x \in A\}$也是线性相关的
				\end{proposition}
				\begin{proof}
					因为$\{\alpha_x \in V| x \in B\}$线性相关,所以存在不全为零的系数$k_x,x \in B$,使得$\sum\limits_{x \in B} k_x \alpha_x = \mathbf{0}$

					我们可以将$k_x$扩展到$A$上,令$k_x = 0,x \in A-B$,那么$\sum\limits_{x \in A} k_x \alpha_x = \sum\limits_{x \in B} k_x \alpha_x=\mathbf{0}$,所以$\{\alpha_x \in V|x \in A\}$线性相关
				\end{proof}
				反过来,我们可以推论出,线性无关向量组的任意部分组也是线性无关的
				\begin{corollary}{线性无关向量组的任意部分组都线性无关}{}
					若向量组$\{\alpha_x \in V|x \in A\}$线性无关,那么它的任意一个部分组$\{\alpha_x \in V| x \in B\},B \subseteq A$也是线性无关的
				\end{corollary}
				\begin{proof}
					如果并非如此,那么至少有一个部分组线性相关,那么按照命题,整个向量组线性相关,这与假设矛盾,所以命题得证
				\end{proof}
				因此,一个显然的事实是,一个包含零向量的向量组是一定线性相关的
				\begin{proposition}
					如果$ \mathbf{0} \in \{\alpha_x \in V|x \in A\}$,那么$\{\alpha_x \in V|x \in A\}$线性相关
				\end{proposition}
				\begin{proof}
					显然
				\end{proof}
			\point{向量组线性相关的充要条件是:存在一个向量可以由其他向量线性表出}
				\begin{proposition}
					向量组线性相关$\Leftrightarrow$向量组中存在一个向量可以由其他向量线性表出
				\end{proposition}
				\begin{proof}
					先证充分性。如果$\{\alpha_x \in V|x \in A\}$线性相关,那么存在不全为零的系数$k_x,x \in A$,使得$\sum\limits_{x \in A} k_x \alpha_x = \mathbf{0}$

					取其中任意一个不为零的系数$k_a \neq 0$,那么有$\alpha_a =-\frac{1}{k_a}\sum\limits_{x \in A-\{a\}} k_x \alpha_x$,充分性得证

					接下来证明必要性,如果$\alpha_a =\sum\limits_{x \in A-\{a\}} k_x \alpha_x$,那么我们只需令$k_a=-1$

					那么就有$\sum\limits_{x \in A} k_x \alpha_x = \mathbf{0}$
				\end{proof}
			\point{能唯一地表出一个向量的向量组线性无关}
				\begin{proposition}
					$\beta \in V$可以被$\{\alpha_x \in V| x\in A\}$唯一地线性表出$\Leftrightarrow\{\alpha_x \in V| x\in A\}$线性无关
				\end{proposition}
				\begin{proof}
					先证充分性。

					这其实是显然的,因为如果$\{\alpha_x \in V| x\in A\}$线性相关,那么有一个线性组合$\sum\limits_{x \in A} k_x \alpha_x = \mathbf{0}$并且$k_x$不全为零

					假设$\beta =\sum\limits_{x \in A} l_x \alpha_x$,那么也有$\sum\limits_{x \in A} (k_x + l_x) \alpha_x$,这与假设矛盾

					再证必要性。

					不妨假设$\beta =\sum\limits_{x \in A} l_x \alpha_x=\sum\limits_{x \in A} k_x \alpha_x \exists x \in A,k_x \neq l_x$

					那么$\sum\limits_{x \in A} (l_x-k_x) \alpha_x = \mathbf{0}$,又因为$\{\alpha_x \in V| x\in A\}$线性无关,所以$l_x-k_x=0$。因此表出方式是唯一的,命题得证
				\end{proof}
			\point{向线性无关向量组中加入一个向量,如果变得线性相关,那么这个向量可以由其他向量线性表出}
				\begin{proposition}
					如果$\{\alpha _x \in V| x \in A\}$线性无关,$\{\beta ,\alpha_x \in V| x\in A\}$线性相关

					那么$\beta $可由$\{\alpha_x \in V| x\in A\}$线性表出
				\end{proposition}
				\begin{proof}
					因为$\{\beta ,\alpha_x \in V| x\in A\}$线性相关,所以存在不全为零的系数$k_x,l\in F,x \in A$,使得$\sum\limits_{x \in A} k_x \alpha_x + l \beta = \mathbf{0}$

					我们只需证$l \neq 0$。我们不妨假设$l=0$

					那么$\sum\limits_{x \in A} k_x \alpha_x = \mathbf{0}$。但是$\{\alpha_x \in V| x\in A\}$线性无关,所以$\forall x \in A,k_x =0$,这与系数不全为零的假设矛盾

					那么有$l \neq 0$,于是$\beta  =-\frac{1}{l}\sum\limits_{x \in A} k_x \alpha_x$,命题得证
				\end{proof}
		\end{para}
	\section{极大线性无关向量组、向量组的秩}
		前面我们讨论了线性无关向量组,它可以说是“所有元素都有贡献”的,接下来我们讨论,如果把一个向量集剔除至恰好包含所有“有贡献”的向量,这种向量组有什么性质

		我们的一个想法是,如果剔除至向量组本身依旧线性无关,但是加入任意向量就变得线性相关,那么这个向量组就是我们想要的“极大线性无关向量组”
		\subsection{极大线性无关向量组}
			\begin{defn}{极大线性无关向量组}{}
				设$\{\alpha_x \in V| x \in A\}$是一个向量组

				如果它的一个部分组$\{\alpha_x \in V| x \in B\},B\subseteq A$线性无关,并且$\forall \beta \in V,\{\beta ,\alpha_x| x \in B\}$线性相关

				那么我们称$\{\alpha_x \in V| x \in B\}$是$\{\alpha_x \in V| x \in A\}$的一个极大线性无关向量组
			\end{defn}
			我们直观地认为,每一个极大线性无关向量组应该具备相同的元素数(或势),就像在三维空间中,我们可以选取不同的坐标系,但是每个坐标系都恰好三个坐标轴
		\subsection{向量组的等价}
			我们在向量组上定义以下等价关系:
			\begin{defn}{向量组的线性表出}{}
				如果向量组$\{\alpha_x \in V| x \in A\}$中的任意向量可由$\{\beta_x \in V| x \in B\}$线性表出

				那么我们称$\{\alpha_x \in V| x \in A\}$可由$\{\beta _x \in V| x \in B\}$线性表出
			\end{defn}
			\begin{defn}{向量组的等价}{}
				如果向量组$\{\alpha_x \in V| x \in A\}$可由$\{\beta_x \in V| x \in B\}$线性表出,并且$\{\beta_x \in V| x \in B\}$可由$\{\alpha_x \in V| x \in A\}$线性表出

				那么我们称$\{\alpha_x \in V| x \in A\}$和$\{\beta_x \in V| x \in B\}$是等价的,记作:$\{\alpha_x \in V| x \in A\} \cong \{\beta_x \in V| x \in B\}$
			\end{defn}
			我们首先验证它的确是一个等价关系
			\begin{proposition}
				向量组的等价是一个等价关系
			\end{proposition}
			\begin{proof}
				自反性是显然的,因为向量组中的任意向量$\alpha_x$可以由它自己线性表出;

				对称性由定义是显然的;

				最后证明传递性,假设$\{\alpha_x \in V| x \in A\} \cong \{\beta _x \in V| x \in B\},\{\beta _x \in V| x \in B\}\cong\{\gamma _x \in V| x \in C\}$

				设$\alpha_x = \sum\limits_{y \in B} k_{xy} \beta_y,\beta_y = \sum\limits_{z \in C} l_{yz} \gamma_z$

				那么$\alpha_x = \sum\limits_{y \in B} k_{xy} \left(\sum\limits_{z \in C} l_{yz} \gamma_z\right)=\sum\limits_{y \in B,z\in C} k_{xy}l_{yz} \gamma_z$

				因此$\{\alpha_x \in V| x \in A\}$可由$\{\alpha_x \in V| x \in A\}$线性表出,同理可证$\{\gamma_x \in V| x \in C\}$可由$\{\alpha_x \in V| x \in A\}$线性表出
				
				所以$\{\alpha_x \in V| x \in A\} \cong \{\gamma_x \in V| x \in C\}$,传递性得证。于是命题得证
			\end{proof}
			接下来讨论一些向量组等价的性质:
			\begin{para}{0}
				\point{任何向量组和它的极大线性无关向量组等价}
					\begin{proposition}
						任何向量组和它的极大线性无关向量组等价
					\end{proposition}
					\begin{proof}
						极大线性无关向量组显然可以由原向量组线性表出;

						而按照极大线性无关向量组的定义,向其中加入任意一个向量,向量组会变得线性相关,于是,原向量组中的任意一个向量都可以由极大线性无关向量组线性表出。命题得证
					\end{proof}
				\point{向量组的两个极大线性无关向量组等价}
					\begin{proposition}
						向量组的两个极大线性无关向量组等价
					\end{proposition}
					\begin{proof}
						只需利用上面的性质和等价的传递性即可完成证明。
					\end{proof}
				\point{一个向量组如果可以由一个势比它小的向量组线性表出,那么它线性相关}
					在给出命题及其证明前,我们先证明一个显然的引理
					\begin{lemma}{}{}
						向量组$\{\alpha_1,\cdots,\alpha_n\}$如果线性相关,
						
						那么一定$\exists k,\text{s.t. }\alpha_k$可由$\alpha_1,\cdots,\alpha_{k-1}$线性表出,并且$\text{span}(\alpha_1,\cdots,\alpha_{i-1},\alpha_{i+1},\cdots,\alpha_n)=\text{span}(\alpha_1,\cdots,\alpha_n)$
					\end{lemma}
					接下来给出命题
					\begin{proof}
						因为$\{\alpha_1,\cdots,\alpha_n\}$线性相关,所以一定$\exists k_1,\cdots,k_n,k_1 \alpha_1+\cdots+k_n \alpha_n = \mathbf{0}$

						只需选取满足$k_i \neq 0$的最大的下标$p$,那么有$\alpha_1+\cdots+k_p \alpha_p = \mathbf{0}$

						于是$\alpha_p = -\frac{1}{k_p} \sum\limits_{i=1}^{p-1} \alpha_i$

						此时因为$\alpha_p$可由前面的向量线性表出,因此去除其不会对生成的子空间有任何影响,命题得证。
					\end{proof}
					\begin{proposition}
						如果向量组$\{\alpha_x \in V| x \in A\}$可由$\{\beta_x \in V| x \in B\}$线性表出,并且$\text{Card } A > \text{Card }B$

						那么$\{\alpha_x \in V| x \in A\}$线性相关
					\end{proposition}
					\begin{proof}
						先考虑两者都是有限向量组的情况

						我们可以证明:如果$\{\alpha_1,\cdots,\alpha_m\}$可由$\{\beta_1,\cdots,\beta_n\}$线性表出,并且$m > n$,那么$\{\alpha_1,\cdots,\alpha_m\}$线性相关

						记$W=\text{span}(\beta _1, \dots, \beta _n)$。

						我们采取以下方法证明:用$\alpha_i$逐步替换$\beta_j$,但是保持生成的子空间不变。最后,我们一定会发现,$\beta_j$已经被全部替换,但是此时依旧有$\alpha_i$未曾被加入向量组过。这就说明了$\{\alpha_1,\cdots,\alpha_m\}$中有向量可以被其它向量线性表出,即$\{\alpha_1,\cdots,\alpha_m\}$线性相关
						
						也就是说,我们希望构造一系列向量组 $B_0, B_1, \dots, B_n$,使得对于 $j=0, 1, \dots, n$,向量组 $B_j$ 满足:
						\begin{enumerate}
							\item $B_j$ 的长度为 $n$。
							\item $\text{span}(B_j) = W$。
							\item $B_j$ 包含向量 $\alpha _1, \dots, \alpha _j$ 以及从原始向量组 $\{\beta _1, \dots, \beta _n\}$ 中剩下的 $n-j$ 个向量。
						\end{enumerate}
						
						我们首先取$B_0 = \{\beta _1, \dots, \beta _n\}$

						我们如此由$B_{j-1}$迭代出向量组$B_j$:

						考虑向量组 $\alpha_j \cup B_{j-1}$(并且不失一般性地将$\alpha_j$置于向量组的首位)。由于$\alpha_j \in W$,所以一定有$\text{span}(\alpha_j \cup B_{j-1})=W$
						
						由于 $\text{span}(B_{j-1})=W$,向量 $\alpha _j$ 在 $B_{j-1}$ 生成的子空间中,因此向量组 $\alpha _j\cup B_{j-1}$ 是线性相关的。

						根据前面的引理,线性相关的向量组 $\alpha_j\cup B_{j-1}$ 中存在一个向量,它是其前面向量的线性组合。设这个向量为 $w$。
						
						如果 $w = \alpha_j$,则 $\alpha_j$ 是空向量组的线性组合,意味着 $\alpha_j = \mathbf{0}$,于是该向量组是线性相关的,证明结束。

						接下来考虑$w$是 $B_{j-1}$ 中的一个向量。如果$w \in \{\alpha_1,\cdots,\alpha_{j-1}\}$,那么它可以被$\{\alpha_1,\cdots,\alpha_m\}$中的向量线性表出,那么向量组线性相关,证明结束。

						因此只需考虑这个向量 $w$ 是 $B_{j-1}$ 中那些来自 $\{\beta _1, \dots, \beta _n\}$ 的向量

						由于 $w$ 可以由向量组中前面的向量线性表出,我们可以从 $\alpha_j \cup B_{j-1}$ 中移除 $w$,得到新的向量组 $B_j = (\alpha_j \cup B_{j-1})-w$。这个新的向量组 $B_j$ 仍然生成子空间 $W$
						
						这样,我们就完成了向量组序列$B_0, B_1, \dots, B_n$的构造。
						
						此时,我们只需注意到,$\text{span}(B_n)=\text{span}(\alpha_1,\cdots,\alpha _n)=W,\alpha_{n+1} \in W$。于是$\alpha_{n+1}$可由$\alpha_1,\cdots,\alpha_n$线性表出,于是$\{\alpha_1,\cdots,\alpha_m\}$线性相关,证明完毕。

						最后,我们考虑$\text{Card} A = \N,\text{Card} B < \N$的情况。此时,一定能找到$\{\alpha_x \in V | x \in A\}$的一个有限部分组线性相关,于是整个向量组就线性相关了。

						于是命题得证。
					\end{proof}
				\point{向量组的两个极大线性无关向量组等势}
					\begin{proposition}
						向量组的两个极大线性无关向量组等势
					\end{proposition}
					\begin{proof}
						设有两个极大线性无关向量组$\{\alpha_x \in V| x\in A\},\{\beta_x \in V| x \in B\}$

						根据前面的性质,他们相互等价,因此可以相互线性表出。所以一定有$\text{Card }A \leqslant \text{Card }B,\text{Card }B \text{Card }A$

						那么一定有$\text{Card }A = \text{Card }B$,所以命题得证
					\end{proof}
			\end{para}
		\subsection{向量组的秩}
			在完成了前面的准备工作后,我们可以定义向量组的秩了
			\begin{defn}{向量组的秩}{}
				向量组$G=\{\alpha_x \in V| x \in A\}$的极大线性无关向量组的势称为向量组的秩,记作:$rank(G)$
			\end{defn}
			我们前面的命题已经保证了,无论如何选取极大线性无关组,它的势都相同,因而秩也是相同的。
			
			接下来讨论它的性质
			\begin{para}{0}
				\point{}
					\begin{proposition}
						向量组$\{\alpha_1,\cdots,\alpha_n\}$线性无关$\Leftrightarrow rank(\alpha_1,\cdots,\alpha_n)=n$
					\end{proposition}
					\begin{proof}
						先证充分性,如果向量组线性无关,那么它自身即是自己的一个极大线性无关向量组,所以$rank(\alpha_1,\cdots,\alpha_n)=n$

						再证必要性。如果$rank(\alpha_1,\cdots,\alpha_n)=n$,那么它仅有一个极大线性无关向量组,即向量组本身,所以它线性无关,命题得证
					\end{proof}
			\end{para}

	\section{基、线性空间的维数}
	\section{线性子空间的直和}
	\section{线性空间的同构}
	\section{商空间}
	\section{对偶空间}
\ifx\allfiles\undefined
\end{document}
\fi