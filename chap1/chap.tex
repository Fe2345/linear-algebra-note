\ifx\allfiles\undefined
\documentclass[12pt, a4paper, oneside, UTF8]{ctexbook}
\def\path{../config}
\input{../config/_config}
\begin{document}
% \input{../config/cover}
\else
\fi
%标题
\chapter{线性空间}
	\section{线性空间的定义}
		\subsection{线性空间的定义}
			\begin{defn}{线性空间}{}
				设$F$是一个域,$V$是一个集合,存在两个运算$+:V\times V \rightarrow V$和$\cdot : F \times V \rightarrow V$,分别称为加法和乘法,使得:

				\ding{172} $\exists \mathbf{0} \in V,\forall \alpha \in V,\mathbf{0}+\alpha=\alpha $

				\ding{173} $\forall \alpha \in V,\exists -\alpha \in V,\text{s.t. } \alpha + (-\alpha )=\mathbf{0}$

				\ding{174} $\forall \alpha, \beta \in V,\alpha + \beta = \beta + \alpha$

				\ding{175} $\forall \alpha, \beta, \gamma \in V,(\alpha+\beta)+\gamma=\alpha+(\beta+\gamma)$

				\ding{176} $\forall \alpha \in V,1 \cdot \alpha =\alpha $

				\ding{177} $\forall k,l \in F,\alpha \in V,(k\cdot l)\cdot \alpha =k\cdot (l\cdot \alpha )$

				\ding{178} $\forall k,l \in F,\alpha \in V,(k+l)\cdot \alpha=k\cdot \alpha + l\cdot \alpha  $

				\ding{179} $\forall k \in F,\alpha \in V,k\cdot (\beta + \gamma )=k\cdot \beta + k\cdot \gamma $

				那么我们称$V$是一个$F$上的线性空间(或向量空间)
			\end{defn}
		\subsection{线性子空间}
			\begin{defn}{线性子空间}{}
				设$V$是一个线性空间,集合$W \subseteq V$,如果$W$在$V$的运算构成一个线性空间,那么我们称$W$是$V$的一个线性子空间
			\end{defn}
		\subsection{线性空间的性质}
		\begin{para}{0}
			\point{}
				\begin{proposition}
					$\mathbf{0}$是唯一的
				\end{proposition}
				\begin{proof}
					不妨假设命题不成立,$\mathbf{0}_1,\mathbf{0}_2$均是零元,并且$\mathbf{0}_1 \neq \mathbf{0}_2$

					我们注意到:$\mathbf{0}_1 = \mathbf{0}_1 + \mathbf{0}_2 = \mathbf{0}_2$,与假设矛盾,于是命题得证
				\end{proof}
			\point{}
				\begin{proposition}
					$\forall \alpha \in V$,$-\alpha $是唯一的
				\end{proposition}
				\begin{proof}
					不妨假设命题不成立,$\alpha $有两个逆元$\beta_1,\beta_2$,并且$\beta_1 \neq \beta_2$

					我们注意到:$\beta_1 = \mathbf{0} + \beta_1 = (\beta_2 + \alpha )+\beta_1 = \beta_2 + (\alpha +\beta_1) = \beta_2 + \mathbf{0}= \beta_2$,与假设矛盾,于是命题得证
				\end{proof}
			\point{}
				\begin{proposition}
					$\forall \alpha \in V,0\cdot \alpha = \mathbf{0}$
				\end{proposition}
				\begin{proof}
					$0\cdot \alpha =(0+0)\cdot \alpha =0\cdot \alpha +0\cdot \alpha $

					$\Rightarrow 0\cdot \alpha + (-0\cdot \alpha )=0\cdot \alpha + 0\cdot \alpha +(-0\cdot \alpha )$

					$\Rightarrow \mathbf{0} = 0\cdot \alpha $
				\end{proof}
			\point{}
				\begin{proposition}
					$\forall k \in F,k \cdot \mathbf{0} = \mathbf{0}$
				\end{proposition}
				\begin{proof}
					$k \cdot \mathbf{0} = k\cdot (0\cdot \alpha ) = (k\cdot 0)\cdot \alpha = 0\cdot \alpha = \mathbf{0}$
				\end{proof}
			\point{}
				\begin{proposition}
					$\forall \alpha \in V,(-1)\cdot \alpha = -\alpha $
				\end{proposition}
				\begin{proof}
					$\mathbf{0} = 0\cdot \alpha =\left(1+(-1)\right)\cdot \alpha =1\cdot \alpha + (-1)\cdot \alpha $

					$\Rightarrow \mathbf{0}+(-\alpha )=(-\alpha )+\alpha + (-1)\cdot \alpha \Rightarrow (-1)\cdot \alpha = -\alpha $
				\end{proof}
			\point{}
			事实上,验证一个子集是否是线性子空间,只需要验证封闭性即可,其他的条件都是不必要的。
				\begin{proposition}
					$W \subseteq V$是$V$的线性子空间,当且仅当:
	
					$\forall k \in F,\alpha ,\beta \in V,\alpha +\beta \in V,k\cdot \alpha \in V$
				\end{proposition}
				\begin{proof}
					充分性是显然的,对于必要性,我们依次验证:
	
					首先,$0 \in F$,因此$0\cdot \alpha =\mathbf{0}\in W$

					其次,因为$-1 \in F$,所以$(-1)\cdot \alpha = -\alpha \in W$

					剩余的六条运算律,因为$W$上的运算即是$V$上的运算在$W$上的限制,所以显然成立。那么命题得证。
				\end{proof}
		\end{para}
	\section{线性组合}
	\section{极大线性无关向量组、向量组的秩}
	\section{基、线性空间的维数}
	\section{线性子空间的直和}
	\section{线性空间的同构}
	\section{商空间}
\ifx\allfiles\undefined
\end{document}
\fi