\ifx\allfiles\undefined
\documentclass[12pt, a4paper, oneside, UTF8]{ctexbook}
\def\path{../config}
\input{../config/_config}
\pgfplotsset{compat=1.18}
\begin{document}
% \input{../config/cover}
\else
\fi
%标题
\chapter{线性空间}
	作为第一章,我们研究高等代数的核心研究对象:线性空间

	正如之前研究的群、环、域等代数结构一样,我们如下构建线性空间的理论。
	
	在第一节中,我们先提出线性空间的定义;第二至四节中,我们从线性空间中的基本运算——线性组合出发,逐渐寻找表示一个向量集乃至整个空间的最佳方式

	从第五节开始,我们开始探索不同空间的相互联系。第五节中,我们将研究一种极为良好的关系:同构;第六节中,我们将研究如何将线性空间作唯一分解;第七节中,我们将研究如何从空间和它的一个子空间导出新的结构——商空间;

	在第八节中,作为第一章的最后一节,我们将为第二章的核心研究对象——线性映射作铺垫,研究一类特殊的线性映射的线性空间性质
	\section{线性空间的定义}
		\subsection{线性空间的定义}
			\begin{defn}{线性空间}{}
				设$F$是一个域,$V$是一个集合,存在两个运算$+:V\times V \rightarrow V$和$\cdot : F \times V \rightarrow V$,分别称为加法和乘法,使得:

				\ding{172} $\exists \mathbf{0} \in V,\forall \alpha \in V,\mathbf{0}+\alpha=\alpha $

				\ding{173} $\forall \alpha \in V,\exists -\alpha \in V,\text{s.t. } \alpha + (-\alpha )=\mathbf{0}$

				\ding{174} $\forall \alpha, \beta \in V,\alpha + \beta = \beta + \alpha$

				\ding{175} $\forall \alpha, \beta, \gamma \in V,(\alpha+\beta)+\gamma=\alpha+(\beta+\gamma)$

				\ding{176} $\forall \alpha \in V,1 \cdot \alpha =\alpha $

				\ding{177} $\forall k,l \in F,\alpha \in V,(k\cdot l)\cdot \alpha =k\cdot (l\cdot \alpha )$

				\ding{178} $\forall k,l \in F,\alpha \in V,(k+l)\cdot \alpha=k\cdot \alpha + l\cdot \alpha  $

				\ding{179} $\forall k \in F,\alpha \in V,k\cdot (\beta + \gamma )=k\cdot \beta + k\cdot \gamma $

				那么我们称$V$是一个$F$上的线性空间(或向量空间)
			\end{defn}
		\subsection{线性子空间}
			\begin{defn}{线性子空间}{}
				设$V$是一个线性空间,集合$W \subseteq V$,如果$W$在$V$的运算构成一个线性空间,那么我们称$W$是$V$的一个线性子空间
			\end{defn}
		\subsection{线性空间的性质}
		\begin{para}{0}
			\point{}
				\begin{proposition}
					$\mathbf{0}$是唯一的
				\end{proposition}
				\begin{proof}
					不妨假设命题不成立,$\mathbf{0}_1,\mathbf{0}_2$均是零元,并且$\mathbf{0}_1 \neq \mathbf{0}_2$

					我们注意到:$\mathbf{0}_1 = \mathbf{0}_1 + \mathbf{0}_2 = \mathbf{0}_2$,与假设矛盾,于是命题得证
				\end{proof}
			\point{}
				\begin{proposition}
					$\forall \alpha \in V$,$-\alpha $是唯一的
				\end{proposition}
				\begin{proof}
					不妨假设命题不成立,$\alpha $有两个逆元$\beta_1,\beta_2$,并且$\beta_1 \neq \beta_2$

					我们注意到:$\beta_1 = \mathbf{0} + \beta_1 = (\beta_2 + \alpha )+\beta_1 = \beta_2 + (\alpha +\beta_1) = \beta_2 + \mathbf{0}= \beta_2$,与假设矛盾,于是命题得证
				\end{proof}
			\point{}
				\begin{proposition}
					$\forall \alpha \in V,0\cdot \alpha = \mathbf{0}$
				\end{proposition}
				\begin{proof}
					$0\cdot \alpha =(0+0)\cdot \alpha =0\cdot \alpha +0\cdot \alpha $

					$\Rightarrow 0\cdot \alpha + (-0\cdot \alpha )=0\cdot \alpha + 0\cdot \alpha +(-0\cdot \alpha )$

					$\Rightarrow \mathbf{0} = 0\cdot \alpha $
				\end{proof}
			\point{}
				\begin{proposition}
					$\forall k \in F,k \cdot \mathbf{0} = \mathbf{0}$
				\end{proposition}
				\begin{proof}
					$k \cdot \mathbf{0} = k\cdot (0\cdot \alpha ) = (k\cdot 0)\cdot \alpha = 0\cdot \alpha = \mathbf{0}$
				\end{proof}
			\point{}
				\begin{proposition}
					$\forall \alpha \in V,(-1)\cdot \alpha = -\alpha $
				\end{proposition}
				\begin{proof}
					$\mathbf{0} = 0\cdot \alpha =\left(1+(-1)\right)\cdot \alpha =1\cdot \alpha + (-1)\cdot \alpha $

					$\Rightarrow \mathbf{0}+(-\alpha )=(-\alpha )+\alpha + (-1)\cdot \alpha \Rightarrow (-1)\cdot \alpha = -\alpha $
				\end{proof}
			\point{}
			事实上,验证一个子集是否是线性子空间,只需要验证封闭性即可,其他的条件都是不必要的。
				\begin{proposition}
					$W \subseteq V$是$V$的线性子空间,当且仅当:
	
					$\forall k \in F,\alpha ,\beta \in V,\alpha +\beta \in V,k\cdot \alpha \in V$
				\end{proposition}
				\begin{proof}
					充分性是显然的,对于必要性,我们依次验证:
	
					首先,$0 \in F$,因此$0\cdot \alpha =\mathbf{0}\in W$

					其次,因为$-1 \in F$,所以$(-1)\cdot \alpha = -\alpha \in W$

					剩余的六条运算律,因为$W$上的运算即是$V$上的运算在$W$上的限制,所以显然成立。那么命题得证。
				\end{proof}
		\end{para}
	\section{线性组合}
		本节中,我们着手研究一系列向量以相加,相乘的形式组合而成的新向量,称之为线性组合。

		其最一般的形式是$\sum\limits_{i=1}^{n} \alpha_i$,但是我们希望能够进一步拓展,考虑无穷多个向量的线性组合。

		因此,我们需要先定义向量列的极限,其最好方式就是先定义出线性空间中的度量。
		\subsection{度量线性空间}
			\begin{defn}{度量线性空间、线性组合的定义}{}
				设$V$是一个域$F$上的线性空间,映射$d:V\times V \rightarrow F$如果满足:

				\ding{172} $\forall \alpha ,\beta \in V,d(\alpha ,\beta )=0 \Leftrightarrow \alpha =\beta $
			
				\ding{173} $\forall \alpha ,\beta \in V,d(\alpha ,\beta )=d(\beta ,\alpha )$

				\ding{174} $\forall \alpha,\beta,\gamma \in V,d(\alpha,\beta )+d(\beta ,\gamma ) \geqslant d(\alpha ,\gamma )$

				那么我们称$(V,d)$是一个度量线性空间(或Banach空间),$d$称为$V$中的度量
			\end{defn}
			这样,我们就可以定义出线性组合
			\begin{defn}{线性组合}{}
				设$V$是一个域$F$上的线性空间,$\{\alpha_i \in V\}$是一个至多可数的向量组

				那么如果向量$\beta \in V$满足:$\forall \varepsilon > 0,\exists,n \in \N_+,\text{s.t. }d(\sum\limits_{i=1}^{n} k_i\alpha_i,\beta ) < \varepsilon ,k_i \in F$

				那么我们称$\beta$是一个由向量组$\{\alpha_i \in V\}$张成的的线性组合,并称$\beta $可以由$\{\alpha_i \in V\}$线性表出
				
				如果$\{\alpha_i \in V\}$可数,记作:$\beta =\sum\limits_{i=1}^{\infty} \alpha_i$,否则记作$\beta =\sum\limits_{i=1}^{n} \alpha_i$,其中$n$是向量组的元素数

				由向量组$S$张成的全体线性组合形成的子空间记作:$span_F (S)$(或$L(S)$,$\langle S \rangle $)。特别地,如果$S$不可数,我们记$span_F(S)$为由$S$的有限子集张成的全体向量的集合
			\end{defn}
			我们不考虑不可数的向量组张成的线性组合,因为不可数的向量的求和的定义通常是不良好的。

			此后除非特别说明,我们默认我们所说的向量组指的是至多可数的向量组。
		\subsection{向量组的线性相关性和线性无关性}
			我们观察到一个现象:向量组张成线性子空间时,并非每一个向量都有贡献,有些向量是不必要的。就像下图中,尽管三个向量张成了整个平面,但是其实任取其中之二,就足以张成整个平面。

			% 设置3D视图角度
			\tdplotsetmaincoords{75}{125}
			\begin{tikzpicture}[tdplot_main_coords, scale=1.5]

			% 坐标原点
			\coordinate (A) at (0,0,0);

			% 向量终点坐标
			\coordinate (B) at (3.66,4.16,0);
			\coordinate (C) at (3,-2,0);
			\coordinate (D) at (4.5,0.7,0);

			% 背景平面(淡蓝色)
			\fill[cyan!15, opacity=0.7] (-5,-5,0) -- (5,-5,0) -- (5,5,0) -- (-5,5,0) -- cycle;

			% 向量
			\draw[->, thick, black] (A) -- (B);
			\draw[->, thick, black] (A) -- (C);
			\draw[->, thick, black] (A) -- (D);

			% 标签(无圆点)
			\draw (B) node[anchor=east, blue] {B};
			\draw (C) node[anchor=west, blue] {C};
			\draw (D) node[anchor=south, blue] {D};

			% 点
			\filldraw[gray] (A) circle (2pt) node[anchor=north east] {A};

			\end{tikzpicture}
			我们会自然地认为:如果每一个向量都和其他向量的“方向”不一样,就不会有不必要的向量,或者说,不应该可以在赋予非零系数时,张成零向量

			此时,向量组被称为是“线性无关的”。
			\begin{defn}{线性无关的向量组}{}
				向量组$\{\alpha_x \in V|x \in A\}$如果满足:

				$\sum\limits_{x \in A} k_x \alpha_x = \mathbf{0} \Rightarrow \forall x \in A,k_x = 0$

				那么我们称$\{\alpha_x \in V|x \in A\}$是一个线性无关的向量组
			\end{defn}
			如果一个向量组不是线性无关的,那么我们称它是线性相关的,我们可以写出以下定义
			\begin{defn}{线性相关的向量组}{}
				向量组$\{\alpha_x \in V|x \in A\}$如果满足:

				存在不全为零的一组系数$k_x,x \in A$,使得$\sum\limits_{x \in A} k_x \alpha_x = \mathbf{0}$

				那么我们称$\{\alpha_x \in V|x \in A\}$是一个线性相关关的向量组
			\end{defn}
		\subsection{线性相关和线性无关的性质}
		\begin{para}{0}
			\point{单个向量当且仅当它是零向量是线性相关}
				\begin{proposition}
					$\alpha $线性相关 $\Leftrightarrow \alpha =\mathbf{0}$
				\end{proposition}
				\begin{proof}
					先证充分性,如果$\alpha $线性相关,那么$\exists k \in F,k \neq 0,k\alpha =\mathbf{0} \Rightarrow \alpha =\mathbf{0}$

					必要性是显然的
				\end{proof}
			\point{如果一个部分组线性相关,整个向量组也是线性相关的}
				\begin{proposition}
					如果$\{\alpha_x \in V|x \in A\}$是一个向量组,它的一个部分组$\{\alpha_x \in V| x \in B\},B \subseteq A$线性相关,那么$\{\alpha_x \in V|x \in A\}$也是线性相关的
				\end{proposition}
				\begin{proof}
					因为$\{\alpha_x \in V| x \in B\}$线性相关,所以存在不全为零的系数$k_x,x \in B$,使得$\sum\limits_{x \in B} k_x \alpha_x = \mathbf{0}$

					我们可以将$k_x$扩展到$A$上,令$k_x = 0,x \in A-B$,那么$\sum\limits_{x \in A} k_x \alpha_x = \sum\limits_{x \in B} k_x \alpha_x=\mathbf{0}$,所以$\{\alpha_x \in V|x \in A\}$线性相关
				\end{proof}
				反过来,我们可以推论出,线性无关向量组的任意部分组也是线性无关的
				\begin{corollary}{线性无关向量组的任意部分组都线性无关}{}
					若向量组$\{\alpha_x \in V|x \in A\}$线性无关,那么它的任意一个部分组$\{\alpha_x \in V| x \in B\},B \subseteq A$也是线性无关的
				\end{corollary}
				\begin{proof}
					如果并非如此,那么至少有一个部分组线性相关,那么按照命题,整个向量组线性相关,这与假设矛盾,所以命题得证
				\end{proof}
				因此,一个显然的事实是,一个包含零向量的向量组是一定线性相关的
				\begin{proposition}
					如果$ \mathbf{0} \in \{\alpha_x \in V|x \in A\}$,那么$\{\alpha_x \in V|x \in A\}$线性相关
				\end{proposition}
				\begin{proof}
					显然
				\end{proof}
			\point{向量组线性相关的充要条件是:存在一个向量可以由其他向量线性表出}
				\begin{proposition}
					向量组线性相关$\Leftrightarrow$向量组中存在一个向量可以由其他向量线性表出
				\end{proposition}
				\begin{proof}
					先证充分性。如果$\{\alpha_x \in V|x \in A\}$线性相关,那么存在不全为零的系数$k_x,x \in A$,使得$\sum\limits_{x \in A} k_x \alpha_x = \mathbf{0}$

					取其中任意一个不为零的系数$k_a \neq 0$,那么有$\alpha_a =-\frac{1}{k_a}\sum\limits_{x \in A-\{a\}} k_x \alpha_x$,充分性得证

					接下来证明必要性,如果$\alpha_a =\sum\limits_{x \in A-\{a\}} k_x \alpha_x$,那么我们只需令$k_a=-1$

					那么就有$\sum\limits_{x \in A} k_x \alpha_x = \mathbf{0}$
				\end{proof}
			\point{能唯一地表出一个向量的向量组线性无关}
				\begin{proposition}
					$\beta \in V$可以被$\{\alpha_x \in V| x\in A\}$唯一地线性表出$\Leftrightarrow\{\alpha_x \in V| x\in A\}$线性无关
				\end{proposition}
				\begin{proof}
					先证充分性。

					这其实是显然的,因为如果$\{\alpha_x \in V| x\in A\}$线性相关,那么有一个线性组合$\sum\limits_{x \in A} k_x \alpha_x = \mathbf{0}$并且$k_x$不全为零

					假设$\beta =\sum\limits_{x \in A} l_x \alpha_x$,那么也有$\sum\limits_{x \in A} (k_x + l_x) \alpha_x$,这与假设矛盾

					再证必要性。

					不妨假设$\beta =\sum\limits_{x \in A} l_x \alpha_x=\sum\limits_{x \in A} k_x \alpha_x \exists x \in A,k_x \neq l_x$

					那么$\sum\limits_{x \in A} (l_x-k_x) \alpha_x = \mathbf{0}$,又因为$\{\alpha_x \in V| x\in A\}$线性无关,所以$l_x-k_x=0$。因此表出方式是唯一的,命题得证
				\end{proof}
			\point{向线性无关向量组中加入一个向量,如果变得线性相关,那么这个向量可以由其他向量线性表出}
				\begin{proposition}
					如果$\{\alpha _x \in V| x \in A\}$线性无关,$\{\beta ,\alpha_x \in V| x\in A\}$线性相关

					那么$\beta $可由$\{\alpha_x \in V| x\in A\}$线性表出
				\end{proposition}
				\begin{proof}
					因为$\{\beta ,\alpha_x \in V| x\in A\}$线性相关,所以存在不全为零的系数$k_x,l\in F,x \in A$,使得$\sum\limits_{x \in A} k_x \alpha_x + l \beta = \mathbf{0}$

					我们只需证$l \neq 0$。我们不妨假设$l=0$

					那么$\sum\limits_{x \in A} k_x \alpha_x = \mathbf{0}$。但是$\{\alpha_x \in V| x\in A\}$线性无关,所以$\forall x \in A,k_x =0$,这与系数不全为零的假设矛盾

					那么有$l \neq 0$,于是$\beta  =-\frac{1}{l}\sum\limits_{x \in A} k_x \alpha_x$,命题得证
				\end{proof}
		\end{para}
	\section{极大线性无关向量组、向量组的秩}
		前面我们讨论了线性无关向量组,它可以说是“所有元素都有贡献”的,接下来我们讨论,如果把一个向量集剔除至恰好包含所有“有贡献”的向量,这种向量组有什么性质

		我们的一个想法是,如果剔除至向量组本身依旧线性无关,但是加入任意向量就变得线性相关,那么这个向量组就是我们想要的“极大线性无关向量组”
		\subsection{极大线性无关向量组}
			\begin{defn}{极大线性无关向量组}{}
				设$\{\alpha_x \in V| x \in A\}$是一个向量组

				如果它的一个部分组$\{\alpha_x \in V| x \in B\},B\subseteq A$线性无关,并且$\forall \beta \in V,\{\beta ,\alpha_x| x \in B\}$线性相关

				那么我们称$\{\alpha_x \in V| x \in B\}$是$\{\alpha_x \in V| x \in A\}$的一个极大线性无关向量组
			\end{defn}
			我们直观地认为,每一个极大线性无关向量组应该具备相同的元素数(或势),就像在三维空间中,我们可以选取不同的坐标系,但是每个坐标系都恰好三个坐标轴
		\subsection{向量组的等价}
			我们在向量组上定义以下等价关系:
			\begin{defn}{向量组的线性表出}{}
				如果向量组$\{\alpha_x \in V| x \in A\}$中的任意向量可由$\{\beta_x \in V| x \in B\}$线性表出

				那么我们称$\{\alpha_x \in V| x \in A\}$可由$\{\beta _x \in V| x \in B\}$线性表出
			\end{defn}
			\begin{defn}{向量组的等价}{}
				如果向量组$\{\alpha_x \in V| x \in A\}$可由$\{\beta_x \in V| x \in B\}$线性表出,并且$\{\beta_x \in V| x \in B\}$可由$\{\alpha_x \in V| x \in A\}$线性表出

				那么我们称$\{\alpha_x \in V| x \in A\}$和$\{\beta_x \in V| x \in B\}$是等价的,记作:$\{\alpha_x \in V| x \in A\} \cong \{\beta_x \in V| x \in B\}$
			\end{defn}
			我们首先验证它的确是一个等价关系
			\begin{proposition}
				向量组的等价是一个等价关系
			\end{proposition}
			\begin{proof}
				自反性是显然的,因为向量组中的任意向量$\alpha_x$可以由它自己线性表出;

				对称性由定义是显然的;

				最后证明传递性,假设$\{\alpha_x \in V| x \in A\} \cong \{\beta _x \in V| x \in B\},\{\beta _x \in V| x \in B\}\cong\{\gamma _x \in V| x \in C\}$

				设$\alpha_x = \sum\limits_{y \in B} k_{xy} \beta_y,\beta_y = \sum\limits_{z \in C} l_{yz} \gamma_z$

				那么$\alpha_x = \sum\limits_{y \in B} k_{xy} \left(\sum\limits_{z \in C} l_{yz} \gamma_z\right)=\sum\limits_{y \in B,z\in C} k_{xy}l_{yz} \gamma_z$

				因此$\{\alpha_x \in V| x \in A\}$可由$\{\alpha_x \in V| x \in A\}$线性表出,同理可证$\{\gamma_x \in V| x \in C\}$可由$\{\alpha_x \in V| x \in A\}$线性表出
				
				所以$\{\alpha_x \in V| x \in A\} \cong \{\gamma_x \in V| x \in C\}$,传递性得证。于是命题得证
			\end{proof}
			接下来讨论一些向量组等价的性质:
			\begin{para}{0}
				\point{任何向量组和它的极大线性无关向量组等价}
					\begin{proposition}
						任何向量组和它的极大线性无关向量组等价
					\end{proposition}
					\begin{proof}
						极大线性无关向量组显然可以由原向量组线性表出;

						而按照极大线性无关向量组的定义,向其中加入任意一个向量,向量组会变得线性相关,于是,原向量组中的任意一个向量都可以由极大线性无关向量组线性表出。命题得证
					\end{proof}
				\point{向量组的两个极大线性无关向量组等价}
					\begin{proposition}
						向量组的两个极大线性无关向量组等价
					\end{proposition}
					\begin{proof}
						只需利用上面的性质和等价的传递性即可完成证明。
					\end{proof}
				\point{一个向量组如果可以由一个势比它小的向量组线性表出,那么它线性相关}
					在给出命题及其证明前,我们先证明一个显然的引理
					\begin{lemma}{}{}
						向量组$\{\alpha_1,\cdots,\alpha_n\}$如果线性相关,
						
						那么一定$\exists k,\text{s.t. }\alpha_k$可由$\alpha_1,\cdots,\alpha_{k-1}$线性表出,并且$\text{span}(\alpha_1,\cdots,\alpha_{i-1},\alpha_{i+1},\cdots,\alpha_n)=\text{span}(\alpha_1,\cdots,\alpha_n)$
					\end{lemma}
					接下来给出命题
					\begin{proof}
						因为$\{\alpha_1,\cdots,\alpha_n\}$线性相关,所以一定$\exists k_1,\cdots,k_n,k_1 \alpha_1+\cdots+k_n \alpha_n = \mathbf{0}$

						只需选取满足$k_i \neq 0$的最大的下标$p$,那么有$\alpha_1+\cdots+k_p \alpha_p = \mathbf{0}$

						于是$\alpha_p = -\frac{1}{k_p} \sum\limits_{i=1}^{p-1} \alpha_i$

						此时因为$\alpha_p$可由前面的向量线性表出,因此去除其不会对生成的子空间有任何影响,命题得证。
					\end{proof}
					\begin{proposition}
						如果向量组$\{\alpha_x \in V| x \in A\}$可由$\{\beta_x \in V| x \in B\}$线性表出,并且$\text{Card } A > \text{Card }B$

						那么$\{\alpha_x \in V| x \in A\}$线性相关
					\end{proposition}
					\begin{proof}
						先考虑两者都是有限向量组的情况

						我们可以证明:如果$\{\alpha_1,\cdots,\alpha_m\}$可由$\{\beta_1,\cdots,\beta_n\}$线性表出,并且$m > n$,那么$\{\alpha_1,\cdots,\alpha_m\}$线性相关

						记$W=\text{span}(\beta _1, \dots, \beta _n)$。

						我们采取以下方法证明:用$\alpha_i$逐步替换$\beta_j$,但是保持生成的子空间不变。最后,我们一定会发现,$\beta_j$已经被全部替换,但是此时依旧有$\alpha_i$未曾被加入向量组过。这就说明了$\{\alpha_1,\cdots,\alpha_m\}$中有向量可以被其它向量线性表出,即$\{\alpha_1,\cdots,\alpha_m\}$线性相关
						
						也就是说,我们希望构造一系列向量组 $B_0, B_1, \dots, B_n$,使得对于 $j=0, 1, \dots, n$,向量组 $B_j$ 满足:
						\begin{enumerate}
							\item $B_j$ 的长度为 $n$。
							\item $\text{span}(B_j) = W$。
							\item $B_j$ 包含向量 $\alpha _1, \dots, \alpha _j$ 以及从原始向量组 $\{\beta _1, \dots, \beta _n\}$ 中剩下的 $n-j$ 个向量。
						\end{enumerate}
						
						我们首先取$B_0 = \{\beta _1, \dots, \beta _n\}$

						我们如此由$B_{j-1}$迭代出向量组$B_j$:

						考虑向量组 $\alpha_j \cup B_{j-1}$(并且不失一般性地将$\alpha_j$置于向量组的首位)。由于$\alpha_j \in W$,所以一定有$\text{span}(\alpha_j \cup B_{j-1})=W$
						
						由于 $\text{span}(B_{j-1})=W$,向量 $\alpha _j$ 在 $B_{j-1}$ 生成的子空间中,因此向量组 $\alpha _j\cup B_{j-1}$ 是线性相关的。

						根据前面的引理,线性相关的向量组 $\alpha_j\cup B_{j-1}$ 中存在一个向量,它是其前面向量的线性组合。设这个向量为 $w$。
						
						如果 $w = \alpha_j$,则 $\alpha_j$ 是空向量组的线性组合,意味着 $\alpha_j = \mathbf{0}$,于是该向量组是线性相关的,证明结束。

						接下来考虑$w$是 $B_{j-1}$ 中的一个向量。如果$w \in \{\alpha_1,\cdots,\alpha_{j-1}\}$,那么它可以被$\{\alpha_1,\cdots,\alpha_m\}$中的向量线性表出,那么向量组线性相关,证明结束。

						因此只需考虑这个向量 $w$ 是 $B_{j-1}$ 中那些来自 $\{\beta _1, \dots, \beta _n\}$ 的向量

						由于 $w$ 可以由向量组中前面的向量线性表出,我们可以从 $\alpha_j \cup B_{j-1}$ 中移除 $w$,得到新的向量组 $B_j = (\alpha_j \cup B_{j-1})-w$。这个新的向量组 $B_j$ 仍然生成子空间 $W$
						
						这样,我们就完成了向量组序列$B_0, B_1, \dots, B_n$的构造。
						
						此时,我们只需注意到,$\text{span}(B_n)=\text{span}(\alpha_1,\cdots,\alpha _n)=W,\alpha_{n+1} \in W$。于是$\alpha_{n+1}$可由$\alpha_1,\cdots,\alpha_n$线性表出,于是$\{\alpha_1,\cdots,\alpha_m\}$线性相关,证明完毕。

						最后,我们考虑$\text{Card} A = \N,\text{Card} B < \N$的情况。此时,一定能找到$\{\alpha_x \in V | x \in A\}$的一个有限部分组线性相关,于是整个向量组就线性相关了。

						于是命题得证。
					\end{proof}
				\point{向量组的两个极大线性无关向量组等势}
					\begin{proposition}
						向量组的两个极大线性无关向量组等势
					\end{proposition}
					\begin{proof}
						设有两个极大线性无关向量组$\{\alpha_x \in V| x\in A\},\{\beta_x \in V| x \in B\}$

						根据前面的性质,他们相互等价,因此可以相互线性表出。所以一定有$\text{Card }A \leqslant \text{Card }B,\text{Card }B \text{Card }A$

						那么一定有$\text{Card }A = \text{Card }B$,所以命题得证
					\end{proof}
			\end{para}
		\subsection{向量组的秩}
			在完成了前面的准备工作后,我们可以定义向量组的秩了
			\begin{defn}{向量组的秩}{}
				向量组$G=\{\alpha_x \in V| x \in A\}$的极大线性无关向量组的势称为向量组的秩,记作:$rank(G)$
			\end{defn}
			我们前面的命题已经保证了,无论如何选取极大线性无关组,它的势都相同,因而秩也是相同的。
			
			接下来讨论它的性质
			\begin{para}{0}
				\point{}
					\begin{proposition}
						向量组$\{\alpha_1,\cdots,\alpha_n\}$线性无关$\Leftrightarrow rank(\alpha_1,\cdots,\alpha_n)=n$
					\end{proposition}
					\begin{proof}
						先证充分性,如果向量组线性无关,那么它自身即是自己的一个极大线性无关向量组,所以$rank(\alpha_1,\cdots,\alpha_n)=n$

						再证必要性。如果$rank(\alpha_1,\cdots,\alpha_n)=n$,那么它仅有一个极大线性无关向量组,即向量组本身,所以它线性无关,命题得证
					\end{proof}
					值得注意的是,这个性质只对有限向量组成立,因为对于可数的向量组,可以存在一个可数的极大线性无关组,但是他的秩依旧和整个向量组一致。
				\point{}
					\begin{proposition}
						如果向量组$A=\{\alpha_x \in V| x\in A\}$可由$B=\{\beta_x \in V|x \in B\}$线性表出

						那么$rank(A) \leqslant rank(B)$
					\end{proposition}
					\begin{proof}
						选取两者的任意一个极大线性无关向量组$C \subseteq A,D\subseteq B$

						因为$A \cong C,B \cong D$,所以$C$可由$D$线性表出

						由$C$的线性无关性可知,$rank(A) = \text{Card }C \leqslant \text{Card }D = rank(B)$
					\end{proof}
			\end{para}
	\section{基、线性空间的维数}
		上一节中,我们提出了极大线性无关向量组的概念。

		从线性子空间的角度看,极大线性无关组是张成子空间的最小的向量组;

		本节中,我们尝试进一步地,让极大线性无关组张成整个线性空间。这样的极大线性无关组的存在性并不是平凡的,因为我们甚至不知道是否存在一个向量组可以张成整个空间
		\subsection{基的定义}
			我们定义两种不同的基,后续我们会看到为何我们需要两种基
			\begin{defn}{Harmel基}{}
				设$V$是一个线性空间,它的一个子集$B \subseteq V$如果满足:
				\begin{enumerate}
					\item $B$的任意有限子集线性无关
					\item $V$中的任意向量可由$B$的一个有限子集线性表出
				\end{enumerate}
				那么我们称$B$是$V$的一个Harmel基

				我们特别规定:$\{\mathbf{0}\}$的基是$\emptyset$
			\end{defn}
			\begin{defn}{Schauder基}{}
				设$V$是一个线性空间,它的一个子集$B \subseteq V$如果满足:
				\begin{enumerate}
					\item $B$线性无关
					\item $V$中的任意向量可由$B$线性表出
				\end{enumerate}
				那么我们称$B$是$V$的一个Schauder基

				我们特别规定:$\{\mathbf{0}\}$的基是$\emptyset$
			\end{defn}
			可以看出来,这两种基的根本区别是:Schauder基允许无限个向量来逼近一个向量,而Harmel基要求向量必须由有限个向量表出
			
			我们接下来将看到这个区别的重要影响:即两种基的存在性是不同的
		\subsection{Harmel基的存在性}
			我们将证明:Harmel基总是存在的
			\begin{them}{Harmel基的存在性}{}
				设$V$是一个线性空间,那么$V$中一定存在一个Harmel基
			\end{them}
			\begin{proof}
				记$H$为$V$中所有满足全部有限子集都线性无关的子集的集合。

				我们定义$H$上的一个序关系:$A,B \in H,A \leqslant B \Leftrightarrow A \subseteq B$,显然它的确是一个序,因而$H$成为一个偏序集

				接下来证明,对于任意一个$H$中的链$C$,$\bigcup\limits_{P \in C} P$正是$C$的一个上界

				先证明$\bigcup\limits_{P \in C} P$是$H$中的一个元素。

				取$\bigcup\limits_{P \in C} P$的一个有限子集$\alpha_1,\cdots,\alpha_s$

				事实上,$\alpha_1,\cdots,\alpha_s$一定属于某个$K \in C$(因为链是按照相互包含的关系形成的),因此$\alpha_1,\cdots,\alpha_s$线性无关。

				而$\forall K \in C,K \leqslant \bigcup_{P\in C} P$是显然的。于是$H$中的任意一个链有上界

				那么,依据Zorn引理,$H$中一定存在一个极大元素$B$。我们接下来证明,$B$是$V$的一个Harmel基
				
				首先,由$B$的构造方式,它的任意有限子集线性无关是显然的

				我们证明任意一个向量$\alpha \in V$都可以被$B$的有限子集线性表出

				如果不然,那么任意的$K \cup \{\alpha \} \subseteq B\cup \{\alpha \}$,$K\cup \{\alpha \}$都是线性无关的(利用命题1.2.6)
				
				所以$B \cup \{\alpha \} \in H$,于是,$B \leqslant B\cup \{\alpha \}$,这与$B$是$H$的极大元素矛盾。

				于是命题得证。
			\end{proof}
			在这个证明中,我们使用了Zorn引理,如果我们不承认选择公理,Harmel基的存在性就不再成立。

			事实上,线性空间的Schauder基是不一定存在的,但是它的反例非常复杂,此处我们简单讨论一下为何我们不能按照证明Harmel基存在性的方式来证明Schauder基的存在性

			前面的证明中,有一个容易被忽视的地方:为何任意一个链都有上界。在Harmel基的定义下,这是可以成立的,因为Harmel基要求的是有限子集,从而可以被链上的一个元素包含

			但是当考虑Schauder基,它要求全体向量一起作用,此时,链的并就不一定有这样的性质了,比如考虑以下例子:
			\begin{example}
				考虑平方可和空间$l^2 = \{(a_1,\cdots,a_n,\cdots)| \sum\limits_{i=1}^{\infty} |a_i|^2 < + \infty\}$

				并定义其中的度量为:$d\left((a_1,\cdots,a_n,\cdots),(b_1,\cdots,b_n,\cdots)\right)=\sqrt{\sum\limits_{i=1}^{\infty} (a_i-b_i)^2}$
				
				取它上面的一个向量组$\{e_i\},e_n = (0,\cdots,1,\cdots)$,其中$1$处于第n个位置

				我们构造以下链$\{S_k\}$:

				$S_1 = \{e_1\},S_{n+1}=S_n \cup \{e_{n+1}\}$

				证明:$\forall k,S_k$线性无关,但是$\bigcup_{i=1}^{\infty} S_i$线性相关
			\end{example}
			\begin{proof}
				首先,$\forall k,S_k$线性无关。这是显然的,因为其实$S_k$全部是有限集

				设$c_n = \begin{cases}
					\frac{1}{2k+1},n=2k+1,k \in \N \\
					\frac{1}{2k},n=2k,k\in \N \\
				\end{cases}$

				注意到:$\sum\limits_{i=1}^{\infty} c_i e_i = \mathbf{0}$

				因为:$\forall \varepsilon >0,\exists n \in N_+,\text{s.t.}$
				
				$d\left(\sum\limits_{i=1}^{n} c_i e_i,\mathbf{0}\right)=\sum\limits_{i=1}^{\infty} c_n < \varepsilon $

				此时系数显然不全为零,所以$\bigcup_{i=1}^{\infty} S_i$线性相关
			\end{proof}
			所以,论证至此就失效了,我们并不能同样地证明Schauder基的存在性。

			此后除非特别说明,我们默认我们所说的基指的是Hamel基
		\subsection{线性空间的维数}
			我们常常将基的势称为维数。自然地,因为Schauder基并非一直存在,我们一般选取Hamel基来定义维数。

			但是,目前还有一个问题:我们如何保证选取不同基时,维数依旧不变?也就是说,为何两个基一定等势?

			对于有限基,这个问题是容易解决的,但是对于无限基,特别是不可数基,这个问题非常难以解决。接下来,我们将花费大量篇幅论证这一点。

			\begin{lemma}{Steinitz替换引理}{}
				设$V$是一个线性空间,$S,T \subseteq V$,并且$S$的任意有限子集线性无关,$V=\operatorname{span}(T)$

				那么我们断言:$|S| \leqslant |T|$
			\end{lemma}
			\begin{proof}
				我们希望构造一个映射$S \ni S \mapsto C_s \subseteq T$,但是此时我们发现我们无从下手。

				我们退而求其次,构造$S$的子集和$T$的子集之间的映射,同时用一个偏序集逼近$T$

				为了保证偏序集的每一个链都有上界(后面我们会发现为何我们采取这样的奇怪构造),我们如下构造

				设$\Omega\;:=\;\bigl\{\,(C,f)\mid C\subset T,\;f:C\to S \text{ 单射且 }(S-f(C))\cup C\text{ 线性无关}\,\bigr\}.$
				
				并在它上面定义序关系:$(C_1,f_1)\preceq(C_2,f_2) \iff C_1\subset C_2\text{ 且 }f_2|_{C_1}=f_1$

				我们先验证$\Omega$中的每个链的确有上界

				设 $\mathcal C=\{(C_\alpha,f_\alpha)\}_{\alpha\in\Lambda}$ 是 $\Omega$ 中的全序链

				设$C\;:=\;\bigcup_{\alpha\in\Lambda}C_\alpha,f\;:=\;\bigcup_{\alpha\in\Lambda}f_\alpha$

				此处映射的并指的是将全部元素的映射关系一同合并(由$\Omega$的定义,此次操作的定义显然是良好的)

				那么,$f:C\to S$的单射性是显然的;
				
				而因为$\forall \alpha ,(S-f(C))\cup C_\alpha \subseteq (S-f(C_\alpha )\cup C_\alpha )$线性无关,因此$(S-f(C))\cup C$ 线性无关。
				
				于是 $(C,f)$ 是链的上界。

				故 $\Omega$ 满足 Zorn 引理的条件,我们记其极大元为 $(C,f)$。

				我们接下来证明$S\subset f(C)\cup C$。如果这个成立,我们可以立即用集合的势的性质推出结论。
				
				若不然,那一定可以取$s\in S- \bigl(f(C)\cup C\bigr)$
				
				记$S'\;:=\;(S-f(C))\cup C$。因为$(C,f)$是极大元,因此$S'$ 线性无关,所以$S'-\{s\}$也是线性无关的。
				
				我们说,一定存在一个$t \in T-C,t \notin \operatorname{span}\bigl(S'-\{s\}\bigr)$
				
				如果不是这样,即$T-C\subset \operatorname{span}\bigl(S-\{s\}\bigr)$

				那么$s\in\operatorname{span}(T)=\operatorname{span}(T- C)+\operatorname{span}(C)\subset \operatorname{span}\bigl(S'-\{s\}\bigr)$,
				
				那么$s$可由$S^\prime{}-{s}$线性表出,那么$S'$就是线性相关的,与 $S'$ 的线性无关性矛盾。
				
				因此可取$t\in T- C,t\notin\operatorname{span}\bigl(S'-\{s\}\bigr)$
				
				此时,我们定义:$g:C\cup\{t\}\longrightarrow S,\quad g|_C=f,\;g(t)=s$

				而$(S- g(C\cup\{t\}))\cup (C\cup\{t\})=(S'-\{s\})\cup\{t\}$ 仍线性无关(因为$S'-{s}$线性无关且$t$不能被$S'-{s}$线性表出),
				
				故 $(C\cup\{t\},g)\in\Omega$,与 $(C,f)$ 极大矛盾。于是$S\subset f(C)\cup C$

				至此,我们可以推出最终的结论了,由 $f$ 单射可知;
				
				$|S|\leqslant |f(C)|+|C|= 2|C|\leqslant 2|T|.$
			\end{proof}
			进而我们知道,线性空间的两个基是等势的。
			\begin{lemma}{线性空间的两个基等势}{}
				设$V$是一个线性空间,$B_1,B_2$是$V$的两个基。

				那么:$|B_1|=|B_2|$
			\end{lemma}
			\begin{proof}
				这是显然的,因为$B_1,B_2$均满足有限子集线性无关和张成$V$

				所以一定有$|B_1| \leqslant |B_2|,|B_2| \leqslant |B_1|$,于是有$|B_1|=|B_2|$,命题得证
			\end{proof}
			证明了这个引理后,我们就可以给出维数的定义了
			\begin{defn}{线性空间的维数}{}
				设$V$是一个$F$上的线性空间,$B$是$V$的一个Hamel基

				我们定义:
				\begin{equation}
					dim_F V = \text{Card }B
				\end{equation}
				称为$V$的维数,简记作$dim V$

				如果$dim_F V < +\infty$,我们称$V$是一个有限维线性空间,反之称为无限维线性空间
			\end{defn}
			可以看出来,线性空间的维数并非只有有限和无限两类,比如以下例子分别给出了$\aleph_0$和$\aleph_1$维的线性空间
			\begin{example}
				$\dim_K K[x] = \aleph_0$
			\end{example}
			\begin{proof}
				显然,$\{x^n | n \in \N\}$是$K[x]$的一个基

				因此,$\dim_K K[x] = |\{x^n | n \in \N\}| = \aleph_0$
			\end{proof}
			\begin{example}
				$\dim_{\R} \{f:{\R} \rightarrow {\R}| {\card} supp(f) < {\aleph_0}\} = \aleph_1$
			\end{example}
			\begin{proof}
				我们指出,$\left\{ f_x| x \in \R,f_x(y) = \begin{cases}
				1,y=x \\
				0,y \neq x
				\end{cases}\right\}$是$\{f:\R \rightarrow \R| \card supp(f) < \aleph_0\}$的一个基

				因为:首先这个集合的任意有限子集线性无关,其次空间的任意向量可以被这个集合的一个有限子集线性表出(因为空间内的函数的支撑集都有限)

				因此$\dim_{\R} \{f:{\R} \rightarrow {\R}| {\card} supp(f) < {\aleph_0}\} = \card \left\{ f_x| x \in \R,f_x(y) = \begin{cases}
				1,y=x \\
				0,y \neq x
				\end{cases}\right\} = \aleph_1$
			\end{proof}
		\subsection{线性空间的维数的性质}
			现在讨论一些维数的性质
			\begin{para}{0}
				\point{}
					\begin{proposition}
						设$V$是一个线性空间,集合$S \subseteq V$如果$|S| > dim V$

						那么$S$至少有一个有限子集线性相关。
					\end{proposition}
					\begin{proof}
						这其实是前面的Steinitz替换引理的直接结论

						任取$V$的一个基$B$,由基的定义和引理可知,如果$S$线性无关,那么一定有$|S| \leqslant |B|$,与前提矛盾。

						于是命题得证。
					\end{proof}
				\point{}
					\begin{proposition}
						线性空间$V$的一个任意一个子集$B$,如果它的任意有限子集都线性无关,那么它可以扩充成一个Harmel基
					\end{proposition}
					\begin{proof}
						我们只需要仿照证明Harmel基存在性的方式,构造一个偏序集$H=\{S \supseteq B| S\text{的任意有限子集线性无关}\}$

						同样地,$H$一定存在一个极大元$B' \supseteq B$,并且容易验证它的确是一个基。于是命题得证
					\end{proof}
				\point{}
					\begin{proposition}
						设$V$是一个线性空间,$W \subseteq V$是一个线性子空间,那么有:$dim W \leqslant dim V$
					\end{proposition}
					\begin{proof}
						取$W$的一个基$B$。依定义,$B$的任意子集在$W$中线性无关,于是它的任意子集在$V$中也线性无关
						
						那么按照我们前面证明的命题,$B$可以扩充成$V$的一个基$B'$

						那么有$|B| \leqslant |B'|$,于是$dim W \leqslant dim V$,命题得证
					\end{proof}
				\point{}
					\begin{proposition}
						设$V$是一个线性空间,它的一个子集$S \subseteq V$,那么:

						如果$S$是$V$的一个基,那么$S$的任意有限子集线性无关,并且$\forall \alpha ,S \cup \{\alpha \}$存在一个有限线性相关子集;

						如果$V \neq \{\mathbf{0}\}$,$S$的任意有限子集线性无关,并且$\forall \alpha ,S \cup \{\alpha \}$存在一个有限线性相关子集,那么$S$是$V$的一个基
					\end{proposition}
					\begin{proof}
						这是显然的,因为:

						对于第一条结论,$\forall \alpha \in V,\exists B \subseteq S,|B| < \aleph_0,\alpha \in \text{span}(B)$,此时一定有$B \cup \{\alpha \}$线性相关和线性无关;

						对于第二条结论,$\forall \alpha \in V,\exists B \subseteq S,|B| < \aleph_0,S \cup \{\alpha \}$线性相关,而又$S$线性无关,所以$\alpha $可由$B$线性表出,因此$S$是一个基

						于是命题得证。
					\end{proof}
					如果$S$的任意有限子集线性无关,并且$\forall \alpha ,S \cup \{\alpha \}$存在一个有限线性相关子集,此时我们也说$S$是代数意义下$V$的一个极大线性无关集
				
					接下来考虑几个只适用于有限维的结论	
				\point{}
					\begin{proposition}
						设$V$是一个线性空间,$dim V=n$,那么$V$中的任意$n$个线性无关向量都可以构成$V$的一个基
					\end{proposition}
					\begin{proof}
						这是显然的,因为这$n$个向量可以扩充成一个基,但是因为维数就是$n$,如果添加任意向量都会导致线性相关,因此只有可能它本来就是一个基
					\end{proof}
				\point{}
					\begin{proposition}
						设$V$是一个线性空间,$dim V=n$,如果$\forall \alpha \in V,\alpha \in \text{span}(\gamma_,\cdots,\gamma_n)$,

						那么$\{\gamma_1,\cdots,\gamma_n\}$是$V$的一个基
					\end{proposition}
					\begin{proof}
						任取$V$的一个基$\{\beta_1,\cdots,\beta_n\}$

						于是依命题条件,$\{\beta_1,\cdots,\beta_n\}$可由$\{\gamma_1,\cdots,\gamma_n\}$线性表出

						那么有$rank(\beta_1,\cdots,\beta_n) \leqslant rank(\gamma_1,\cdots,\gamma_n) \leqslant n$

						于是$rank(\gamma_1,\cdots,\gamma_n)=n$,运用前面证明的命题后命题得证。
					\end{proof}
				\point{}
					\begin{proposition}
						设$S=\{\alpha_x \in V| x \in A,|A| \leqslant \aleph_0\}$是一个至多可数的向量组

						那么有:$dim \text{ span}(S)=rank(S)$
					\end{proposition}
					\begin{proposition}
						只需取$S$的一个极大线性无关组$B \subseteq S$。由极大线性无关组的性质,$\text{span}(S)=\text{span}(B)$

						于是有$dim \text{ span }(S)=dim \text{ span }(B)=|B|=rank(B)=rank(S)$,命题得证。
					\end{proposition}
				\point{}
					\begin{proposition}
						设$S=\{\alpha_x \in V| x \in A,|A| \leqslant \aleph_0\},G=\{\alpha_x \in V| x \in B,|B| \leqslant \aleph_0\}$

						那么:$\text{span}(S)=\text{span}(G) \Leftrightarrow S \cong G$
					\end{proposition}
					\begin{proof}
						充分性是显然的,因为如果生成的子空间相同,两个向量组一定可以相互线性表出

						必要性也是显然的,因为如果$S$可由$G$线性表出,一定有$\text{span}(S)\subseteq\text{span}(G)$,同理有$\text{span}(S)\supseteq\text{span}(G)$,于是命题得证。
					\end{proof}
			\end{para}
	\section{线性空间的同构}
		在前面的章节中,我们讨论了线性组合、线性空间的基和维数。

		本节开始,我们着手研究不同的空间之间的关系。本节我们考虑一个问题:不同的空间是否可以拥有相同的结构呢?我们应该如何刻画这个“相同”呢?

		这个问题的答案是肯定的。这种关系称为同构,它具有一系列极为良好的性质。
		\subsection{同构的定义}
			\begin{defn}{线性空间的同构}
				设$V_1,V_2$是两个$F$上的线性空间,如果存在双射$f:V_1 \rightarrow V_2$

				满足:$\forall k \in F,\alpha ,\beta \in V_1$

				$f(k\cdot\alpha )=k\cdot f(\alpha ),f(\alpha+\beta )=f(\alpha )+f(\beta )$

				那么我们称$V_1$和$V_2$同构,记作$V_1 \cong V_2$,并称$f$是$V_1$到$V_2$的同构映射
			\end{defn}
		\subsection{同构的性质}
			接下来讨论同构的一些性质
			\begin{para}{0}
				\point{}
					\begin{proposition}
						同构是一种等价关系,即:

						\begin{enumerate}
							\item $V_1 \cong V_1$
							\item $V_1 \cong V_2 \Rightarrow V_2 \cong V_1$
							\item $V_1 \cong V_2 ,V_2 \cong V_3 \Rightarrow V_1 \cong V_3$
						\end{enumerate}
					\end{proposition}
					\begin{proof}
						首先证明自反性,即$V_1 \cong V_1$。

						考虑恒等映射 $I: V_1 \to V_1$,定义为 $I(\alpha ) = \alpha $ 对于所有 $\alpha  \in V_1$。
							
						显然,$I$是一个双射

						$\forall \alpha_1, \alpha _2 \in V_1,k \in F$

						$I(\alpha _1 + \alpha _2) = \alpha _1 + \alpha _2 = I(\alpha _1) + I(\alpha _2),I(k \alpha _1) = k \alpha _1 = k I(\alpha _1)$

						于是自反性成立。

						接下来证明对称性,即如果 $V_1 \cong V_2$,那么 $V_2 \cong V_1$。

						假设 $V_1 \cong V_2$。这意味着存在一个线性同构 $T: V_1 \to V_2$。
						
						由于 $T$ 是同构映射,它一定是双射的,因此它的逆映射 $T^{-1}: V_2 \to V_1$ 存在,并且也是一个双射

						接下来证明 $T^{-1}$ 也是同构映射

						设 $\beta_1, \beta_2 \in V$。因为 $T$ 是满射,存在唯一的 $\alpha _1, \alpha _2 \in U$ 使得 $T(\alpha _1) = \beta _1$ 且 $T(\alpha _2) = \beta _2$。因此 $\alpha _1 = T^{-1}(\beta _1)$ 且 $\alpha _2 = T^{-1}(\beta _2)$。
						
						$T^{-1}(\beta_1 + \beta _2) = T^{-1}(T(\alpha _1) + T(\alpha _2))$

						$= T^{-1}(T(\alpha _1 + \alpha _2)) = \alpha _1 + \alpha _2 = T^{-1}(\beta _1) + T^{-1}(\beta _2)$。

						设 $\beta a  \in V_2$ 和纯量 $k \in F$。存在唯一的 $\alpha  \in U$ 使得 $T(\alpha ) = \beta $,因此 $T^{-1}(\beta )=\alpha $。
						
						$T^{-1}(k\beta ) = T^{-1}(k T(\alpha ))$

						$= T^{-1}(T(k \alpha )) = k \alpha  = k T^{-1}(\beta )$。
						
						于是对称性是成立的

						最后,我们证明传递性,即如果 $V_1 \cong V_2$ 且 $V_2 \cong V_3$,那么 $V_1 \cong V_3$。

						假设 $V_1 \cong V_2$ 且 $V_2 \cong V_3$。
						
						这意味着存在一个同构映射 $T_1: V_1 \to V_2$ 和一个同构映射 $T_2: V_2 \to V_3$。

						考虑复合映射 $T_2 \circ T_1: V_1 \to V_3$。因为$T_1$和$T_2$都是双射,因此他也是双射

						设 $\alpha _1, \alpha _2 \in V_1$ 和纯量 $k \in F$。
						
						$(T_2 \circ T_1)(\alpha _1 + \alpha _2) = T_2(T_1(\alpha _1 + \alpha _2))$

						$= T_2(T_1(\alpha _1) + T_1(\alpha _2))$

						$= (T_2 \circ T_1)(\alpha _1) + (T_2 \circ T_1)(\alpha _2)$。

						$(T_2 \circ T_1)(k \alpha _1) = T_2(T_1(k \alpha _1))$

						$= T_2(k T_1(\alpha _1))$

						$= k (T_2 \circ T_1)(\alpha _1)$。
						
						于是传递性成立,命题得证。
					\end{proof}
				\point{}
					\begin{proposition}
						如果$f:V_1 \to V_2$是一个同构映射,那么$f(\mathbf{0}_{V_1})=\mathbf{0}_{V_2}$
					\end{proposition}
					\begin{proof}
						$f(\mathbf{0}_{V_1})=f(0\cdot \mathbf{0}_{V_1})=0\cdot f(\mathbf{0}_{V_1})=\mathbf{0}_{V_2}$
					\end{proof}
				\point{}
					\begin{proposition}
						如果$f:V_1 \to V_2$是一个同构映射,那么$f(-\alpha )=-f(\alpha )$
					\end{proposition}
					\begin{proof}
						$f\left((-1)\cdot \alpha \right)=(-1)\cdot f(\alpha )=-f(\alpha )$
					\end{proof}
				\point{}
					\begin{proposition}
						如果$V_1 \cong V_2$,$f:V_1 \to V_2$是一个同构映射,$\alpha_1,\cdots,\alpha_n \in V_1$在$V_1$中线性相关

						那么$f(\alpha_1),\cdots,f(\alpha_n) \in V_2$在$V_2$中线性相关
					\end{proposition}
					\begin{proof}
						$k_1 \alpha_1 + \cdots + k_n \alpha_n = \mathbf{0}_{V_1}$

						$\Leftrightarrow f(k_1 \alpha_1 + \cdots + k_n \alpha_n) = f(\mathbf{0}_{V_1})=\mathbf{0}_{V_2}$

						$\Leftrightarrow k_1f(\alpha_1) + \cdots + k_nf(\alpha_n) = \mathbf{0}_{V_2}$

						因此,$\alpha_1,\cdots,\alpha_n$线性相关

						$\Leftrightarrow \exists k_1,\cdots,k_n,k_1 \alpha_1 + \cdots + k_n \alpha_n = \mathbf{0}_{V_1}$,其中$k_i$不全为零

						$\Leftrightarrow \exists k_1,\cdots,k_n,k_1 f(\alpha_1) + \cdots + k_n f(\alpha_n) = \mathbf{0}_{V_2}$,其中$k_i$不全为零
					
						$\Leftrightarrow f(\alpha_1),\cdots,f(\alpha_n)$线性相关
					\end{proof}
					其实,这个命题也指出了线性无关的向量组在同构映射作用下也是线性无关的,也就是说同构线性相关和线性无关性
				
					另外一个有趣的点是,其实在这个命题的证明中,我们并没有用到同构映射的满射性,只需要它是单射即可。
				\point{}
					\begin{proposition}
						设$V_1,V_2$是两个线性空间,$\dim V_1  = \dim V_2 = n < \aleph_0$

						映射$T:V_1 \to V_2$如果是单射,并且满足:
						
						$T(\alpha +\beta )=T(\alpha )+T(\beta ),T(k\alpha )=kT(\alpha )$

						那么$T$是一个同构映射
					\end{proposition}
					\begin{proof}
						取$V_1$的一组基$\{\alpha_1,\cdots,\alpha_n\}$,按基的定义,它一定线性无关

						运用前面命题的证明方法可得,$T(\alpha_1),\cdots,T(\alpha_n)$线性无关。

						而$\dim V_2 = n$,于是$\{T(\alpha_1),\cdots,T(\alpha_n)\}$是$V_2$的一个基

						那么,我们说,$T$一定也是满射。因为:

						$\forall \beta = k_1 T(\alpha_1)+\cdots+k_n T(\alpha_n) \in V_2$,注意到:

						$T(k_1 \alpha_1+\cdots+k_n \alpha_n) = k_1 T(\alpha_1)+\cdots+k_n T(\alpha_n) =\beta $,于是命题得证。
					\end{proof}
					这个命题指出,其实同构映射满射的条件也可以用维数有限且相等代替。值得注意的是,维数有限是必须的。
				\point{}
					\begin{proposition}
						如果$V_1 \cong V_2$,$f:V_1 \to V_2$是一个同构映射,$B_1 \subseteq V_1$是$V_1$的一个基

						那么$B_2 = f(B_1) \subseteq V_2$是$V_2$的一个基
					\end{proposition}
					\begin{proof}
						由前面的命题可知$B_2=f(B_1)$的线性无关性

						现在取$\forall \beta \in V_2$

						一定存在$\alpha \in V_1,f(\alpha )=\beta $

						因为$B_1$是$V_1$的基,所以一定能找到它的一个有限子集$\{\eta_1,\cdots,\eta_n\}$

						$\exists k_1,\cdots,k_n ,k_1 \eta_1 +\cdots + k_n \eta_n = \alpha $

						$\Rightarrow k_1 f(\eta_1) + \cdots, k_n f(\eta_n) = \beta $

						而$\{f(\eta_i)\} \subseteq B_2 = f(B_2)$

						因此$B_2$的确是$V_2$的基,命题得证
					\end{proof}
				\point{}
					\begin{proposition}
						$V_1 \cong V_2 \Leftrightarrow \dim V_1  = \dim V_2$
					\end{proposition}
					\begin{proof}
						充分性是显然的,因为任取$V_1$的一个基$B_1$和一个同构映射$f$,依照前面的命题,$f(B_1)$一定是$V_2$的一个基

						接下来证明必要性。分别取$V_1,V_2$的一个基$B_1,B_2$

						因为$\dim V_1 = \dim V_2$,一定有一个双射$g:B_1 \to B_2$

						我们构造以下映射:

						\begin{equation}
							f: V_1 \ni \alpha = \sum\limits_{i=1}^{n} k_i \alpha_i \mapsto \beta = \sum\limits_{i=1}^{n} k_i g(\alpha_i) \in V_2,\alpha_i \in B_1
						\end{equation}
						我们验证它的确是一个同构映射。

						首先,它是一个满射,因为$k_i$是可以在$F$中任意选取的,而$B_2=g(B_1)$是一个基。

						其次,它是一个单射,因为基的线性无关性,$k_i$一定是唯一的。

						我们最后验证可加性和齐次性

						我们不失一般性的认为接下来的$\alpha ,\beta $可由同一有限子集线性表出,因为如果表出它们的子集不同,我们只需要取并,随后在没有分量的位置赋予零的系数

						设$\alpha =\sum\limits_{i=1}^{n} k_i \eta_i ,\beta =\sum\limits_{i=1}^{n} l_i \eta_i,p \in F$

						$f(\alpha +\beta )=f(\sum\limits_{i=1}^{n} k_i \eta_i+\sum\limits_{i=1}^{n} l_i \eta_i)$

						$=f\left(\sum\limits_{i=1}^{n} (k_i+l_i) \eta_i\right)=\sum\limits_{i=1}^{n} (k_i+l_i) g(\eta_i)$

						$=\sum\limits_{i=1}^{n} k_i g(\eta_i)+\sum\limits_{i=1}^{n} l_i g(\eta_i)=f(\alpha )+f(\beta )$

						$f(p \alpha )=f(p\sum\limits_{i=1}^{n} k_i \eta_i)=f(\sum\limits_{i=1}^{n} pk_i \eta_i)$

						$=\sum\limits_{i=1}^{n} pk_i g(\eta_i)=p\sum\limits_{i=1}^{n} k_i g(\eta_i)=pf(\alpha )$

						于是命题得证
					\end{proof}
					它的一个常用推论是:
					\begin{corollary}{}{}
						如果$\dim_F V = n < \aleph_0$,那么$V \cong F^n$
					\end{corollary}
					\begin{proof}
						依照上面的命题,这是显然的
					\end{proof}
				\point{}
					\begin{proposition}
						设$V_1 \cong V_2$是同构的两个线性空间,$f:V_1 \to V_2$是一个同构映射
						
						那么,如果$W$是$V_1$的一个子空间,那么$f(W)$是$V_2$的一个子空间,并且$\dim W = \dim f(W)$
					\end{proposition}
					\begin{proof}
						首先,因为$\mathbf{0}_{V_1} \in W$,所以$f(\mathbf{0}_{V_1}) \in f(W)$,因此$f(W) \neq \emptyset$

						取$\forall f(\alpha ),f(\beta )\in f(W),k \in F$

						$f(\alpha )+f(\beta )=f(\alpha +\beta ) \in f(W),kf(\alpha )=f(k\alpha )\in f(W)$,因此$f(W)$是$V_2$的子空间

						接下来证明$\dim W = \dim f(W)$。设$g=f|_W : W \to f(W)$

						首先,$g$显然是一个双射,因为$f$就是一个双射

						注意到:$g(\alpha +\beta )=g(\alpha )+g(\beta ),g(k\alpha )=kg(\alpha )$

						因此$g$是一个$W$到$f(W)$的同构映射,于是$W \cong f(W)$,那么$\dim W=\dim f(W)$,命题得证
					\end{proof}
			\end{para}
	\section{线性子空间的直和}
		在研究了同构后,我们转而研究另一种关系:直和。与之间我们研究的是结构相同的空间不同,我们此时寻求一个空间可以分裂为数个空间,而且最好可以对每一个向量唯一分解。
		
		先提出线性空间和的概念。
		\subsection{子空间的和}
			\begin{defn}{子空间的和}{}
				设$V$是一个线性空间,$V_1,V_2 \subseteq V$是$V$的子空间

				我们定义:$V_1 + V_2 = \{\alpha_1+\alpha_2 | \alpha_1 \in V_1,\alpha _2 \in V_2\}$

				称为$V_1$和$V_2$的和
			\end{defn}
			接下来讨论子空间和的性质
			\begin{para}{0}
				\point{}
					\begin{proposition}
						设$V$是一个线性空间,$V_1,V_2 \subseteq V$是$V$的子空间

						$V_1+V_2$也是$V$的一个子空间
					\end{proposition}
					\begin{proof}
						只需验证$V_1+V_2$非空且对加法和纯量乘法封闭

						首先,$\mathbf{0}\in V_1 ,\mathbf{0}\in V_2$,所以一定有$\mathbf{0}=\mathbf{0}+\mathbf{0} \in V_1+V_2$

						取$\forall k \in F,\alpha_1+\beta_1,\alpha_2+\beta_2 \in V_1+ V_2,\alpha_1,\alpha_2 \in V_1,\beta_1,\beta_2 \in V_2$

						于是$(\alpha_1+\beta_1)+(\alpha_2+\beta_2)=(\alpha_1+\alpha_2)+(\beta_1+\beta_2)\in V_1+V_2$

						$k\cdot (\alpha_1+\beta_1)=k\cdot \alpha_1+k \cdot \beta_1 \in V_1+V_2$,于是命题得证
					\end{proof}
				\point{}
					\begin{proposition}
						$V_1+V_2=V_2+V_1$
					\end{proposition}
					\begin{proof}
						这是显然的,因为$\forall \alpha \in V_1,\beta \in V_2,\alpha +\beta =\beta +\alpha $
					\end{proof}
				\point{}
					\begin{proposition}
						$V_1+(V_2+V_3)=(V_1+V_2)+V_3$
					\end{proposition}
					\begin{proof}
						这是显然的,因为$\forall \alpha \in V_1,\beta \in V_2,\gamma \in V_3,\alpha +(\beta+\gamma ) =(\alpha +\beta) +\gamma $
					\end{proof}
				\point{}
					\begin{proposition}
						$\span(S)+\span{span}(B)=\span(S \cup B),|S|,|B| \leqslant \aleph_0$
					\end{proposition}
					\begin{proof}
						设$S=\{\alpha_x \in V| x\in A\},B=\{\beta_y \in V| y \in B\}$

						$\forall \sum\limits_{x \in A} k_x \alpha_x \in \text{span}(S),\sum\limits_{y \in B} l_y \beta_y \in \text{span}(B)$
					
						$\sum\limits_{x \in A} k_x \alpha_x + \sum\limits_{y \in B} l_y \beta_y \in \span(S\cup B) \Rightarrow \span(S)+\span(B)\subseteq \span(S\cup B)$

						$\forall \sum\limits_{z \in A \cup B} b_z \gamma_z,\gamma_z = \begin{cases}
							\alpha_z , z \in A \\
							\beta_z,z \in B-A \\
						\end{cases}$
						
						$\sum\limits_{z \in A \cup B} b_z \gamma_z= \sum\limits_{x \in A} b_x \alpha_x+\sum\limits_{y \in B-A} b_y \beta_y \in \span(S)+\span(B)$

						于是命题得证
					\end{proof}
				\point{}
					\begin{lemma}{}{}
						设$V$是一个线性空间,$V_1,V_2$是$V$的一个子空间

						那么$V_1 \cap V_2$也是$V$的一个子空间
					\end{lemma}
					\begin{proof}
						因为$\mathbf{0} \in V_1,V_2,\mathbf{0} \in V_1 \cap V_2$

						现在取$\forall k \in F,\alpha ,\beta \in V_1 \cap V_2$

						$\because \alpha \in V_1,\beta \in V_1$

						$\therefore \alpha+\beta \in V_1$

						同理有$\alpha +\beta \in V_2$。因此,$\alpha+\beta \in V_1 \cap V_2$

						$\because \alpha \in V_1,\therefore k \alpha \in V_1$

						同理有$k\alpha \in V_2$,于是有$k \alpha \in V_1 \cap V_2$

						于是命题得证

					\end{proof}
					\begin{proposition}
						$\dim (V_1+V_2)+\dim (V_1 \cap V_2)=\dim(V_1)+\dim(V_2)$
					\end{proposition}
					\begin{proof}
						取$V_1 \cap V_2$的一个基$B_0$。按照前面已经证明的结论,我们可以将其分别补充为$B_0 \cup B_1,B_0 \cup B_2$分别作为$V_1,V_2$的基

						我们只需证明:$B_0 \cup B_1 \cup B_2$是$V_1+V_2$的一个基

						我们首先证明它线性无关。我们只需取它的一个有限子集$S_0 \cup S_1 \cup S_2 \subseteq B_0 \cup B_1 \cup B_2,S_0 \subseteq B_0,S_1 \subseteq B_1,S_2 \subseteq B_2$
					
						设$B_0 = \{\alpha_1,\cdots,\alpha_m\},B_1 = \{\beta_1,\cdots,\beta_n\},B_2=\{\gamma_1,\cdots,\gamma_l\}$

						考虑线性组合$\sum\limits_{i=1}^{m} a_i \alpha_i +\sum\limits_{j=1}^{n} b_j \beta_j +\sum\limits_{k=1}^{l} c_k \gamma_k = \mathbf{0}$
					
						那么,$\sum\limits_{i=1}^{m} a_i \alpha_i +\sum\limits_{j=1}^{n} b_j \beta_j = - \sum\limits_{k=1}^{l} c_k \gamma_k$

						但是,等式左边的向量是由$S_0 \cup S_1$张成的,因此是$V_1$的元素;右边的向量由$S_2$张成,因此是$V_2$的元素

						再由线性空间的性质可得$\sum\limits_{i=1}^{m} a_i \alpha_i +\sum\limits_{j=1}^{n} b_j \beta_j \in V_1 \cap V_2$
					
						事实上,一定有$\forall b_j = 0$,因为:

						$\sum\limits_{i=1}^{m} a_i \alpha_i +\sum\limits_{j=1}^{n} b_j \beta_j \in V_1 \cap V_2$,那么它一定也可以在$S_0$上表示,即

						$\exists \{d_i\},\sum\limits_{i=1}^{m} a_i \alpha_i +\sum\limits_{j=1}^{n} b_j \beta_j=\sum\limits_{i=1}^{m} d_i \alpha_i$

						此时如果$\exists b_j \neq 0$,那么就给出了一个向量在线性无关向量组$S_0 \cap S_1$上的两种不同表示(注意等式右侧可以视为$S_1$部分缺省的一种表出),而这是不可能的。

						于是有$\sum\limits_{i=1}^{m} a_i \alpha_i = - \sum\limits_{k=1}^{l} c_k \gamma_k$,同理可推出$\forall a_i,c_k = 0$。

						于是我们证明了$B_0 \cup B_1 \cup B_2$线性无关。而由构造的方式可知,它一定可以表出$V_1+V_2$中的任意向量。

						此时有$\card (B_0 \cup B_1 \cup B_2) + \card B_0 = \card (B_0 \cup B_1) + \card (B_0 \cup B_2)$

						$\Rightarrow \dim (V_1+V_2) + \dim (V_1 \cap V_2) = \dim V_1 + \dim V_2$,于是命题得证
					\end{proof}
					可以立即得到下面的推论:
					\begin{corollary}{}{}
						$\dim (V_1+V_2)=\dim V_1+\dim V_2 \Leftrightarrow V_1 \cap V_2 = \{\mathbf{0}\}$
					\end{corollary}
					\begin{proof}
						这是显然的,因为维数等于零的线性空间有且只有$\{\mathbf{0}\}$
					\end{proof}
			\end{para}
		\subsection{子空间的直和}
			我们接下来转入一种特殊的和:使和空间中任意向量表出都唯一的空间。这样的空间的好处是:未来我们对向量做分解时,只会“投影”出唯一的分量
			\begin{defn}{子空间的直和}{}
				设$V$是一个线性空间,$V_1,V_2 \subseteq V$是$V$的子空间

				$W=V_1+V_2$如果满足:$\forall \alpha \in W,\alpha =\alpha_1+\alpha_2,\alpha_1 \in V_1,\alpha_2 \in V_2$的表出方式唯一

				那么我们称$W$是$V_1$和$V_2$的直和,记作$W=V_1 \oplus V_2$。
				
				如果$V=V_1 \oplus V_2$,我们称$V_2$是$V_1$的补空间
			\end{defn}
			我们首先探究一些更加简洁易用的直和充要条件
			\begin{them}{子空间直和的等价命题}{}
				设$V$是一个线性空间,$V_1,V_2 \subseteq V$是$V$的子空间

				以下四个命题等价:

				\begin{enumerate}
					\item $V_1+V_2$是直和
					\item $V_1+V_2$中$\mathbf{0}$的表法唯一
					\item $V_1 \cap V_2 = \{\mathbf{0}\}$
					\item 分别任取$V_1,V_2$的一个基$B_1,B_2$,$B_1 \cup B_2$是$V_1+V_2$的一个基
				\end{enumerate}
			\end{them}
			\begin{proof}
				我们首先证明前三个命题等价。

				第一个命题显然可以推出第二个命题,因为这就是直和的定义;

				接下来证明第二个命题可以推出第三个命题。取$\alpha \in V_1 \cap V_2$,这是一定可以取到的,因为$V_1,V_2$至少包含零向量

				那么,因为$V_1 \cap V_2$是一个线性空间,一定$\exists -\alpha \in V_1 \cap V_2 ,\alpha +(-\alpha )=\mathbf{0}$

				此时,$\alpha \in V_1,-\alpha \in V_2$,但是$\mathbf{0}$仅有一种表出方式,所以一定有$\alpha =\mathbf{0}$,得证

				接下来证明第三个命题可推出第一个命题。取$\gamma \in V_1+V_2$,假设它有两种表出方式$\gamma =\alpha_1+\beta_1=\alpha_2+\beta_2$
			
				那么有$(\alpha_1-\alpha_2)+(\beta_1-\beta_2)=\mathbf{0}$

				此时,$\beta_1-\beta_2=-(\alpha_1-\alpha_2) \in V_1$,于是$\beta_1 - \beta_2 \in V_1 \cap V_2$

				所以$\beta_1-\beta_2 = \mathbf{0} \Rightarrow \alpha_1 - \alpha_2 = \mathbf{0}$,这说明表出方式的确是唯一的。

				接下来证明第三个命题和第四个命题可以互推。只需要仿照前面证明和空间维数性质的方式证明即可。

				于是命题得证。
			\end{proof}
			我们可以得出一个后面很常用的推论
			\begin{corollary}{}{}
				$V_1+V_2$是直和$\Leftrightarrow \dim (V_1+V_2) = \dim V_1 +\dim V_2$
			\end{corollary}
			\begin{proof}
				$V_1+V_2$是直和$\Leftrightarrow V_1 \cap V_2 = \{\mathbf{0}\} \Leftrightarrow \dim (V_1+V_2) = \dim V_1 +\dim V_2$
			\end{proof}
			这个定理的另一个推论是:补空间总是存在的。
			\begin{corollary}{}{}
				线性空间$V$的任意子空间$W$的补空间总是存在的。
			\end{corollary}
			\begin{proof}
				取$W$的一个基$B_0$,将其补充至其成为$V$的一个基$B_0 \cap B_1$

				设$E = \{\alpha | \alpha \in \span(B),B \subseteq B_1,|B| \leqslant \aleph_0\}$,它的一个基是

				事实上,显然一定有$W+E=V$,而它们的基的并是和的基

				因此有$W \oplus E = V$,命题得证。
			\end{proof}
		\subsection{多个子空间的直和}
			我们仿照前面的方式定义有限个子空间的直和
			\begin{defn}{有限个子空间的直和}{}
				设$V$是一个线性空间,$V_1,\cdots,V_n \subseteq V$是$V$的子空间

				$W=V_1+\cdots+V_n$如果满足:$\forall \alpha \in W,\alpha =\alpha_1+\cdots+\alpha_n,\alpha_1 \in V_1,\cdots,\alpha_n \in V_n$的表出方式唯一

				那么我们称$W$是$V_1,\cdots,V_2$的直和,记作$W=\bigoplus_{i=1}^{n} V_i = V_1 \oplus \cdots \oplus V_2$。
			\end{defn}
			但是,如果希望拓展到可数乃至不可数个子空间的直和,就无法这样定义了,因为此时向量求和逐渐变得难以定义

			我们转而定义直和为一系列映射的集合,我们后面将看到,对于有限情形,它和前面的定义是同构的。
			\begin{defn}{子空间直和的一般定义}
				设$V$是一个线性空间,$\{V_x \subseteq V| x \in I\}$是$V$的一族子空间

				我们定义:

				\begin{equation}
					\bigoplus_{x \in I} V_x = \{f: I \to \bigcup_{x \in I} V_x| f(x) \in V_x,|supp(f)| < \aleph_0\}
				\end{equation}

				称为这一族子空间的直和
			\end{defn}
			接下来讨论直和的性质
			\begin{para}{0}
				\point{}
					\begin{proposition}
						$V_1 \oplus \cdots \oplus V_n \cong \{f: I \to \bigcup\limits_{i=1}^{n} V_i| f(i) \in V_i,|supp(f)| < \aleph_0\},I=\{1,\cdots,n\}$
					\end{proposition}
					\begin{proof}
						考虑映射$T:V_1 \oplus \cdots \oplus V_n \ni \alpha = \alpha_1 + \cdots+\alpha_n \mapsto \left(f_\alpha : I \ni i \mapsto \alpha_i \in \bigcup_{i=1}^{n} V_i\right) \in \{f: I \to \bigcup\limits_{i=1}^{n} V_i| f(i) \in V_i,|supp(f)| < \aleph_0\}$
					
						这个映射的定义是良好的,因为直和中向量的标出方式唯一

						我们验证它是一个同构映射。

						首先,它显然是一个单射;其次,它也是一个满射,因为直和中每一个子空间中的向量的选取没有任何限制

						接下来验证可加性和齐次性

						$\forall \alpha =\alpha_1+\cdots+\alpha_n,\beta =\beta_1+\cdots+\beta_n \in V_1 \oplus \cdots \oplus V_n,k \in F$

						$T(\alpha +\beta )=f_{\alpha +\beta },f_{\alpha +\beta }(i)=\alpha_i + \beta_i = f_\alpha (i)+f_\beta (i)$

						$\Rightarrow T(\alpha+\beta )=f_{\alpha +\beta }=f_\alpha +f_\beta =T(\alpha )+T(\beta )$
					
						$T(k\alpha )=f_{k\alpha },f_{k\alpha }(i)=k\alpha_i=kf_\alpha $

						$\Rightarrow T(k\alpha )=f_{k\alpha }=kf_\alpha =kT(\alpha )$

						于是命题得证。
					\end{proof}
					因此第二种定义的确可以视为第一种定义的推广。
					
					值得注意的是,我们尽管定义了无限个子空间的直和,但是并没有定义和,所以之前讨论的性质很多只能受限在有限范围
				\point{}
					\begin{proposition}
						设$V$是一个线性空间,$V_1,\cdots,V_n \subseteq V$是$V$的子空间

						以下四个命题等价:

						\begin{enumerate}
							\item $\bigcup\limits_{i=1}^{n} V_i$是直和
							\item $\bigcup\limits_{i=1}^{n} V_i$中$\mathbf{0}$的表法唯一
							\item $\forall i,V_i \cap (\bigcup\limits_{j\neq i} V_j) = \{\mathbf{0}\}$
							\item 分别任取$V_1,\cdots,V_n$的一个基$B_1,\cdots,B_n$,$\bigcup\limits_{i=1}^{n} B_i$是$\bigcup\limits_{i=1}^{n} V_i$的一个基
						\end{enumerate}
					\end{proposition}
					\begin{proof}
						仿照两个子空间的情形即可。
					\end{proof}
					也可以得出类似的推论:
					\begin{corollary}{}{}
						$V_1 + \cdots + V_n$是直和$\Leftrightarrow \dim (V_1+\cdots+V_n)=\dim V_1 + \cdots + \dim V_n$
					\end{corollary}
					\begin{proof}
						这是前面命题的直接推论。因为从证明过程我们知道,直和当且仅当子空间的基互不相交地组成和空间的基
					\end{proof}
			\end{para}
	\section{商空间}
		本节中,我们考虑一种新的线性空间的构造方式:商空间。商空间是将一个线性空间中的向量划分为等价类,进而构造出一个新的线性空间。

		比如说,在三维空间$\R^3$中,我们可以把所有的水平面看作一个子空间,然后将每个平面上的向量看作一个等价类。这样,我们就可以得到一个新的线性空间,其中的元素是平面,而不是具体的向量。

		同时,我们也可以看到,于是同时每一个巨大的平面变得可以由与$z$轴的交点来唯一标识,从而我们做到了“消去二维平面的影响”。

		商空间比较抽象,但是在后续线性映射的研究中有巨大作用。
		\subsection{商空间的定义}
			在定义商空间之前,我们先定义一个关于子空间的等价关系
			\begin{defn}{关于子空间的等价}{}
				设$V$是一个线性空间,$W$是$V$的子空间,我们定义以下等价关系$\sim$:

				$\alpha ,\beta \in V,\alpha \sim  \beta \Leftrightarrow \alpha -\beta \in W$
			\end{defn}
			我们首先验证它的确是一个等价关系
			\begin{proposition}
				$\sim$是一个等价关系
			\end{proposition}
			\begin{proof}
				首先验证它满足自反性,$\forall \alpha ,\alpha -\alpha = \mathbf{0} \in W$

				接下来验证它满足对称性,$\alpha \sim \beta \Leftrightarrow \alpha -\beta \in W \Leftrightarrow \beta -\alpha = -(\alpha -\beta) \in W \Leftrightarrow \beta \sim \alpha $
				
				最后验证它满足传递性,$\alpha \sim \beta ,\beta \sim \gamma \Leftrightarrow \alpha -\beta \in W,\beta -\gamma \in W \Rightarrow \alpha -\gamma \in W \Rightarrow \alpha \sim \gamma $
			\end{proof}
			在这个等价关系下$\alpha \in V$的等价类$\bar{\alpha } $也常常记作$\alpha +W$,这个表示方式是自然的,因为:

			$\bar{\alpha }=\{\beta | \beta -\alpha \in W\}=\{\alpha +\beta | \beta \in W\}$
			现在我们来定义商空间
			\begin{defn}{商空间}{}
				设$V$是一个线性空间,$W$是$V$的一个子空间,我们定义:
				\begin{equation}
					V/W := \{\alpha + W|\alpha \in V\}
				\end{equation}
				称为$V$关于$W$的商空间,其中的加法和纯量乘法定义为:

				$(\alpha +W)+(\beta +W):= (\alpha +\beta )+W$

				$k \cdot (\alpha +W)=k\alpha + W$
			\end{defn}
			我们首先验证$V/W$的加法和乘法的定义是良好的
			\begin{proposition}
				如果$\alpha_1 \sim \alpha_2,\beta_1 \sim \beta_2,k \in F$,那么

				$(\alpha_1+W)+(\beta_1+W)=(\alpha_2+W)+(\beta_2+W)$

				$k(\alpha_1 + W)=k(\alpha_2+W)$
			\end{proposition}
			\begin{proof}
				首先,$\alpha_1 \sim \alpha_2 \Leftrightarrow \alpha_1 -\alpha_2 \in W$

				$\beta_1 \sim \beta_2 \Leftrightarrow \beta_1 -\beta_2 \in W$

				于是有$(\alpha_1 +\beta_1) - (\alpha_2 +\beta_2) = (\alpha_1 -\alpha_2)+(\beta_1-\beta_2) \in W$

				因此$(\alpha_1+W)+(\beta_1+W)=(\alpha_2+W)+(\beta_2+W)$

				同理可得$k(\alpha_1 + W)=k(\alpha_2 + W)$,命题得证
			\end{proof}
			于是它的定义的确是良好的。$V/W$中的加法和乘法符合公理是显然的。

			这样,我们就完成了商空间的定义,并验证了它的确是一个线性空间
		\subsection{商空间的性质}
			\begin{para}{0}
				\point{}
					\begin{them}{}{商空间的同构}{}
						设$V$是一个线性空间,$V = W \oplus U$

						那么,$V/W \cong U$
					\end{them}
					\begin{proof}
						设$T:U \ni \alpha \mapsto \alpha + W \in V/W$

						我们验证它的确是一个同构映射

						首先验证它是一个单射。设$T(\alpha )=T(\beta ),\alpha ,\beta \in U$

						$\Rightarrow \alpha +W=\beta +W \Rightarrow \alpha -\beta =W$

						记$\alpha -\beta =\gamma $,事实上,只能有$\gamma =\mathbf{0}$,否则$\alpha =\beta + \gamma $就给出了$\alpha $的第二种表示方式,但这在直和中是不可能的。

						因此$\alpha =\beta$,$T$的确是单射

						接下来证明它是一个满射,取$\forall \alpha \in V,\alpha = \alpha_1+\alpha_2,\alpha_1 \in W,\alpha_2 \in U$

						注意到,$T(\alpha_2) = \alpha_2+W=\alpha + W$,因为$\alpha -\alpha_2 = \alpha_1 \in W$,于是$T$的确是一个满射
					
						最后我们验证可加性和齐次性。

						$T(\alpha +\beta )=(\alpha +\beta )+W = (\alpha +W)+(\beta +W)=T(\alpha )+T(\beta)$

						$T(k\alpha )=k\alpha + W=k(\alpha +W)=kT(\alpha )$

						于是命题得证
						
						
					\end{proof}
					值得注意的是,这个定理的逆命题并不成立(尽管逆命题的形式优美到让人难以想象它是错误的),比如说:
					
					考虑$\R^3 / {(\R,\R,0)} \cong (\R,0,0)$,但是$(\R,\R,0) + (\R,0,0) = (\R,\R,0) \neq \R^3$

					后面我们看到,如果追加一些基的限制,我们可以让逆命题成立
				\point{商空间的维数}
					
					利用上面的定理可以立即得出以下结论:
					\begin{corollary}{商空间的维数}{}
						$\dim V/W = \dim V-\dim W$
					\end{corollary}
					\begin{proof}
						我们之前已经证明,任意子空间都有补空间

						所以只需取$W$的补空间$U$,此时有$V=W \oplus U$

						$\Rightarrow V/W \cong U \Rightarrow \dim V/W = \dim U=\dim V-\dim W$,命题得证
					\end{proof}
				\point{}
					\begin{proposition}
						设$V$是一个线性空间,$V=W+U,U = \{\alpha | \alpha \in \span(S),S \subseteq B,|S| < \aleph_0\}$

						那么,$V/W$的一个基是$\{\alpha + W| \alpha \in B\} \Leftrightarrow V=W \oplus U,B$是$U$的一个基
					\end{proposition}
					\begin{proof}
						先证明充分性。

						首先,$B$一定是线性无关的,因为:$\sum\limits_{i=1}^{n} k_i (\alpha_i+ W) = \mathbf{0}_{V/W} = W$

						$\Leftrightarrow \sum\limits_{i=1}^{n} k_i \alpha_i + W = W \Leftrightarrow \sum\limits_{i=1}^{n} k_i \alpha_i = \mathbf{0}$

						那么因为$\{\alpha + W| \alpha \in B\}$是基,$B$也一定线性无关

						于是$B$是$U$的一个基,接下来证明$V=W\oplus U$

						取$\forall \alpha \in W \cap U$,一定有$\alpha = \sum\limits_{i=1}^{n} k_i \alpha_i$,其中$\alpha_i \in B$

						又因为$\alpha \in W$,那么有$\alpha + W = W \Rightarrow \sum\limits_{i=1}^{n} k_i (\alpha_i+ W) = W \Rightarrow k_i = 0$

						于是$\alpha = \mathbf{0} \Rightarrow W \cap U = \{\mathbf{0}\}$,因此$V=W\oplus U$,充分性得证。

						必要性利用前面的定理是显然的,因为$V/W \cong U$,其同构映射为$V/W \ni \alpha + W \mapsto \alpha \in U$

						那么按照基在同构映射下映射为一个基的性质,必要性得证。

						于是命题得证。
					\end{proof}
			\end{para}
	\section{对偶空间}
		本节中,我们研究一种比较具体的线性空间:对偶空间。对偶空间是一个映射集,作为下一章的铺垫。我们不会聚焦于线性映射的具体结构,而是关注它的线性空间性质
		\subsection{对偶空间的定义}
			\begin{defn}{对偶空间}{}
				设$V$是一个$F$上的线性空间,我们定义:
				\begin{equation}
					V^* = \{f:V \to F| \forall \alpha ,\beta \in V,k \in F,f(\alpha +\beta )=f(\alpha )+f(\beta ),f(k\alpha )=kf(\alpha )\}
				\end{equation}

				其中$V^*$的基域是$F$,它的加法和纯量乘法是即是函数的加法和纯量乘法

				称为$V$的对偶空间
			\end{defn}
		\subsection{有限维对偶空间的对偶基}
			我们希望知道,对偶空间的维数到底是多少。在本节中,我们仅仅探讨有限维线性空间的对偶空间,因为进一步的讨论需要线性映射的进一步知识

			我们指出,此时有一类与原空间的基密切相关的基,我们称之为对偶基
			\begin{defn}{对偶基}{}
				设$V$是一个线性空间,$\dim V= n < \aleph_0$,$B=\{\alpha_1,\cdots,\alpha_n\}$是它的一个基

				集合$B^*=\{f_i \in V^*| f_i(\alpha_j) = \delta _{ij} = \begin{cases}
					1, i=j \\
					0, i \neq j
				\end{cases},1 \leqslant i \leqslant n\}$

				称为$B$的对偶基
			\end{defn}
			
			对偶基的定义的确是良好的,因为$B=\{\alpha_1,\cdots,\alpha_n\}$是一个$V$的基,又函数是线性的,所以其在基上的映射的确给出了函数的完整定义

			我们接下来验证$B^*$是$V^*$的一个基。
			\begin{proposition}
				$B^*$是$V^*$的一个基
			\end{proposition}
			\begin{proof}
				首先证明$B^*$是线性无关的。

				取$B^*$的任意有限子集$\{f_1,\cdots,f_n\}$

				考虑线性组合$k_1 f_1 + \cdots+k_n f_n = \mathbf{0}_{V^*}$

				代入$\alpha_i$,得:$k_1 f_1(\alpha_i) + \cdots+k_n f_n(\alpha_i) = 0 \Rightarrow k_i = 0$。因此$B^*$线性无关。

				接下来证明它的确可以生成整个对偶空间

				$\forall A \in V^*$,考虑线性组合$T = \sum\limits_{i=1}^{n} A(\alpha_i) f_i \in V^*$

				我们来证明,$A=T$。

				$\forall \gamma = \sum\limits_{i=1}^{n} k_i \alpha_i \in V$

				$T(\gamma ) = \sum\limits_{i=1}^{n} A(\alpha_i) f_i(\gamma )=\sum\limits_{i=1}^{n} A(\alpha_i) f_i(\sum\limits_{i=1}^{n} k_i \alpha_i)$

				$= \sum\limits_{i=1}^{n} A(\alpha_i) k_i= A(\sum\limits_{i=1}^{n} k_i \alpha_i)=A(\gamma )$。
b    
				于是命题得证。
			\end{proof}
			它的直接推论是我们之前提出的问题的答案,即:
			\begin{corollary}{有限维线性空间的对偶空间的维数}{}
				如果$\dim V = n < \aleph_0$,那么$\dim V^* = n$
			\end{corollary}
			证明过程中还使用了一个后续很常用的性质:
			\begin{proposition}
				设$\{\alpha_1,\cdots,\alpha_n\}$是$V$的一个基,$\{f_1,\cdots,f_n\}$是它的对偶基,那么

				$\forall A \in V^*,A=\sum\limits_{i=1}^{n} A(\alpha_i) f_i$
			\end{proposition}
			类似地,我们也可以对向量作一个展开
			\begin{proposition}
				设$\{\alpha_1,\cdots,\alpha_n\}$是$V$的一个基,$\{f_1,\cdots,f_n\}$是它的对偶基,那么

				$\forall \gamma  \in V,\gamma  =\sum\limits_{i=1}^{n} f_i(\gamma )\alpha_i$
			\end{proposition}
		\subsection{双重对偶空间}
			\begin{defn}{双重对偶空间}{}
				$V^*$的对偶空间称为$V$的双重对偶空间,记作$V^{**}$
			\end{defn}
			双重对偶空间的最大意义是,它体现了对偶并非是一个“越来越远”的操作,反而每两次操作在同构意义下可以抵消

			接下来的命题将逐步指出为什么我们这么说
			\begin{them}{线性空间与其双重对偶空间的自然同构}{}
				设$V$是一个$F$上的线性空间,$\dim V = n < \aleph_0$。
				
				那么$V \cong V^{**}$,并且存在不依赖于基的自然同构

				\begin{equation}
					\psi : V \ni \alpha \mapsto \left(\alpha^{**}:V^* \ni f \mapsto f(\alpha ) \in F\right) \in V^{**}
				\end{equation}
			\end{them}
			\begin{proof}
				首先验证$\psi $是一个单射。

				如果不成立,那么$\exists \alpha_1 \neq \alpha_2,\psi (\alpha_1)=\psi(\alpha_2)=\alpha^{**}$

				依照$\psi $的定义,此时有:$\forall f \in V^*,f(\alpha_1)=f(\alpha_2)$。

				我们只需举出一个$V^*$中的映射即可推出矛盾。比如说:
				
				记$\gamma =\alpha_1-\alpha_2$,将其补充为$V$的一个基$\{\gamma ,\beta_1,\cdots,\beta_{n-1}\}$

				考虑映射$f\in V^*,f(\gamma )=1,f(\beta_i)=0$。但是本来理应$f(\gamma )=0$,与假设矛盾。

				所以$\psi $是一个单射。

				我们接下来验证可加性和齐次性。

				$\forall \alpha ,\beta \in V,k \in F$

				$\psi (\alpha +\beta )(f) = f(\alpha +\beta )=f(\alpha )+f(\beta )=\psi (\alpha )(f)+\psi (\beta )(f) \Rightarrow \psi (\alpha +\beta )=\psi (\alpha )+\psi (\beta )$
			
				$\psi (k\alpha )(f)=f(k\alpha )=kf(\alpha )=k\psi (\alpha )(f) \Rightarrow \psi (k\alpha )=k\psi (\alpha )$
			
				我们其实不需要验证满射性,因为$\dim V = \dim V^{**} = n < \aleph_0$,结合单射性和可加性和齐次性,我们可以直接得出$\psi $是一个同构映射。
			
				于是命题得证。
			\end{proof}
			之所以我们说“两次对偶相互抵消”,是因为在一般的线性空间之间,两次同构会使基的作用堆叠,其同构依旧是依赖于基的。但是自然同构却不依赖于任何基

			接下来的命题让我们看到,自然同构不仅仅是消除了基的影响的同构,而且还恰好是由两次对偶基导出的同构映射的复合
			\begin{proposition}
				设$V$是一个$F$上的线性空间,$\dim V = n < \aleph_0$。

				设$\{\alpha_1,\cdots,\alpha_n\}$是$V$的一个基,$\{f_1,\cdots,f_n\}$是它的对偶基,$\{\alpha_1^{**},\cdots,\alpha_n^{**}\}$是$\{f_1,\cdots,f_n\}$的对偶基
				
				设$\sigma : V \ni \sum\limits_{i=1}^{n} k_i \alpha_i \mapsto \sum\limits_{i=1}^{n} k_i f_i \in V^*$

				$\tau : V^* \ni \sum\limits_{i=1}^{n} l_i f_i \mapsto \sum\limits_{i=1}^{n} l_i \alpha_i^{**} \in V^{**}$
			
				那么,$\psi = \tau \circ \sigma $
			\end{proposition}
			\begin{proof}
				$\forall \gamma  = \sum\limits_{i=1}^{n} k_i \alpha_i \in V,f\in V^*$

				$(\tau \circ \sigma) (\gamma )(f)=\tau \left(\sigma (\sum\limits_{i=1}^{n} k_i \alpha_i)\right)(f)$
			
				$=\tau (\sum\limits_{i=1}^{n} k_i f_i)(f) = \sum\limits_{i=1}^{n} k_i \alpha_i^{**}(f)$

				$=\sum\limits_{i=1}^{n} k_i \alpha_i^{**}\left(\sum\limits_{j=1}^{n} f(\alpha_j)f_j\right)$

				$=\sum\limits_{i=1}^{n} k_i f(\alpha_i)=f \left(\sum\limits_{i=1}^{n} k_i \alpha_i\right)$
				
				$=f(\gamma )=\psi (\gamma )(f) \Rightarrow \psi =\tau \circ \sigma $

				于是命题得证。
			\end{proof}
\ifx\allfiles\undefined
\end{document}
\fi