\ifx\allfiles\undefined
\documentclass[12pt, a4paper, oneside, UTF8]{ctexbook}
\setCJKmainfont{SimSun}
\def\path{../config}
\input{../config/_config}
\begin{document}
	% \input{../config/cover}
	\else
	\fi
	%标题
	\chapter{附录}
	这一部分中,对于正文中因为逻辑结构无法提及的部分,进行补充。
	%--------------------正文---------------------------
		\section{一些典型的域}
			\subsection{$F_p$}
				首先约定,这一部分的讨论中,都认为$p$是一个素数。
				
				我们首先讨论的是一个典型的有限域——模$p$剩余类域。
				
				\begin{defn}{$F_p$}{}
					设$F$是一个域,$Char F \geqslant p$且$p$是一个素数,
					
					我们定义$F_p = N(\Z_p)$
					
					并定义其中的加法和乘法为:
					
					$+_F = N^-1 \circ + \circ N,{\cdot}_F = N^-1 \circ \cdot \circ N$
				\end{defn}
				\begin{them}{$F_p$没有真子域}{}
					设$p$是一个素数,那么域$F_p$不存在真子域,即$F_p \textbackslash E \rightarrow E = F_p$
				\end{them}
				\begin{proof}
					不妨假设命题不成立,那么一定有真子域$E \subseteq F_p$
					
					不妨假设$[F_p : E]=d$,因为$F_p$是有限域,那么$|E|,|F_p|$都是有限的。
					
					但是,$|F_p|={|E|}^d$
					
					$\Rightarrow p = {|E|}^d$,但是$p$是素数,因此只可能$d=1$
					
					于是$|F_p|=|E|$,那么只可能$F_p=E$,与假设矛盾,于是命题得证。
				\end{proof}
			\subsection{$\Q$}
				\begin{them}{$Q$没有真子域}
					域$\Q$不存在真子域,即$\Q \textbackslash E \rightarrow E = \Q$
				\end{them}
				\begin{proof}
					不妨假设命题不成立,$E \subset \Q$是$\Q$真子域。
					
					那么,因为$0,1 \in E$,由域对加法封闭,那么一定有$\N \subseteq E$
					
					进一步,因为任意元素的加法逆存在,于是有$\Z \subseteq E$
					
					于是,由任意非零元素的逆存在,一定有$\Q \subseteq E$。
					
					但是,我们假设$E \subset \Q$,矛盾。于是命题成立
				\end{proof}
	
	\ifx\allfiles\undefined
\end{document}
\fi