\ifx\allfiles\undefined
\documentclass[12pt, a4paper, oneside, UTF8]{ctexbook}
\setCJKmainfont{SimSun}
\def\path{../config}
\usepackage{amsmath}
\usepackage{amsthm}
\usepackage{amssymb}
\usepackage{graphicx}
\usepackage{mathrsfs}
\usepackage{enumitem}
\usepackage{geometry}
\usepackage[colorlinks, linkcolor=black]{hyperref}
\usepackage{stackengine}
\usepackage{yhmath}
\usepackage{extarrows}
\usepackage{unicode-math}
\usepackage{tikz}
\usepackage{tikz-cd}
\usepackage{pifont}
\usepackage{pgfplots}
\usepackage{tikz-3dplot}

\usepackage{fancyhdr}
\usepackage[dvipsnames, svgnames]{xcolor}
\usepackage{listings}

\definecolor{mygreen}{rgb}{0,0.6,0}
\definecolor{mygray}{rgb}{0.5,0.5,0.5}
\definecolor{mymauve}{rgb}{0.58,0,0.82}

\graphicspath{ {figure/},{../figure/}, {config/}, {../config/} }

\linespread{1.6}

\geometry{
    top=25.4mm, 
    bottom=25.4mm, 
    left=20mm, 
    right=20mm, 
    headheight=2.17cm, 
    headsep=4mm, 
    footskip=12mm
}

\setenumerate[1]{itemsep=5pt,partopsep=0pt,parsep=\parskip,topsep=5pt}
\setitemize[1]{itemsep=5pt,partopsep=0pt,parsep=\parskip,topsep=5pt}
\setdescription{itemsep=5pt,partopsep=0pt,parsep=\parskip,topsep=5pt}

\lstset{
    language=Mathematica,
    basicstyle=\tt,
    breaklines=true,
    keywordstyle=\bfseries\color{NavyBlue}, 
    emphstyle=\bfseries\color{Rhodamine},
    commentstyle=\itshape\color{black!50!white}, 
    stringstyle=\bfseries\color{PineGreen!90!black},
    columns=flexible,
    numbers=left,
    numberstyle=\footnotesize,
    frame=tb,
    breakatwhitespace=false,
} 
\usepackage[strict]{changepage} 
\usepackage{framed}
\usepackage{tcolorbox}
\tcbuselibrary{most}

\definecolor{greenshade}{rgb}{0.90,1,0.92}
\definecolor{redshade}{rgb}{1.00,0.88,0.88}
\definecolor{brownshade}{rgb}{0.99,0.95,0.9}
\definecolor{lilacshade}{rgb}{0.95,0.93,0.98}
\definecolor{orangeshade}{rgb}{1.00,0.88,0.82}
\definecolor{lightblueshade}{rgb}{0.8,0.92,1}
\definecolor{purple}{rgb}{0.81,0.85,1}

% #### 将 config.tex 中的定理环境的对应部分替换为如下内容
% 定义单独编号,其他四个共用一个编号计数 这里只列举了五种,其他可类似定义(未定义的使用原来的也可)
\newtcbtheorem[number within=section]{defn}%
{定义}{colback=OliveGreen!10,colframe=Green!70,fonttitle=\bfseries}{def}

\newtcbtheorem[number within=section]{lemma}%
{引理}{colback=Salmon!20,colframe=Salmon!90!Black,fonttitle=\bfseries}{lem}

% 使用另一个计数器 use counter from=lemma
\newtcbtheorem[use counter from=lemma, number within=section]{them}%
{定理}{colback=SeaGreen!10!CornflowerBlue!10,colframe=RoyalPurple!55!Aquamarine!100!,fonttitle=\bfseries}{them}

\newtcbtheorem[use counter from=lemma, number within=section]{criterion}%
{准则}{colback=green!5,colframe=green!35!black,fonttitle=\bfseries}{cri}

\newtcbtheorem[use counter from=lemma, number within=section]{corollary}%
{推论}{colback=Emerald!10,colframe=cyan!40!black,fonttitle=\bfseries}{cor}
% colback=red!5,colframe=red!75!black

% 这个颜色我不喜欢
%\newtcbtheorem[number within=section]{proposition}%
%{命题}{colback=red!5,colframe=red!75!black,fonttitle=\bfseries}{cor}

% .... 命题 例 注 证明 解 使用之前的就可以(全文都是这种框框就很丑了),也可以按照上述定义 ...
\renewenvironment{proof}{\par\textbf{证明:}\;}{\qed\par}
\newenvironment{solution}{\par{\textbf{解:}}\;}{\qed\par}
\newtheorem{proposition}{\indent 命题}[section]
\newtheorem{example}{\indent \color{SeaGreen}{例}}[section] % 绿色文字的 例 ,不需要就去除\color{SeaGreen}{}
\newtheorem*{rmk}{\indent 注}
\usepackage{amssymb}
\setmathfont{LatinModernMath-Regular}
\setmathfont[range=\mathbb]{TeXGyrePagellaMath-Regular}
\def\d{\mathrm{d}}
\def\R{\mathbb{R}}
\def\C{\mathbb{C}}
\def\Q{\mathbb{Q}}
\def\N{\mathbb{N}}
\def\Z{\mathbb{Z}}
\newcommand{\bs}[1]{\boldsymbol{#1}}
\newcommand{\ora}[1]{\overrightarrow{#1}}
\newcommand{\myspace}[1]{\par\vspace{#1\baselineskip}}
\newcommand{\xrowht}[2][0]{\addstackgap[.5\dimexpr#2\relax]{\vphantom{#1}}}
\newenvironment{ca}[1][1]{\linespread{#1} \selectfont \begin{cases}}{\end{cases}}
\newenvironment{vx}[1][1]{\linespread{#1} \selectfont \begin{vmatrix}}{\end{vmatrix}}
\newcommand{\tabincell}[2]{\begin{tabular}{@{}#1@{}}#2\end{tabular}}
\newcommand{\pll}{\kern 0.56em/\kern -0.8em /\kern 0.56em}
\newcommand{\dive}[1][F]{\mathrm{div}\;\bs{#1}}
\newcommand{\rotn}[1][A]{\mathrm{rot}\;\bs{#1}}
\usepackage{xeCJK}
\setCJKmainfont{SimSun}[BoldFont={SimHei}, ItalicFont={KaiTi}] % 设置中文支持

\newcommand{\point}[1]{\item {#1}}
\newenvironment{para}[1]{%
\ifcase#1\relax
\begin{enumerate}[label=\arabic*.] % 1.2.3.
\or
\begin{enumerate}[label=\textcircled{\arabic*}] % ①②③
\or
\begin{enumerate}[label=(\roman*)] % (i)(ii)(iii)
\else
\begin{enumerate}[label=\arabic*.] % 默认格式
\fi
}{
\end{enumerate}
}

\def\myIndex{0}
% \input{\path/cover_package_\myIndex.tex}

\def\myTitle{高等代数笔记}
\def\myAuthor{Zhang Liang}
\def\myDateCover{\today}
\def\myDateForeword{\today}
\def\myForeword{前言标题}
\def\myForewordText{
    前言内容
}
\def\mySubheading{副标题}


\begin{document}
	% \input{\path/cover_text_\myIndex.tex}

\newpage
\thispagestyle{empty}
\begin{center}
    \Huge\textbf{\myForeword}
\end{center}
\myForewordText
\begin{flushright}
    \begin{tabular}{c}
        \myDateForeword
    \end{tabular}
\end{flushright}

\newpage
\pagestyle{plain}
\setcounter{page}{1}
\pagenumbering{Roman}
\tableofcontents

\newpage
\pagenumbering{arabic}
\setcounter{chapter}{0}
\setcounter{page}{0}

\pagestyle{fancy}
\fancyfoot[C]{\thepage}
\renewcommand{\headrulewidth}{0.4pt}
\renewcommand{\footrulewidth}{0pt}








	\else
	\fi
	%标题
	\chapter{附录}
	这一部分中,对于正文中因为逻辑结构无法提及的部分,进行补充。
	%--------------------正文---------------------------
		\section{$\hom(V,W)$的维数}
			这篇附录中,我们解决一个问题:$\hom(V,W)$的维数是多少。
			\begin{lemma}{}{}
				$\hom(V,W) \cong \bigoplus_{i=1}^{n} \hom(F,W)$
			\end{lemma}
			\begin{proof}
				任取$V$的一个基$\{\alpha_1,\cdots,\alpha_n\}$,我们定义:
						
				$\phi : \hom(V,W) \ni f \mapsto \left(\{1,\cdots,n\} \ni i \mapsto \left(F \ni k \mapsto kf(\alpha_i) \in W\right) \in \hom(F,W)\right) \in \bigoplus_{i=1}^{n} \hom(F,W)$
					
				(即满足$\phi (f)(i)(k)=kf(\alpha_i)$的唯一映射)
						
				接下来验证$\phi $是同构映射

				首先,容易注意到$\phi (f)=\phi (g) \Rightarrow f=g$,因此$\phi $是单射

				其次,$\forall \left(\{1,\cdots,n\} \ni i \mapsto g_i \in \hom(F,W)\right) \in \bigoplus_{i=1}^{n} \hom(F,W)$

				注意到,如果设$f:V \ni \sum\limits_{i=1}^{n} k_i \alpha_i \mapsto \sum\limits_{i=1}^{n} k_i g_i(1_F)$,那么$\phi (f)(i)=g_i$,于是$\phi $是满射

				接下来验证它是一个线性映射

				$\forall f,g \in \hom(V,W),\phi (f+g)(i)(k)=k(f+g)(\alpha_i)=kf(\alpha_i)+kg(\alpha_i)=\phi (f)(i)(k)+\phi (g)(i)(k) \Rightarrow \phi (f+g)=\phi (f)+\phi (g)$

				$\forall f \in \hom(V,W),l \in F,\phi (lf)(i)(k)=k(lf)(\alpha_i)=lkf(\alpha_i)=l\phi (f)(i)(k) \Rightarrow \phi (lf)=l\phi (f)$

				因此它是一个同构映射,命题得证。
			\end{proof}
			\begin{lemma}{}{}
				$\hom(F,W) \cong W$
			\end{lemma}
			\begin{proof}
				任取$V$的一个基$\{\alpha_1,\cdots,\alpha_n\}$,我们定义:

				$\psi : \alpha \in W \mapsto \left(F \ni k \mapsto k\alpha \in W\right) \in \hom(F,W)$

				注意到:$\psi  (\alpha )=\psi  (\beta ) \Rightarrow \psi  (\alpha )(k)=\psi  (\beta )(k) \Rightarrow k\alpha =k\beta \Rightarrow \alpha =\beta $,于是$\psi  $是单射

				$\forall f \in \hom(F,W),\psi  (f(1_F))(k)=kf(1_F)=f(k)\Rightarrow \psi  (f(1_F))=f$,于是$\psi $是满射

				$\forall \alpha ,\beta \in W,\psi (\alpha +\beta )(k)=k(\alpha +\beta )=k\alpha +k\beta =\psi (\alpha )(k)+\psi (\beta )(k) \Rightarrow \psi (\alpha +\beta )=\psi (\alpha )+\psi (\beta )$

				$\forall \alpha \in W,l \in F,\psi (l\alpha )(k)=lk\alpha =l\psi (\alpha )(k) \Rightarrow \psi (l\alpha )=l\psi (\alpha )$

				于是$\psi $是一个同构映射。
			\end{proof}
			\begin{lemma}{}{}
				$|\hom(V,W)|=|W|^{\dim V}$
			\end{lemma}
			\begin{proof}
				事实上,线性映射只由基上的作用决定,因此如果设$V$的一个基为$B$

				$|\hom(V,W)|=\card \{f:B \to W\}=|W|^{|B|}=|W|^{\dim V}$
			\end{proof}
			\begin{lemma}{}{}
				$\dim \hom_F (V,W) \geqslant |F|$
			\end{lemma}
			\begin{proof}
				
			\end{proof}
			\begin{lemma}{}{}
				设$U$是$F$上的一个线性空间,那么$|U|=|F|\cdot \dim_F U$
			\end{lemma}
			\begin{proof}
				
			\end{proof}
			\begin{them}{$\hom_F (V_1,V_2)$的维数}{}
				$\dim \hom_F (V,W)=\begin{cases}
					0, \dim V = 0 \vee \dim W=0 \\
					(\dim V)\cdot (\dim W),\dim V = n < \aleph_0 \\
					|\hom(V,W)|=|W|^{\dim V} = (|F|\cdot \dim W)^{\dim V},\dim \geqslant \aleph_0
				\end{cases}$
			\end{them}
			\begin{proof}
				首先考虑$V,W$中存在零空间的情形。

				如果$\dim V=0$,即$V=\{\mathbf{0}_{V}\}$,而我们知道线性映射仅能把零向量映射到零向量,此时$\hom(V,W)$中仅存在零映射,维数为零

				如果$\dim W=0$。即$W=\{\mathbf{0}_W\}$,此时映射只有零映射(因为$W$中仅存在零向量),维数为零

				接下来考虑$V,W$中不存在零空间,且$\dim V = n < \aleph_0$的情形。

				此时,$\dim \hom(V,W) =\dim \bigoplus_{i=1}^{n} \hom(F,W) = n \cdot \dim \hom(F,W) = n \cdot \dim W = (\dim V)\cdot (\dim W)$

				最后我们证明$\dim V,\dim W$均是无限的情形。


			\end{proof}
	\ifx\allfiles\undefined
\end{document}
\fi