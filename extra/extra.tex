\ifx\allfiles\undefined
\documentclass[12pt, a4paper, oneside, UTF8]{ctexbook}
\setCJKmainfont{SimSun}
\def\path{../config}
\input{../config/_config}
\begin{document}
	% \input{../config/cover}
	\else
	\fi
	%标题
	\chapter{附录}
	这一部分中,对于正文中因为逻辑结构无法提及的部分,进行补充。
	%--------------------正文---------------------------
		\section{$\hom(V,W)$的维数}
			这篇附录中,我们解决一个问题:$\hom(V,W)$的维数是多少。
			\begin{lemma}{}{}
				$\hom(V,W) \cong \bigoplus_{i=1}^{n} \hom(F,W)$
			\end{lemma}
			\begin{proof}
				任取$V$的一个基$\{\alpha_1,\cdots,\alpha_n\}$,我们定义:
						
				$\phi : \hom(V,W) \ni f \mapsto \left(\{1,\cdots,n\} \ni i \mapsto \left(F \ni k \mapsto kf(\alpha_i) \in W\right) \in \hom(F,W)\right) \in \bigoplus_{i=1}^{n} \hom(F,W)$
					
				(即满足$\phi (f)(i)(k)=kf(\alpha_i)$的唯一映射)
						
				接下来验证$\phi $是同构映射

				首先,容易注意到$\phi (f)=\phi (g) \Rightarrow f=g$,因此$\phi $是单射

				其次,$\forall \left(\{1,\cdots,n\} \ni i \mapsto g_i \in \hom(F,W)\right) \in \bigoplus_{i=1}^{n} \hom(F,W)$

				注意到,如果设$f:V \ni \sum\limits_{i=1}^{n} k_i \alpha_i \mapsto \sum\limits_{i=1}^{n} k_i g_i(1_F)$,那么$\phi (f)(i)=g_i$,于是$\phi $是满射

				接下来验证它是一个线性映射

				$\forall f,g \in \hom(V,W),\phi (f+g)(i)(k)=k(f+g)(\alpha_i)=kf(\alpha_i)+kg(\alpha_i)=\phi (f)(i)(k)+\phi (g)(i)(k) \Rightarrow \phi (f+g)=\phi (f)+\phi (g)$

				$\forall f \in \hom(V,W),l \in F,\phi (lf)(i)(k)=k(lf)(\alpha_i)=lkf(\alpha_i)=l\phi (f)(i)(k) \Rightarrow \phi (lf)=l\phi (f)$

				因此它是一个同构映射,命题得证。
			\end{proof}
			\begin{lemma}{}{}
				$\hom(F,W) \cong W$
			\end{lemma}
			\begin{proof}
				任取$V$的一个基$\{\alpha_1,\cdots,\alpha_n\}$,我们定义:

				$\psi : \alpha \in W \mapsto \left(F \ni k \mapsto k\alpha \in W\right) \in \hom(F,W)$

				注意到:$\psi  (\alpha )=\psi  (\beta ) \Rightarrow \psi  (\alpha )(k)=\psi  (\beta )(k) \Rightarrow k\alpha =k\beta \Rightarrow \alpha =\beta $,于是$\psi  $是单射

				$\forall f \in \hom(F,W),\psi  (f(1_F))(k)=kf(1_F)=f(k)\Rightarrow \psi  (f(1_F))=f$,于是$\psi $是满射

				$\forall \alpha ,\beta \in W,\psi (\alpha +\beta )(k)=k(\alpha +\beta )=k\alpha +k\beta =\psi (\alpha )(k)+\psi (\beta )(k) \Rightarrow \psi (\alpha +\beta )=\psi (\alpha )+\psi (\beta )$

				$\forall \alpha \in W,l \in F,\psi (l\alpha )(k)=lk\alpha =l\psi (\alpha )(k) \Rightarrow \psi (l\alpha )=l\psi (\alpha )$

				于是$\psi $是一个同构映射。
			\end{proof}
			\begin{lemma}{}{}
				$|\hom(V,W)|=|W|^{\dim V}$
			\end{lemma}
			\begin{proof}
				事实上,线性映射只由基上的作用决定,因此如果设$V$的一个基为$B$

				$|\hom(V,W)|=\card \{f:B \to W\}=|W|^{|B|}=|W|^{\dim V}$
			\end{proof}
			\begin{lemma}{}{}
				$\dim \hom_F (V,W) \geqslant |F|$
			\end{lemma}
			\begin{proof}
				
			\end{proof}
			\begin{lemma}{}{}
				设$U$是$F$上的一个线性空间,那么$|U|=|F|\cdot \dim_F U$
			\end{lemma}
			\begin{proof}
				
			\end{proof}
			\begin{them}{$\hom_F (V_1,V_2)$的维数}{}
				$\dim \hom_F (V,W)=\begin{cases}
					0, \dim V = 0 \vee \dim W=0 \\
					(\dim V)\cdot (\dim W),\dim V = n < \aleph_0 \\
					|\hom(V,W)|=|W|^{\dim V} = (|F|\cdot \dim W)^{\dim V},\dim \geqslant \aleph_0
				\end{cases}$
			\end{them}
			\begin{proof}
				首先考虑$V,W$中存在零空间的情形。

				如果$\dim V=0$,即$V=\{\mathbf{0}_{V}\}$,而我们知道线性映射仅能把零向量映射到零向量,此时$\hom(V,W)$中仅存在零映射,维数为零

				如果$\dim W=0$。即$W=\{\mathbf{0}_W\}$,此时映射只有零映射(因为$W$中仅存在零向量),维数为零

				接下来考虑$V,W$中不存在零空间,且$\dim V = n < \aleph_0$的情形。

				此时,$\dim \hom(V,W) =\dim \bigoplus_{i=1}^{n} \hom(F,W) = n \cdot \dim \hom(F,W) = n \cdot \dim W = (\dim V)\cdot (\dim W)$

				最后我们证明$\dim V,\dim W$均是无限的情形。


			\end{proof}
	\ifx\allfiles\undefined
\end{document}
\fi