\ifx\allfiles\undefined
\documentclass[12pt, a4paper, oneside, UTF8]{ctexbook}
\def\path{../config}
\input{../config/_config}
\begin{document}
% \input{../config/cover}
\else
\fi
%标题
\chapter{内积空间}
	\section{内积空间的定义}
		\subsection{内积空间的定义}
			\begin{defn}{共轭}{}
				设$F$是一个域,自同构$\sigma : F \to F$如果满足:

				\begin{equation}
					\forall a \in F,\sigma^2(a)=a
				\end{equation}
				那么我们称$\sigma $是一个共轭映射,$\sigma (a)$称为$a$的共轭,也记作$\overline{a}$。
			\end{defn}
			\begin{defn}{内积空间}{}
				设$F$是一个定义了共轭和偏序的域,$V$是一个$F$上的线性空间,映射$f:V \times V \to F$如果满足:

				\begin{itemize}
					\item $f(\alpha ,\beta )=\overline{f(\beta ,\alpha )}$(共轭对称性)
					\item $f(\alpha +\beta ,\gamma )=f(\alpha ,\gamma )+f(\beta ,\gamma )$(对第一个变量可加)
					\item $f(k\alpha ,\beta )=kf(\alpha ,\beta )$(对第一个变量线性)
					\item $f(\alpha ,\alpha )\geqslant 0,f(\alpha ,\alpha )=0 \Leftrightarrow \alpha =\mathbf{0}$(正定性)
				\end{itemize}
				那么我们称$f$是$V$上的一个内积,此时称$(F,V,+,\cdot,f)$是一个内积空间(我们也简称$V$是一个内积空间)

				习惯上,我们也常常将$f(\alpha ,\beta )$记作$\langle \alpha ,\beta \rangle_W $。如果我们仅在一个线性空间中讨论,我们也常常简写为$\langle \alpha ,\beta \rangle $。
			\end{defn}
			基于抽象内积空间,我们可以提出一些相关的概念:
			\begin{defn}{正交}{}
				设$V$是一个内积空间,$\langle \alpha ,\beta \rangle =0$,
				
				那么我们称$\alpha $与$\beta $是正交的,记作$\alpha \perp \beta $
			\end{defn}
			如果一个基是相互正交的,那么它被称为正交基:
			\begin{defn}{正交基}{}
				设$V$是一个内积空间,$V$的一个基$B$如果满足:

				$\forall \alpha ,\beta \in B,\alpha \neq \beta ,\langle \alpha ,\beta \rangle =0$,
				
				那么我们称$B$是$V$的一个正交基。
			\end{defn}
			我们特别考虑一种特殊的”补空间“,称为正交补:
			\begin{defn}{正交补}{}
				设$V$是$F$上的一个内积空间,$S \subseteq V$。我们定义:

				\begin{equation}
					S^{\perp}=\{\alpha \in V | \forall \beta \in S,\alpha \perp \beta \}
				\end{equation}
				称为$S$的正交补。
			\end{defn}
			值得注意的是,上述定义中,我们并不要求$S$是一个子空间;我们也将看到,就算$S$是子空间,正交补和$S$的直和也不一定是整个空间。

			接下来我们讨论两类重要的内积空间
		\subsection{实内积空间、复内积空间}
			\begin{defn}{实内积空间}{}
				考虑$\R$及其上的平凡自同构$\sigma(x)=x$

				此时,实数域$\R$上的内积空间,称为实内积空间。

				特别地,如果此时内积空间还是有限维的,我们称之为Euclidean空间
			\end{defn}
			由定义可知,实内积空间拥有以下独特的特性:不仅仅是对单变量线性,而是对双变量线性,而且完全对称。也就是说,它是一个对称双线性映射;特别地,由于实数域是全序集,因此任意向量的内积都是可比的。
			\begin{defn}{复内积空间}{}
				考虑$\C$及其上的共轭映射$\sigma(a+bi)=a-bi$

				复数域$\C$上的内积空间,称为复内积空间(或酉空间)
			\end{defn}
			实内积空间和复内积空间的独特特点是:可以定义范数,并导出度量,进而产生拓扑
			\begin{defn}{内积空间中的范数}{}
				设$V$是$F$上的一个内积空间,$F=\R$或者$F=\C$,$V$上的自然范数$\|\cdot \|:V \to \R$定义为:
				\begin{equation}
					\|\alpha \|=\sqrt{\langle \alpha ,\alpha \rangle }
				\end{equation}
			\end{defn}
			我们这里特意要求基域只能是实数或复数域是因为:在一般域上,不能保证向量与自身的内积可以定义良好的平方根。
			
			在实内积空间和酉空间中,如果正交基的每一个向量的范数都是$1$,我们也称之为标准正交基:
			\begin{defn}{标准正交基}{}
				设$V$是$F$上的一个内积空间,$F=\R$或者$F=\C$,$V$的一个正交基$B$如果满足:

				$\forall \alpha \in B,\|\alpha \|=1$

				那么我们称$B$是$V$的一个标准正交基。
			\end{defn}

			此时,内积空间转变为一个线性赋范空间;进而,我们可以用范数定义度量:
			\begin{defn}{内积空间中的度量}{}
				设$V$是$F$上的一个内积空间,$F=\R$或者$F=\C$,$V$上的自然度量$d:V \times V \to F$定义为:
				\begin{equation}
					d(\alpha ,\beta )=\|\alpha -\beta \|
				\end{equation}
			\end{defn}
			进而,我们可以定义极限和连续:
			\begin{defn}{内积空间中向量列的极限}{}
				设$V$是一个实内积空间或酉空间,$\{\alpha_i\}$是$V$中的一个向量列,如果存在一个向量$\alpha \in V$,使得:
				\begin{equation}
					\forall \varepsilon > 0,\exists N \in \N_+,\forall n > N,d(\alpha_n,\alpha ) < \varepsilon 
				\end{equation}
				我们称向量列$\{\alpha_i\}$收敛于$\alpha $,记作$\lim_{n \to \infty}\alpha_n=\alpha $
			\end{defn}
			\begin{defn}{连续线性泛函}{}
				设$V,W$分别是一个实内积空间或酉空间,$f \in \hom(V,W)$,如果有:
				\begin{equation}
					\forall \{\alpha_i\},\lim_{n \to \infty}\alpha_n=\lim_{n \to \infty}f(\alpha_n)
				\end{equation}
				那么我们称$f$是一个连续线性泛函
			\end{defn}
			进而可以定义完备性:
			\begin{defn}{Hillbert空间}{}
				设$V$是一个实内积空间或酉空间,如果$V$中符合下面条件的任意向量列$\{\alpha_i\}$
				\begin{equation}
					\forall \varepsilon > 0,\exists N \in \N_+,\forall m,n \geqslant N,d(\alpha_m,\alpha_n) < \varepsilon 
				\end{equation}
				收敛。

				那么,我们称$V$是完备的,并称它是一个Hillbert空间。
			\end{defn}
			我们也可以定义开集和闭集
			\begin{defn}{开集和闭集}{}
				设$V$是一个实内积空间或酉空间,$S \subseteq V$,如果对于任意$\alpha \in S$,存在$r > 0$,使得:
				\begin{equation}
					B(\alpha ,r)=\{\beta \in V | d(\alpha ,\beta ) < r\} \subseteq S
				\end{equation}
				那么我们称$S$是一个开集。

				如果$V-S$是一个开集,那么我们称$S$是一个闭集。
			\end{defn}
		\subsection{内积空间的性质}
			我们讨论一些性质。
			\begin{para}{0}
				\point{}
					首先,我们讨论内积的线性性质。
					\begin{proposition}
						$\langle \alpha ,\beta + \gamma \rangle =\langle \alpha ,\beta \rangle +\langle \alpha ,\gamma \rangle$

						$\langle \alpha ,k\gamma \rangle =\overline{k}\langle \alpha ,\gamma \rangle$
					\end{proposition}
					\begin{proof}
						$\langle \alpha ,\beta + \gamma \rangle =\overline{\langle \beta+\gamma ,\alpha  \rangle}=\overline{\langle \beta ,\alpha  \rangle}+\overline{\langle \gamma ,\alpha  \rangle}=\langle \alpha ,\beta \rangle +\langle \alpha ,\gamma \rangle$

						$\langle \alpha ,k\gamma \rangle =\overline{\langle k\gamma ,\alpha  \rangle}=\overline{k}\overline{\langle \gamma ,\alpha  \rangle}=\overline{k}\langle \alpha ,\gamma \rangle$
					\end{proof}
					接下来我们考察一种特别的内积,它只定义在实内积空间或酉空间,被称为标准内积:
				\point{}
					\begin{proposition}
						设$V$是$\R$上的一个$n$维线性空间,$\{\alpha_1,\alpha_2,\cdots,\alpha_n\}$是$V$的一个基,
						
						$\forall \alpha ,\beta \in V$,如果$\alpha =\sum_{i=1}^n a_i\alpha_i,\beta =\sum_{i=1}^n b_i\alpha_i$,那么我们说:

						\begin{equation}
							f(\alpha ,\beta)=\sum_{i=1}^n a_i b_i
						\end{equation}
						是$V$上的一个内积
					\end{proposition}
					\begin{proof}
						设$\alpha =\sum\limits_{i=1}^{n} a_i \alpha_i,\beta =\sum\limits_{i=1}^{n} b_i \alpha_i,\gamma =\sum\limits_{i=1}^{n} c_i \alpha_i$,那么:

						$f(\alpha ,\beta )=f(\sum\limits_{i=1}^{n} a_i \alpha_i,\sum\limits_{i=1}^{n} b_i \alpha_i )=\sum\limits_{i=1}^{n} a_i b_i=\sum\limits_{i=1}^{n} b_i a_i = f(\sum\limits_{i=1}^{n} b_i \alpha_i,\sum\limits_{i=1}^{n} a_i \alpha_i )=f(\beta ,\alpha )$
						
						$f(\alpha ,\beta +\gamma )=f(\sum\limits_{i=1}^{n} a_i\alpha_i,\sum\limits_{i=1}^{n} (b_i + c_i) \alpha_i )=\sum\limits_{i=1}^{n} a_i (b_i + c_i)=\sum\limits_{i=1}^{n} a_i b_i + \sum\limits_{i=1}^{n} a_i c_i =f(\sum\limits_{i=1}^{n} a_i \alpha_i,\sum\limits_{i=1}^{n} b_i \alpha_i ) + f(\sum\limits_{i=1}^{n} a_i \alpha_i,\sum\limits_{i=1}^{n} c_i \alpha_i )=f(\alpha ,\beta )+f(\alpha ,\gamma )$

						$f(\alpha ,k\beta )=f(\sum\limits_{i=1}^{n} a_i \alpha_i,k\sum\limits_{i=1}^{n} b_i \alpha_i )=f(\sum\limits_{i=1}^{n} a_i \alpha_i,\sum\limits_{i=1}^{n} kb_i \alpha_i )=\sum\limits_{i=1}^{n} a_i (kb_i)=k\sum\limits_{i=1}^{n} a_i b_i = k f(\sum\limits_{i=1}^{n} a_i \alpha_i,\sum\limits_{i=1}^{n} b_i \alpha_i )=k f(\alpha ,\beta )$

						$\forall \alpha \neq \mathbf{0},f(\alpha ,\alpha )=f(\sum\limits_{i=1}^{n} a_i \alpha_i,\sum\limits_{i=1}^{n} a_i \alpha_i )=\sum\limits_{i=1}^{n} a_i^2>0$

						于是命题得证。
					\end{proof}
					酉空间情形类似如下:
				\point{}
					\begin{proposition}
						设$V$是$\C$上的一个$n$维线性空间,$\{\alpha_1,\alpha_2,\cdots,\alpha_n\}$是$V$的一个基,
						
						$\forall \alpha ,\beta \in V$,如果$\alpha =\sum_{i=1}^n a_i\alpha_i,\beta =\sum_{i=1}^n b_i\alpha_i$,那么我们说:

						\begin{equation}
							f(\alpha ,\beta)=\sum_{i=1}^n a_i \overline{b_i}
						\end{equation}
						是$V$上的一个内积
					\end{proposition}
					\begin{proof}
						设$\alpha =\sum\limits_{i=1}^{n} a_i \alpha_i,\beta =\sum\limits_{i=1}^{n} b_i \alpha_i,\gamma =\sum\limits_{i=1}^{n} c_i \alpha_i$,那么:

						$f(\alpha ,\beta )=f(\sum\limits_{i=1}^{n} a_i \alpha_i,\sum\limits_{i=1}^{n} b_i \alpha_i )=\sum\limits_{i=1}^{n} a_i \overline{b_i}=\sum\limits_{i=1}^{n} \overline{b_i}\overline{\overline{a_i}}=\overline{\sum\limits_{i=1}^{n} b_i\overline{a_i}}=\overline{f(\beta ,\alpha )}$

						$f(\alpha ,\beta +\gamma )=f(\sum\limits_{i=1}^{n} a_i\alpha_i,\sum\limits_{i=1}^{n} (b_i + c_i) \alpha_i )=\sum\limits_{i=1}^{n} a_i \overline{b_i + c_i} = \sum\limits_{i=1}^{n} a_i (\overline{b_i} + \overline{c_i}) = \sum\limits_{i=1}^{n} a_i \overline{b_i} + \sum\limits_{i=1}^{n} a_i \overline{c_i} = f(\alpha ,\beta ) + f(\alpha ,\gamma )$

						$f(\alpha ,k\beta )=f(\sum\limits_{i=1}^{n} a_i \alpha_i,k\sum\limits_{i=1}^{n} b_i \alpha_i )=f(\sum\limits_{i=1}^{n} a_i \alpha_i,\sum\limits_{i=1}^{n} (k b_i) \alpha_i )=\sum\limits_{i=1}^{n} a_i \overline{k b_i} = \overline{k} \sum\limits_{i=1}^{n} a_i \overline{b_i} = \overline{k} f(\alpha ,\beta )$

						$\forall \alpha \neq \mathbf{0},f(\alpha ,\alpha )=f(\sum\limits_{i=1}^{n} a_i \alpha_i,\sum\limits_{i=1}^{n} a_i \alpha_i )=\sum\limits_{i=1}^{n} a_i \overline{a_i} = \sum\limits_{i=1}^{n} |a_i|^2 > 0$

						于是命题得证。
					\end{proof}
					我们接下来探讨正交的性质。
				\point{}
					\begin{proposition}
						设$V$是一个内积空间,$S \subseteq V$,那么$S^{\perp}$是$V$的一个子空间
					\end{proposition}
					\begin{proof}
						取$\forall \alpha ,\beta \in S^{\perp},k\in F$

						依定义,有$\forall \eta \in S,\langle \alpha ,\eta \rangle=0,\langle \beta ,\eta \rangle=0$

						注意到:$\langle \alpha +\beta ,\eta \rangle=\langle \alpha ,\eta \rangle+\langle \beta ,\eta \rangle=0+0=0 \Rightarrow \alpha +\beta \in S^{\perp}$

						$\langle k\alpha ,\eta \rangle=k\langle \alpha ,\eta \rangle=0 \Rightarrow k\alpha \in S^{\perp}$

						于是命题得证。
					\end{proof}
				\point{}
					\begin{proposition}
						设$V$是一个内积空间,$V$的有限子集$S$如果不存在零向量,且向量两两正交,那么它线性无关
					\end{proposition}
					\begin{proof}
						设$S=\{\alpha_1,\cdots,\alpha_n\},\forall i \neq j,\langle \alpha_i,\alpha_j \rangle=0,\alpha_i \neq \mathbf{0}$

						考察线性组合$\sum\limits_{i=1}^{n} k_i \alpha_i = \mathbf{0}$

						对$\forall i \leqslant l,k_i = 0$作强归纳法。

						首先,$\mathbf{0}=\langle \sum\limits_{i=1}^{n} k_i \alpha_i,\alpha_1\rangle = k_1 \langle \alpha_1,\alpha_1 \rangle \Rightarrow k_1=0$,于是$i=1$时成立;

						现在假设$i \leqslant l$时成立,考察$l+1$时:

						$\mathbf{0}=\langle \sum\limits_{i=l+1}^{n} k_i \alpha_i,\alpha_{l+1}\rangle = k_{l+1} \langle \alpha_{l+1},\alpha_{l+1} \rangle \Rightarrow k_{l+1}=0$,于是$i=l+1$时成立。

						由强归纳法原理,$\forall i \leqslant n,k_i=0$,于是命题得证。
					\end{proof}
					它的直接推论是,正交基是一定存在的:
				\point{}
					\begin{proposition}
						任意有限维内积空间的正交基都是存在的。
					\end{proposition}
					\begin{proof}
						取$V$的一个基$\{\alpha_1,\cdots,\alpha_n\}$

						考察以下一系列线性组合:$\eta_i = \alpha_i - \sum\limits_{j=1}^{i-1} \frac{\langle \alpha _i ,\eta_j \rangle}{\langle \eta_j ,\eta_j \rangle} \eta_j$

						对归纳假设:$\forall i,j\leqslant k,i \neq j,\langle \eta_i,\eta_j \rangle = 0,\langle \eta_i,\eta_i \rangle \neq 0$使用数学归纳法。

						首先,当$k=1$,$\eta_1=\alpha_1$,归纳假设显然成立;

						我们现在假设$k$时成立,考察$k+1$时:

						$\forall j < k+1$

						$\langle \eta_{k+1},\eta_j \rangle = \langle \alpha_{k+1},\eta_j \rangle - \sum\limits_{i=1}^{k} \frac{\langle \alpha_{k+1},\eta_i \rangle}{\langle \eta_i,\eta_i \rangle} \langle \eta_i,\eta_j \rangle$

						$=\langle \alpha_{k+1},\eta_j \rangle - \frac{\langle \alpha_{k+1},\eta_j \rangle}{\langle \eta_j,\eta_j \rangle} \cdot \langle \eta_j,\eta_j \rangle = \mathbf{0}$

						$\langle \eta_j,\eta_{k+1} \rangle = \overline{\langle \eta_{k+1},\eta_j \rangle} = 0$,此时归纳假设成立。

						于是$\{\eta_i\}$的确是相互正交的;由有限正交向量组线性无关,以及基的性质,$\{\eta_i\}$是$V$的一个正交基。
					\end{proof}
				\point{}
					\begin{proposition}
						设$V$是$F$上的一个内积空间,$\{\eta_1,\cdots,\eta_n\}$是$V$的一个标准正交基,那么对于$\forall \alpha \in V$

						$\alpha = \sum\limits_{i=1}^{n} \langle \alpha , \eta_i \rangle \eta_i$
					\end{proposition}
					\begin{proof}
						因为$\{\eta_i\}$的确是一个基,因为一定能找到一系列系数$k_i$,使得$\alpha = \sum\limits_{i=1}^n k_i \eta_i $

						此时只需注意到:$\langle \alpha ,\eta_i \rangle = \langle \sum\limits_{j=1}^n k_j \eta_j ,\eta_i \rangle = k_i \langle \eta_i ,\eta_i \rangle = k_i$

						$\Rightarrow \alpha =\sum\limits_{i=1}^{n} \langle \alpha , \eta_i \rangle \eta_i$
					\end{proof}
				\point{}
					\begin{them}{Cauchy-Buniakowski-Schwarz不等式}{}
						设$V$是一个实内积空间或酉空间,$\alpha ,\beta \in V$,那么:
						\begin{equation}
							|\langle \alpha ,\beta \rangle | \leqslant \|\alpha \| \|\beta \|
						\end{equation}
						等号成立的充分必要条件是$\alpha $与$\beta $线性相关
					\end{them}
					\begin{proof}
						当$\alpha ,\beta $线性相关,要不$\alpha =\mathbf{0}$,要不$\beta =k\alpha $,此时不等式显然成立

						考察$\alpha ,\beta $线性无关的情形

						此时,一定有:$\forall k,\alpha -k\beta \neq\mathbf{0}$

						此时,$\langle \alpha -k\beta ,\alpha -k\beta \rangle > 0$

						$\Rightarrow \|\alpha \|^2-k \langle \alpha ,\beta \rangle-\overline{k}\langle \beta ,\alpha \rangle + |k|^2 \|\beta \|^2 > 0$

						代入$k = -\frac{\langle \beta ,\alpha \rangle}{\|\beta \|^2}$,得:

						$\|\alpha \|^2 - \frac{\overline{\langle \alpha ,\beta \rangle}\langle \alpha ,\beta \rangle}{\|\beta \|^2}-\frac{\langle\alpha ,\beta \rangle\overline{\langle \alpha ,\beta  \rangle}}{\|\beta \|^2} + \frac{|\langle \beta ,\alpha \rangle|^2}{\|\beta \|^2} > 0$

						$\Rightarrow \|\alpha \|^2 - \frac{|\langle \alpha ,\beta \rangle|^2}{\|\beta \|^2} > 0$

						$\Rightarrow |\langle \alpha ,\beta \rangle|^2 < \|\alpha \|^2 \|\beta \|^2$
						
						于是命题得证。
					\end{proof}
					上述定理适用于两类内积空间,尽管实内积空间并不能简单视为复内积空间的一个子空间,但是证明过程中利用的性质都是通用的,我们统一地写出这个证明。
				\point{}
					
					最后,我们讨论Hillbert空间上的性质
					\begin{proposition}
						设$V$是一个$F$上的Hillbert空间,$S \subseteq V$是$V$的一个闭子空间,那么有$S \oplus S^{\perp}=V$
					\end{proposition}
					\begin{proof}
						我们先证明$S + S^{\perp} = V$

						$\forall \alpha \in V$,考察函数$f(\theta )=\|\alpha -\theta \|,\theta \in S$

						由于$S$是闭集,那么$f$的最小值一定存在,取$\beta $使得$f(\beta )=\min_\theta  f(\theta )$,并记$\gamma =\alpha -\beta $

						我们来证明,的确有$\gamma \in S^{\perp}$

						$\forall \eta \in S$,我们希望证明$\langle \gamma ,\eta \rangle = 0$

						考察函数$g(t)=\|\alpha - (t\eta + \beta )\|^2,t \in \R$

						由于$t\eta +\beta \in S$,而$\|\alpha -\theta  \|$在$\theta =\beta $时取得最小值,因此,$g(t)$在$t=0$时取得最小值。

						显然,$g(t)=\|\alpha - (t\eta + \beta )\|^2=\|\gamma - t\eta \|^2=\|\gamma \|^2-2t\operatorname{Re}\langle \gamma ,\eta  \rangle+t^2\|\eta \|^2$连续可导

						于是,一定有$g'(0)=(-2\operatorname{Re}\langle \gamma ,\eta  \rangle+2t\|\eta \|^2)|_{t=0}=-2\operatorname{Re}\langle \gamma ,\eta  \rangle=0\Rightarrow \operatorname{Re}\langle \gamma ,\eta  \rangle=0$。

						若$F=\R$,那么$\langle \gamma ,\eta  \rangle=0$,论证完成;
						
						对于$F=\C$的情形,考虑$g(t)=\|\alpha - (t i \eta + \beta )\|^2,t \in \R$

						同理可得$\operatorname{Re}\langle \gamma ,i\eta  \rangle=0$。

						$\Rightarrow \operatorname{Re}(-i\langle \gamma ,\eta  \rangle)=0 \Rightarrow \operatorname{Re}(\operatorname{Im}\langle \gamma ,\eta  \rangle-i\operatorname{Re}\langle \gamma ,\eta  \rangle)=0\Rightarrow \operatorname{Im}\langle \gamma ,\eta  \rangle=0$。
					
						于是此时有$\langle \gamma ,\eta  \rangle=0$,论证完成。

						最后我们证明$S \cap S^{\perp}=\{\mathbf{0}\}$

						设$\alpha \in S\cap S^{\perp}$,那么$\langle \alpha ,\alpha \rangle = 0 \Rightarrow \alpha = \mathbf{0}$

						于是命题得证。
					\end{proof}
					利用这个结论可以证明以下重要事实:
					\begin{them}{Riesz表示定理}{}
						设$V$是一个$F$上的Hillbert空间,如果$f \in \hom(V,F)$是一个连续泛函,那么存在唯一的$\phi \in V$,使得:
						\begin{equation}
							f(\alpha ) = \langle \alpha,\phi  \rangle
						\end{equation}
					\end{them}
					\begin{proof}
							我们首先证明:$\ker f$是一个闭集。

							注意到:任意收敛序列$\{\alpha_n\} \subseteq \ker f$,因为$f$是连续的,因此有

							$f(\lim_{n \to \infty} \alpha_n)=\lim_{n \to \infty} f(\alpha_n)=\lim_{n \to \infty} 0=0 \Rightarrow \lim_{n \to \infty} \alpha_n \in \ker f$

							即$\ker f$对任意序列的极限封闭。

							接下来考察集合$\overline{\ker f}:=\{\lambda | \forall \lambda \in O,O\text{是开集},O \cap \ker f \neq \emptyset\}$

							注意到,一定有$\ker f \subseteq \overline{\ker f}$,因为$\ker f$中的任意一点,包含其的开集与$\ker f$的交集至少包含这个点本身。

							考察$\alpha \in \overline{\ker f}$

							取$\alpha_i \in B(\alpha ,\frac{1}{n}) \cap \ker f$,其中$B(\alpha ,s):=\{\beta |d(\alpha ,\beta )< s\}$,由$\overline{\ker f}$的定义可知序列是的确可以取得的。

							那么,此时一定有$\lim_{n \to \infty} \alpha_n = \alpha $,因为$0 < d(\alpha ,\alpha_k)<\frac{1}{k}$。

							但是,我们已经指出:$\ker f$对任意收敛序列的极限封闭,因此$\alpha \in \ker f$。那么,$\ker f = \overline{\ker f}$

							那么,我们说,$\ker f$是闭集。因为:$\ker f$已经包含了所有与开集有非空交集的点,那么对于$\forall \beta \in V-\ker f$,一定存在包含$\beta $的一个开集$O$,使得$O\in V-\ker f$,而这正是$\ker f$闭的定义。

							于是依照之前证明的结论,一定有$\ker f \oplus (\ker f)^{\perp} = V$

							于是$\im f \cong V/\ker f \im \cong (\ker f)^{\perp}$

							但是,$\im f \subseteq F \Rightarrow \dim \im f = 1$,因此$\dim (\ker f)^{\perp} = 1$

							取$(\ker f)^{\perp}$的一个标准正交基$\{\eta \}$,那么此时有:

							$f(\alpha )=\langle \alpha ,\eta \rangle f(\eta) =\langle \alpha ,\overline{f(\eta )}\eta \rangle$

							记$\phi = \overline{f(\eta )}\eta$,那么此时有$f(\alpha )=\langle \alpha ,\phi \rangle$。

							于是命题得证。
						\end{proof}
			\end{para}
	\section{正规算子和自伴算子}
		本节中,如无特别说明,我们假定所有线性空间都定义了内积。
		\subsection{伴随算子}
			我们先定义伴随算子,它是我们后续讨论的基础。
			\begin{defn}{共轭算子}{}
				设$V,W$是$F$上的两个内积空间,$A \in \hom(V,W)$,我们定义满足以下条件的算子$A^* : W \to V$
				\begin{equation}
					\langle A(\alpha) ,\beta \rangle_W =\langle \alpha ,A^*(\beta) \rangle_V
				\end{equation}
			\end{defn}
			我们先研究伴随算子的一些性质。
			\begin{para}{0}
				\point{}
					\begin{proposition}
						设$A \in \hom(V,W)$,那么$A^*$是唯一的。
					\end{proposition}
					\begin{proof}
						不妨假设命题不成立,那么存在$B,C : W \to V,B \neq C$,此时有:

						$\forall \alpha\in V,\beta \in W,\langle A\alpha ,\beta \rangle_W =\langle \alpha ,B\beta \rangle_V =\langle \alpha ,C\beta \rangle_V$

						显然,一定$\exists \gamma ,B(\gamma )\neq C(\gamma )$

						此时注意到:$\langle B(\gamma )-C(\gamma ),B(\gamma )-C(\gamma )\rangle_V = \langle B(\gamma ),B(\gamma )\rangle_V - \langle B(\gamma ),C(\gamma )\rangle_V =\mathbf{0} \Rightarrow B(\gamma )=C(\gamma )$,与假设矛盾。

						于是命题得证。
					\end{proof}
				\point{}
					\begin{proposition}
						$\forall A \in \hom(V,W),A^* \in \hom(W,V)$
					\end{proposition}
					\begin{proof}
						$\forall \alpha\in V,\beta \in W,k \in F$

						$\langle \alpha ,(A^*+B^*)(\beta ) \rangle_V =\langle A(\alpha ),\beta \rangle_W +\langle B(\alpha ),\beta \rangle_W =\langle \alpha ,A^*(\beta ) \rangle_V +\langle \alpha ,B^*(\beta ) \rangle_V =\langle \alpha ,A^*(\beta )+B^*(\beta ) \rangle_V$

						$\Rightarrow (A^*+B^*)(\beta )=A^*(\beta )+B^*(\beta )$

						$\langle \alpha ,(kA^*)(\beta ) \rangle_V =k\langle A(\alpha ),\beta \rangle_W =k\langle \alpha ,A^*(\beta ) \rangle_V =\langle \alpha ,kA^*(\beta ) \rangle_V$

						$\Rightarrow (kA^*)(\beta )=kA^*(\beta )$

						于是命题得证。
					\end{proof}
				\point{}
					\begin{proposition}
						$\forall A,B \in \hom(V,W),(A+B)^* = A^* + B^*$
					\end{proposition}
					\begin{proof}
						$\forall \alpha \in V,\beta \in W$

						$\langle \alpha ,(A+B)^*(\beta ) \rangle_V =\langle (A+B)(\alpha ),\beta \rangle_W =\langle A(\alpha ),\beta \rangle_W +\langle B(\alpha ),\beta \rangle_W =\langle \alpha ,A^*(\beta ) \rangle_V +\langle \alpha ,B^*(\beta ) \rangle_V =\langle \alpha ,A^*(\beta )+B^*(\beta ) \rangle_V$

						由于$\alpha ,\beta $是任意选取的,命题得证。
					\end{proof}
				\point{}
					\begin{proposition}
						$(kA)^*=\overline{k}A^*$
					\end{proposition}
					\begin{proof}
						$\forall \alpha\in V,\beta \in W$

						$\langle \alpha ,(kA)^*(\beta ) \rangle_V =\langle (kA)(\alpha ),\beta \rangle_W=k\langle A(\alpha ),\beta \rangle_W=k\langle \alpha ,A^*(\beta ) \rangle_V =\langle \alpha ,\overline{k}A^*(\beta ) \rangle_V$

						由于$\alpha ,\beta $是任意选取的,命题得证。
					\end{proof}
				\point{}
					\begin{proposition}
						$\forall A \in \hom(V,W), (A^*)^* = A$
					\end{proposition}
					\begin{proof}
						$\forall \alpha  \in V,\beta \in W$

						$\langle \alpha ,(A^*)^*(\beta ) \rangle_V =\langle (A^*)(\alpha ),\beta \rangle_W =\langle \alpha ,A(\beta ) \rangle_V$

						由于$\alpha ,\beta $是任意选取的,命题得证。
					\end{proof}
				\point{}
					\begin{proposition}
						$\forall A \in \hom(V,W), B \in \hom(U,V),(AB)^*=B^*A^*$
					\end{proposition}
					\begin{proof}
						$\forall \alpha \in U,\beta \in W$

						$\langle \alpha ,(AB)^*(\beta )\rangle_U = \langle (AB)(\alpha ),\beta \rangle_W = \langle B(\alpha ),A^*(\beta )\rangle_V = \langle \alpha ,(B^*A^*)(\beta )\rangle_U$

						由于$\alpha ,\beta $是任意选区的,命题得证。
					\end{proof}
				\point{}
					\begin{proposition}
						如果$A$可逆,那么此时$A^*$也可逆,并且有$(A^*)^{-1}=(A^{-1})^*$
					\end{proposition}
				\point{}
					\begin{proposition}
						设$A\in \hom(V,W)$,有以下结论:
						\begin{enumerate}
							\item $\ker A = (\im A^*)^{\perp}$
							\item $\im A = (\ker A^*)^{\perp}$
							\item $\ker A^* = (\im A)^{\perp}$
							\item $\im A^* = (\ker A)^{\perp}$
						\end{enumerate}
					\end{proposition}
			\end{para}
		\subsection{正规算子}
			先给出定义。
			\begin{defn}{正规算子}{}
				设$V$是$F$上的一个内积空间,$A \in \hom(V,V)$,如果$AA^*=A^*A$,那么我们称$A$是一个正规算子
			\end{defn}
	\section{保距算子、幺正算子}
	\section{酉算子}
	\section{奇异值与奇异值分解}
	\section{UR、QR、Schur分解}
\ifx\allfiles\undefined
\end{document}
\fi